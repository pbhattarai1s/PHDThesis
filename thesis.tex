% !TeX root = ./thesis.tex
\documentclass[red]{brandeis-dissertation}
\usepackage[utf8]{inputenc}
\usepackage{titletoc}
\usepackage{amsmath}
\usepackage{amssymb}
\usepackage{epigraph}\renewcommand{\epigraphflush}{center}
\usepackage{graphicx}
\usepackage{caption}
\usepackage{subcaption}
\usepackage{xspace}
\usepackage{multirow}
\usepackage{tocloft}
\usepackage{titlesec}
\usepackage{appendix}
\usepackage{float}
\usepackage{cite}
\usepackage[colorlinks = true,
            linkcolor = black,
            urlcolor  = blue,
            citecolor = blue,
            anchorcolor = black]{hyperref}
%\input{defs.tex}
\usepackage{AnalysisCommonDefs} %change this to input if it doesn't work
\numberwithin{equation}{section}
\graphicspath{figures/}
\usepackage{newclude}
%%Zach has a brandeis hack to remove all the large fonts and so on but i want to stick with the default for now maybe i will change in future
\titleformat{\part}[hang]{\normalfont\bfseries}{\partname\ \thepart:}{0.5em}{} 
%%%%
% Epigraph
\setlength\epigraphwidth{\textwidth}
\setlength\epigraphrule{0pt}
% Footnotes
\dimen\footins=20\baselineskip\relax
\interfootnotelinepenalty=10000
% Titles
\renewcommand{\cfttoctitlefont}{\LARGE\bfseries}
\renewcommand{\cftlottitlefont}{\LARGE\bfseries}
\renewcommand{\cftloftitlefont}{\LARGE\bfseries}
\titleformat*{\section}{\Large\bfseries}
\titleformat*{\subsection}{\large\bfseries}
\newcommand{\unnumsubsec}[1]{\subsection*{\normalsize{#1}}}
% Table of Contents
\setcounter{tocdepth}{2}
\setlength{\cftsubsecindent}{4em}
\renewcommand{\cftpartfont}{\large\bfseries}
\renewcommand{\cftpartpagefont}{\large\bfseries}
\renewcommand{\cftsecfont}{\normalsize}
\renewcommand{\cftsecpagefont}{\normalsize}
\renewcommand{\cftsubsecfont}{\normalsize}
\renewcommand{\cftsubsecpagefont}{\normalsize}

%Title of Thesis
%\title{First differential cross-sections measurement for $ZZ$ production in association with two jets in the four-leptons final state in ATLAS.}
\title{Standard Model Precision Measurements with two Z bosons and two jets in ATLAS.}
\author{Prajita Bhattarai}
\graduationmonth{May}
\graduationyear{2023}
\program{Department of Physics}
\advisor{Professor Gabriella Sciolla}
\signoff{Wendy Cadge}{Dean}
\committee{Professor Gabriella Sciolla, Department of Physics, Brandeis University \\
Professor Aram Apyan, Department of Physics, Brandeis University\\
Dr. Alessandro Tricoli, Brookhaven National Laboratory
}

\begin{document}

% Title
\maketitlepage
\clearpage
% Approval 
\makeapproval
\clearpage
% Copyright
\makecopyright
\clearpage
% Acknowledgements
\addcontentsline{toc}{section}{Acknowledgements}
\begin{center}
\textbf{\large{Acknowledgements}}
\end{center}
I am sincerely grateful to many people who have had a signifcant impact on my  journey. 
\clearpage


%Abstract
\phantomsection
\begin{dissertation-abstract}
\addcontentsline{toc}{section}{Abstract}
This thesis presents the cross-section measurements of two Z bosons production in association with two jets in $ZZ^*jj\rightarrow 4 \ell jj$ final state, differentially as a function of several kinematic observables, which are sensitive to the vector boson scattering. The electroweak $ZZjj$ production includes the rare triple and quartic self-couplings of gauge bosons, whose scattering amplitude is regularized by the Standard Model $H\rightarrow ZZ^{*}$ processes. Measuring differential cross-sections of the $ZZjj$ process in electroweak-enhanced phase space is imperative to probe the high-energy behavior of the vector boson scattering, where modifications are expected from several Beyond the Standard Model theories. The analysis is performed using the proton-proton collision data collected by the ATLAS experiment during LHC Run-2 at $\sqrt{s}=13$ TeV center-of-mass collision energy, corresponding to the integrated luminosity of $139~fb^{-1}$. The measured differential cross-sections are corrected for the detector effects and compared with the state-of-the-art Standard Model theoretical predictions. The differential cross-sections are also used to constrain the anomalous quartic gauge couplings using a dimension-8 Effective Field Theory formalism. 

\end{dissertation-abstract}
\clearpage

\doublespacing

% Table of Contents
\tableofcontents
\clearpage
\phantomsection
\addcontentsline{toc}{section}{\listtablename}
\listoftables
\clearpage
\phantomsection
\addcontentsline{toc}{section}{\listfigurename}
\listoffigures
\clearpage

\addcontentsline{toc}{section}{List of Abbreviations}
\begin{itemize}
\item{BSM: Beyond the Standard Model}
\item {C: Charge conjugation}
\item{EFT: Effective Field Theory}
\item{EWK: Electroweak}
\item{FSR: Final State Radiation}
\item{GRL: Good Run List}
\item{H: Weak Hypercharge}
\item{ I: Weak Isospin}
\item{$\mathcal{L_{SM}}$: Lagrangian}
\item{LB: Luminosity Block}
\item{LH:  Left Handed}
\item{MC: Monte Carlo}
\item{P: Parity}
\item{PDF: Parton Distribution Function}
\item{Q: Electric Charge}
\item{QGC: Quartic Gauge Coupling}
\item{QED: Quantum Electrodynamics}
\item{QCD: Quantum Chromodynamics}
\item{$(\mathcal{P})$: Poincare group}
\item{RH: RightHanded}
\item{SF-OC: Same-flavor, Opposite-charged} 
\item{SM: Standard Model}
\item{T: Time-reversal}
\item{TGC: Triple Gauge Coupling}
\item{TTVA: Track-to-vertex association}
\item{VBS: Vector Boson Scattering}
\item{VEV: Vacuum Expectation Value}
\end{itemize}
\clearpage

% sections

\startbody

\renewcommand{\partname}{Chapter}

\part{\LARGE{Introduction}}
\label{sec:Introduction}
Particle Physics investigates the fundamental nature of the universe. The Standard Model (SM), the fundamental theory of Particle Physics, provides a theoretical formulation that explains all known elementary particles, their interactions, and three of the four fundamental forces observed in nature: strong, electromagnetic, and weak forces. Fifty years after its formulation, the parameters predicted by the SM have been measured experimentally with high precision. The experimental discovery of the Higgs boson in 2012 established the SM as a complete and highly successful theory. However, the lack of description of the fourth fundamental force, gravity, and other experimentally evident phenomena, such as the existence of dark matter, suggest that the SM provides only an incomplete description of nature. Still, experimental evidence of new physics Beyond the Standard Model (BSM) has yet to be observed. The current primary objective of the Large Hadron Collider (LHC) at CERN is to look for experimental evidence of new physics which might explain or resolve some of the shortcomings of the SM. 

New physics searches can be broadly categorized into two types, direct and indirect. The direct search focuses on finding experimental evidence of new physics signatures, such as new particles. In contrast, the indirect approach focuses on precisely measuring the parameters of the SM-predicted processes, looking for deviations compared to the state-of-the-art theoretical predictions. One critical phenomenon of the SM is the vector boson scattering (VBS) in final states involving vector bosons, which are force-mediating particles for the electroweak (EWK) force. VBS processes are mediated by the rare triple and quartic self-couplings between the vector bosons, whose SM amplitudes interfere destructively with the Higgs-mediated processes. Several BSM theories modify either the strength of the vector boson self-couplings or that of the Higgs-mediated processes, thus, altering the extent of the interference and, consequently, the cross-sections from the predicted values. As many new physics particles are expected to exist at high energies, such deviations are expected to appear at higher energies that have not been probed experimentally yet. 

This thesis presents an indirect approach to new physics searches in one of the VBS-sensitive multiboson final states. The measurement analyzes the data collected by the ATLAS experiment at the LHC from 2015-2018 to measure the VBS-sensitive production of two $Z$ bosons in association with two jets, where each $Z$ boson decays into a pair of same-flavor opposite-charge (SF-OC) leptons. The quartic self-coupling of the vector bosons in $ZZ^*(\rightarrow 4\ell)jj$ final state is experimentally accessible with the collected LHC dataset for the first time. The measurement uses a model-independent approach to look for anomalous quartic gauge couplings. Thus, measurements presented in this thesis are at the frontier of particle physics, pushing the boundaries of new physics searches through an indirect approach. 

The theory of SM, its shortcomings, and the $ZZ^*(\rightarrow 4\ell)jj$ process are discussed in Chapter $II$. The LHC and ATLAS experiments are then introduced in Chapter $III$. Chapters $IV$ and $V$ discuss the details of the measurement. The final results are presented in Chapter $VI$. 

\clearpage

\part{\LARGE{Theory}}
\label{sec:theory}

This chapter describes the theoretical framework of the experimental measurements discussed in this thesis. Section  \ref{sec:SM} introduces the SM of particle physics and concepts relevant to the thesis. Section \ref{sec:SM_Incomplete} discusses the outstanding problems with the Standard Model, thus, motivating the experimental measurement. Section \ref{sec:Pheno} discusses the phenomenology of the proton-proton collisions, and Section \ref{sec:EWKPheno} discusses the physics of two $Z$ bosons production in an association of two jets. 
	\section{The Standard Model}

\label{sec:SM}
The SM of particle physics is a mathematical framework based on quantum field theory which incorporates quantum mechanics and special relativity. The SM describes all known fundamental particles in nature and their interactions. It consists of two sets of particles with intrinsic angular momentum, half-integer-spin fermions that are fundamental constituents of matter particles, and force-carrying integer-spin bosons. The seventeen fundamental particles of the SM and their properties such as mass, charge, and intrinsic spin are shown schematically by figure \ref{fig:SM}. Discussion in this section is written with the guidance from two textbooks on particle physics, Mark Thomson's Modern Particle Physics \cite{Thomson:2013zua}, and Halzen $\&$ Martin's Quarks $\&$ Leptons \cite{Halzen:1984mc}.


\subsection{Symmetries}
\label{subsec:Symmetries}
The fundamental particles of the SM and their interactions can be described by constructing a general renormalizable Lagrangian $(\mathcal{L}_{SM})$ that respects certain sets of given symmetries. The Lagrangian of the SM is independent of the reference frame, naturally respecting the complete external symmetries of special relativity, the Poincare group $(\mathcal{P})$. Thus, the SM is invariant under spacetime translations, boosts, and rotations. Additionally, by the construction of the Lagrangian, the SM respects an internal local gauge symmetry $SU(3)_{C}~\otimes~SU(2)_{L}~\otimes~U(1)_{Y}$. The $SU(3)_{C}$ symmetry is associated with the Quantum Chromodynamics (QCD) discussed in detail in Section \ref{subsubsec:QCD}. The $SU(2)_{L}~\otimes~U(1)_{Y}$ gauge symmetry discussed in \ref{subsubsec:EWkUni} is associated with the unified electroweak theory that combines Quantum Electrodynamics (QED) and the weak interactions. 

According to Noether's theorem, a quantity is conserved for each continuous transformation that leaves the Lagrangian invariant \cite{NoetherTheorem}. Several interesting physical quantum numbers are conserved as a consequence of the symmetries respected by the SM. The $SU(3)_{C}$ in QCD conserves color charge. Weak isospin (I) and weak hypercharge (Y) are the quantum numbers associated with the $SU(2)_{L}$ and $U(1)_{Y}$ gauge groups respectively. At low energies the $SU(2)_{L}~\otimes~U(1)_{Y}$ symmetry is spontaneously broken and will be discussed in Section \ref{subsubsec:EWkUni}. The $SU(2)_{L}$ group follows a chiral structure where the gauge fields couple explicitly to the left-handed (LH) chiral fermions states and the right-handed (RH) chiral anti-fermions states.

The SM also respects CPT symmetry, a combination of three additional discrete symmetries, charge conjugation (C), parity (P), and time-reversal (T). The charge-conjugation transformation transforms particles to anti-particles by reversing the quantum numbers, whereas, the parity transformation transforms left-handed particles to right-handed particles.


\subsection{Particles and Fields}
\label{subsec:Constituents}

\begin{figure}[H]
	\centering
    \includegraphics[width=0.7\textwidth] {figures/SMparticles.pdf}\hspace{1cm}
    \caption{ The seventeen fundamental particles of the SM include three generations of twelve fermions, four gauge bosons, and the scalar Higgs bosons. \cite{SMFigureWiki}}
    \label{fig:SM}
\end{figure}

The twelve half-integer-spin fermions can be distinguished further into two categories, leptons, and quarks, each having three generations of particles with similar properties as shown schematically by figure \ref{fig:SM}. For each fermion, there exists an anti-fermion with the same additive quantum numbers but with opposite signs. Four spin $1$ bosons shown in Table \ref{tab:VectorBosons} are collectively called the gauge bosons. Quanta of these gauge fields mediate the electromagnetic, weak, and strong interactions and are invariant under various local gauge transformations \cite{Bernabeu2021}. As summarized by Table \ref{tab:FermionInteraction}, fermions take part in different interactions. The strength of interaction is governed by a gauge coupling parameter.

Massless photon ($\gamma$) mediates the electromagnetic interaction. The massive $W$ and $Z$ bosons mediate weak interaction. The electric charge (Q) which is conserved in all interactions is related to the isospin and hypercharge by $Q=I_3 + \frac{Y}{2}$, where $I_3$ is the third component of the weak isospin. As a consequence of the chiral structure of $SU(2)_{L}$, each generation of fermion contains a left-handed doublet with $I_{3}=\pm\frac{1}{2}$ and a right-handed singlet carrying $I_{3}=0$ which are shown in Table \ref{tab:Fermions}. 

\begin{table}
\caption{Properties of SM gauge bosons.\cite{PDG}}
\begin{center}
\begin{tabular}{| c | c | c | c | c | c |}
\hline
\multicolumn{2}{|c|}{Interaction Type }	& Particle 		          & 	Q 		& 	Mass $[GeV]$ 		& Symmetry Group \\ 
\hline
\multirow{3}{*} {Electroweak}  & Electromagnetic 		& Photon ($\gamma$)      &   	   0                    & 	$0$	 			&  \multirow{3}{*}{$SU(2)~\otimes~U(1)$}		\\
					      &  \multirow{2}{*} { Weak }          		& $W^{\pm}$	&   $\pm1$	&	$80.4$	&		\\
    	  				       &	  &	$Z$ boson  			& $0$ 		    	         & 	          $91.2$			&   		  \\
\hline
\multicolumn{2}{|c|}{Strong } & gluons (g) &  0 & 0 & $SU(3)$ \\
\hline 
\end{tabular}
\label{tab:VectorBosons}
\end{center}
\end{table}

\begin{table}
\caption{ Summary of different interactions of fermions under different gauge theory. The check mark suggests that the fermions interact via associated force.}
\begin{center}
\begin{tabular}{| c | c | c |  c | c |}
\hline
\multicolumn{2}{|c|} {Particles} & Strong $SU(3)$ & Electromagnetic $U(1)$ & Weak $SU(2)$ \\
\hline
\hline
\multirow{2}{*} {Quarks} & $u, c, t$ &  \multirow{2}{*} \checkmark & \multirow{2}{*} \checkmark & \multirow{2}{*} \checkmark \\
  & d, s, b &  & &\\
\hline
\multirow{2}{*} {Leptons} & e, $\mu$, $\tau$ &  - &  \checkmark &  \checkmark \\
 & $\nu_{e}$, $\nu_{\mu}$, $\nu_{\tau}$ & - & \checkmark & - \\
\hline
\end{tabular}
\label{tab:FermionInteraction}
\end{center}
\end{table}

Each generation of lepton, electron $(e)$, muon $(\mu)$ and tau $(\tau)$ is accompanied by a neutral particle called neutrino $(\nu)$ with same lepton flavor $(\nu_e, \nu_{\mu} \& \nu_{\tau})$. The SM neutrinos are their own anti-particles and only left-handed neutrinos are predicted by the theory. The lepton flavor is conserved by the SM in all interactions.

The quarks can be further categorized into two categories, the up-type quarks with $+\frac{2}{3}$ charge and the down-type quarks with $-\frac{1}{3}$ charge. Up $(u)$, charm  $(c)~,~\&$ top $(t)$ are the first, second, and third generation of the up-type quarks, while the down $(d)$, strange $(s)$ $\&$ bottom $(b)$ are the three generations of the down-type quarks. The down-type left-handed quarks in $SU(2)_{L}$ quark doublets $d^{'},~s^{'}~ \&~b^{'}$ summarized in table \ref{tab:Fermions} are linear combinations of $d,~s,~b$ quarks. The quarks interact strongly with one another by strong interaction mediated by the massless neutral gluons which follow from $SU(3)$ gauge symmetry by exchange of color charges. Each quark can have either one of the three color charges (red, blue $\&$, green), whereas an anti-quark can have either an anti-red, anti-blue or anti-green color charge. There are eight gluons in the SM with color charges formed by a combination of either of the two color charges. Since gluons have a color charge, they interact with other gluons strongly. Only color neutral hadronic states formed by a combination of quarks and gluons are observed experimentally.

\begin{table}
\caption{Electroweak quantum numbers of the SM half-integer spin fermions (quarks and leptons) arranged in a left-handed $SU(2)$ doublet and right-handed $SU(2)$ singlet.\cite{Halzen:1984mc}}
\begin{center}
\begin{tabular}{| c | c | c | c | c | c | c |}
\hline
{Particle Types }			& First		& Second	&   Third        & 	$I_{3}$ 	& Y & Q  \\ 
\hline
\hline					
\multirow{5}{*} {Leptons}  	
	& & & & & & \\
					& $\begin{pmatrix}  e \\ \nu_{e} \end{pmatrix}_{L}$ 
					& $\begin{pmatrix}  \mu \\ \nu_{\mu} \end{pmatrix}_{L}$
					& $\begin{pmatrix}  \tau \\ \nu_{\tau} \end{pmatrix}_{L}$  
					& $\begin{matrix} -\frac{1}{2} \\[0.15cm] \frac{1}{2} \end{matrix}$ 
					& $\begin{matrix} -1 \\ -1 \end{matrix}$   
					& $\begin{matrix} -1 \\ 0 \end{matrix}$ \\		
	& & & & & & 	\\				
					& $e_{R}$ & $\mu_{R}$ &  $\tau_{R}$ & $0$ & $-2$  & $-1$ \\
\hline
\hline
\multirow{8}{*} {	Quarks}  	
& & & & & & \\
					 &$\begin{pmatrix}  u \\ d^{'} \end{pmatrix}_{L}$ 
					 &$\begin{pmatrix}  c \\ s^{'} \end{pmatrix}_{L}$
					 &$\begin{pmatrix}  t \\ b{'} \end{pmatrix}_{L}$  
					&$\begin{matrix} \frac{1}{2} \\[0.15cm] -\frac{1}{2} \end{matrix}$
					&$\begin{matrix} \frac{1}{3} \\[0.15cm] \frac{1}{3} \end{matrix}$
					&$\begin{matrix} \frac{2}{3} \\[0.15cm] -\frac{1}{3} \end{matrix}$\\
				& & & & & & \\
					& $u_{R}$ & $c_{R}$ &  $t_{R}$ & $0$ & $\frac{4}{3}$  & $\frac{2}{3}$ \\
					& & & & & & \\
					& $d_{R}$ & $s_{R}$ &  $b_{R}$ & $0$ & $-\frac{2}{3}$  & $-\frac{1}{3}$ \\
& & & & & & \\
\hline
\end{tabular}
\label{tab:Fermions}
\end{center}
\end{table}

Higgs boson is the only spin-0 scalar particle in the theory with no charge and gives masses to all other particles through Spontaneous Symmetry Breaking which will be discussed in Section \ref{subsubsec:EWkUni}.

\subsection{Theoretical Formulation of the Standard Model}
\label{subsec:TheoryFormulation}

\subsubsection{Relativistic Quantum Field Theory}
\label{subsubsec:RelQFT}
Relativistic quantum field theory is the theoretical framework of the SM that describes elementary particles and their interactions. This section introduces the framework. 

\subsubsection{Lagrangian of the Standard Model}
\label{subsubsec:SMLag}
The dynamics of the SM can be described by the Lagrangian density given in equation \ref{eqn:SMLagrangian} which is invariant under the local gauge transformation of the $SU(3)~\otimes~SU(2)_{L}~\otimes~U(1)_{Y}$ symmetry group. 
\begin{equation}
\mathcal{L_{SM}} = -\frac{1}{4}F_{\mu\nu}F^{\mu\nu} ~+~ i\bar{\psi}\gamma^{\mu}D_{\mu}\psi ~+~ |D_{\mu}\phi|^{2} ~+~ -V(\phi) + \bar{\psi_{i}}y_{ij}\psi_{j}\phi ~+~ h.c.
\label{eqn:SMLagrangian}
\end{equation}

The first term $-\frac{1}{4}F_{\mu\nu}F^{\mu\nu}$ describes the dynamics of the gauge boson interactions, the second term $i\bar{\psi}\gamma^{\mu}D_{\mu}\psi$ describes the interaction of the fermion fields. The third term $|D_{\mu}\phi|^{2}$ describes the couplings between the Higgs boson and gauge bosons, whereas the term $V(\phi)$ describes the Higgs potential and its self-interactions. The second last term $\bar{\psi_{i}}y_{ij}\psi_{j}\phi$ generates masses for fermions based on their Yukawa couplings $y_{ij}$ to the Higgs field. Similarly, the last term $h.c.$ generates masses for antifermions. 

\subsubsection{Quantum Electrodynamics}
\label{subsubsec:QED}
Quantum electrodynamics describes electromagnetic interaction. The Lagrangian density ($\mathcal{L}_{Dirac}$) describes the free propagation of a fermion in a vacuum as:  

\begin{equation}
\mathcal{L}_{Dirac} = \bar{\psi} i \gamma^{\mu} \partial_{\mu} \psi ~-~ m\bar{\psi}\psi
\label{eqn:DiracLag}
\end{equation},

where $\psi$ is the fermionic spinor, $\gamma^{\mu}$ represents the Dirac matrices with $\mu$ being the Lorentz index running from $0$ to $3$, $\partial^{\mu}$ is the covariant derivative and m is the mass of the fermion. 

The Lagrangian in equation \ref{eqn:DiracLag} is invariant under a $U(1)$ global gauge transformation, 
\begin{equation}
\psi\rightarrow \psi^{'}=e^{iq\alpha}\psi, 
\label{eqn:QEDGlobalTrans}
\end{equation}

where q is a parameter of the transformation itself and $\alpha$ is a real phase factor. However, under the local gauge transformation of form 

\begin{equation}
\psi\rightarrow \psi^{'}=e^{iq\alpha(x)}\psi
\label{eqn:QEDLocalTrans}
\end{equation}

which $\alpha$ depends on $x~=~(x_{0},x_{1},x_{2},t)$ the Dirac Lagrangian in equation \ref{eqn:DiracLag} is not invariant. 

To make the Lagrangian of equation \ref{eqn:DiracLag} invariant, a gauge field $A_{\mu}$ is introduces with the following transformation properties, 

\begin{equation}
A_{\mu}\rightarrow A_{\mu} - \partial _{\mu} \alpha
\label{eqn:QEDGaugeField}
\end{equation}  

The $A_{\mu}$ couples to fermionic fields $\psi(x,t)$ with strength q. A covariant derivative specific to the local gauge transformation is defined as:

\begin{equation}
D_{\mu} = \partial_{\mu} - iqA_{\mu}
\label{eqn:QEDCovDerv}
\end{equation}  

The quantity $q$ can be interpreted as the electric charge $-e$ of fermion which gives the coupling strength of QED. With these substitutions, the Dirac Lagrangian in equation \ref{eqn:DiracLag} changes to following

\begin{equation}
\mathcal{L} = \bar{\psi} ( i \gamma^{\mu} D_{\mu} ~-~ m) \psi
\label{eqn:QEDInvLag}
\end{equation},

which is invariant under $U(1)$ gauge transformation respecting the $U(1)$ gauge symmetry. 

The gauge field $A_{\mu}$ can be interpreted as the photon field and is related to the electromagnetic field tensor by

\begin{equation}
F_{\mu\nu} = \partial_{\mu}A_{\nu} - \partial_{\nu}A_{\mu}
\label{QEDFieldTensor}
\end{equation}

The gauge invariant kinetic term of photon $-\frac{1}{4}F_{\mu\nu}F^{\mu\nu}$ can be inserted into the Lagrangian in equation \ref{eqn:QEDInvLag} which gives us the full Lagrangian of QED invariant under $U(1)$ gauge transformation. 

\begin{equation}
\mathcal{L}_{QED} = -\frac{1}{4}F_{\mu\nu}F^{\mu\nu} + \bar{\psi} ( i \gamma^{\mu} D_{\mu} ~-~ m) \psi
\label{eqn:QEDFullLag}
\end{equation}

$\mathcal{L}_{QED}$ in equation \ref{eqn:QEDFullLag} is the full Lagrangian for QED and the electromagnetic phenomena can be derived by solving for the equations of motion applying the Lorentz gauge condition $\partial_{\mu}A^{\mu}=0$. 

\subsubsection{Quantum Chromodynamics }
\label{subsubsec:QCD}

Interaction between the quarks is defined by Quantum Chromodynamics requiring $SU(3)$ gauge transformation on the quark field with color charge $j$ (red, blue, or green).
 
The Dirac Lagrangian for a quark can be modified to include all possible colors of quark field $q_{j}$ as

\begin{equation}
\mathcal{L} = \bar{q_{j}}(i\gamma^{\mu}\partial_{\mu} - m )q_{j}
\label{eqn:QCDStartL}
\end{equation}

The generators of the $SU(3)$ group are eight linearly independent traceless Gell-Mann matrices that do not commute with each other such that 

\begin{equation}
[ T_{a},T_{b} ] = if_{abc}T_{c}
\label{eqn:SU3GellManMat}
\end{equation}

where $f_{abc}$ is the structure constant of $SU(3)$

The local $SU(3)$ gauge transformation is 

\begin{equation}
q(x) \rightarrow e^{i \alpha_a(x) T_{a}} q(x)
\label{eqn:QCDSU3LT}
\end{equation}

where $T_{a} = \frac{\lambda_{a}}{2}$, and $a = {1,2...8}$. To understand the source of gauge invariance in the Lagrangian in equation \label{eqn:SU3LocalGaugeTransform}, we can consider an infinitesimal transformation of the color field as 

\begin{equation}
q(x) \rightarrow [ 1 + i\alpha_{a}(x)T_{a}]q(x) \\
\ni ~ \partial_{\mu}q \rightarrow ( 1 + i\alpha_{a}T_{a})\partial_{\mu}q + iT_{a}q\partial_{\mu} \alpha_{a}
\end{equation}

The last term $iT_{a}q\partial_{\mu} \alpha_{a}$ breaks the gauge invariance. Similar to QED, eight gauge fields corresponding to each $a = {1,2...8}$ $G_{\mu}^{a}$ with following transformation properties are introduced 

\begin{equation}
G_{\mu}^{a} \rightarrow G_{\mu}^{a} - \frac{1}{g_{s}} \partial_{\mu} \alpha_{a} - f_{abc}\alpha_{b}G^{c}_{\mu}
\label{eqn:SU3GaugeField}
\end{equation}

These gauge fields $G_{\mu}^{a}$ are the gluon fields. Similar to QED the covariant derivative is defined as
\begin{equation}
D_{\mu} = \partial_{\mu} + ig_{s}\frac{\lambda_{a}}{2}G_{\mu}^{a} 
\label{eqn:SU3CovDerv}
\end{equation}

where $g_{s}$ is the coupling strength of the gluon fields to the quark fields.

The Lagrangian density in equation \ref{eqn:QCDStartL} is then 

\begin{equation}
\mathcal{L} = \bar{q_{j}}(i\gamma^{\mu}D_{\mu} - m )q_{j}
\label{eqn:QCDInvLag}
\end{equation}

Similar to QED, a gauge-invariant kinetic term 
$-\frac{1}{4}G^{a}_{\mu\nu}G^{\mu\nu}_{a}$, 

dependent on the field strength tensor $G^{a}_{\mu\nu}$ is added to equation \ref{eqn:QCDInvLag} to give the full QCD Lagrangian. The kinetic terms allow self-interaction within the gluon fields which is an important feature of QCD. $G^{a}_{\mu\nu}$ is the field strength tensor defined as

\begin{equation}
G^{a}_{\mu\nu} = \partial_{\mu}G^{a}_{\nu} - \partial_{\nu}G^{a}_{\mu} - g_{s}f_{abc}G^{b}_{\mu}G^{c}_{\nu}
\label{eqn:QCDFullLag}
\end{equation}

Therefore, the complete $SU(3)$ gauge invariant Lagrangian describing the quarks and gluons interaction is
\begin{equation}
\mathcal{L}_{QCD} = \bar{q_{j}}(i\gamma^{\mu}D_{\mu} - m )q_{j} -\frac{1}{4}G^{a}_{\mu\nu}G^{\mu\nu}_{a} 
\label{eqn:QCDCompleteLag}
\end{equation}

\subsubsection{Electroweak Theory}
\label{subsubsec:EWkUni}
Weak interactions describe the interactions mediated by massive gauge bosons. The first formulation of the weak interaction was made by Fermi in $1934$, to explain the beta decay using four fermion interaction vertex. The formulation successfully describes the beta decay at low energies when the interaction energy is much smaller than the $W$ boson mass. A unified electroweak theory was formulated by Glashow in $1961$ \cite{GLASHOW1961579}, by extending the $SU(2)$ symmetric non-Abelian gauge theory formulated by Yang and Mills in $1954$ \cite{PhysRev.96.191} to $SU(2)~\otimes~U(1)$ gauge theory. Above the unification threshold, the differences in the electromagnetic and weak interactions are not apparent.

Experimental observations suggest that weak interactions violate parity by only affecting the left-handed fermion fields and the right-handed anti-fermion fields. Thus the unified electroweak theory are described by $SU(2)_{L}~\otimes~U(1)_{Y}$ gauge interactions. Similar to the electric charge $Q$ conserved in QED by $U(1)$ symmetry, the weak hypercharge ($Y=2(Q-I_{3})$) related to the electric charge and the weak isospin $I_{3})$ is conserved by the $U(1)_{Y}$ symmetry. The fermion fields are represented by the left-handed doublets $\chi_{L}$ and the right-handed singlets $\psi_{R}$, introduced in table \ref{tab:Fermions}. The doublet and singlet field for the first generation of leptons and quarks are, 

\begin{center}
$ \chi_{L} = \begin{pmatrix} \nu_{e} \\ e\end{pmatrix}_{L}$ \hspace{5pt} $\&$ \hspace{5pt} $ \chi_{L} = \begin{pmatrix}  u \\ d \end{pmatrix}_{L}$ \\
\vspace{10pt}
$ \psi_{R} = e_{R}$ \hspace{10pt} $\&$ \hspace{5pt} $\psi_{R} = u_{R} ~\&~ d_{R}$
\end{center}
The Lagrangian for these fermion fields should be invariant under local gauge transformation corresponding to both $SU(2)_{L}$ and $U(1)_{Y}$ symmetry as, 

\begin{equation}
\chi_{L} \rightarrow e^{i\beta(x)Y+i\alpha_{a}(x)\tau_{a}} \chi_{L}
\label{eqn:SU2LHTransform}
\end{equation}

\begin{equation}
\psi_{R} \rightarrow e^{i\beta(x)Y} \psi_{R}
\label{eqn:SU2RHTransform}
\end{equation}

where, $\beta(x)$ and $\alpha(x)$ are the local phase transformation for $U(1)_{Y}$ and $SU(2)_{L}$ symmetry groups respectively. Weak hypercharge operator $Y$ and Pauli matrices $\tau_{a=1,2,3}$ are generators of $U(1)_{Y}$ and $SU(2)_{L}$ symmetry groups respectively. Similar to the formulation in QED and QCD discussed in Section \ref{subsubsec:QED} and \ref{subsubsec:QCD}, four new field strength tensors $B_{\mu\nu}$ and $W^{a}_{\mu\nu}$ corresponding to respectively the $U(1)_{Y}$ and $SU(2)_{L}$ transformations are introduced. The  $SU(2)_{L}~\otimes~U(1)_{Y}$ gauge-invariant Lagrangian for a massless fermion and massless gauge fields is:

\begin{equation}
\mathcal{L}_{0} = \bar{\chi_{L}}\gamma^{\mu} [i\partial_{\mu} - g \frac{\tau_{a}}{2} W^a_{\mu} + \frac{g^{'}}{2} B_{\mu} ] \chi_{L} + \bar{\psi_{R}} \gamma^{\mu} [ i \partial_{\mu} + g^{'} B_{\mu} ] \psi_{R} - \frac{1}{4} W_{\mu\nu}^{a} W^{\mu\nu}_{a} - \frac{1}{4} B_{\mu\nu} B^{\mu\nu}
\label{eqn:EWKLagrangian1}
\end{equation}

where similar to QED and QCD, the field strength tensors are defined in terms of the covariant derivative to preserve the gauge-invariance in the kinetic terms as,

\begin{equation}
B_{\mu\nu} = \partial_{\mu}B_{\nu} - \partial_{\nu}B_{\mu}
\label{eqn:U1YFST}
\end{equation}

\begin{equation}
W_{\mu\nu}^{a} = \partial_{\mu}W_{\nu}^{a} - \partial_{\nu}W_{\mu}^{a} + g\epsilon^{abc}W_{\mu}^{b}W_{\nu}^{c}
\label{eqn:SU2FST}
\end{equation}

The non-Abelian part of the $SU(2)_{L}$ transformation is represented by the last term of equation \ref{eqn:SU2FST} which gives the quartic and triple self-interactions between the gauge bosons with coupling strength $g$. 

The electroweak Lagrangian in equation \ref{eqn:EWKLagrangian1} contains two terms, one of which gives rise to the charged-current interaction with the two $SU(2)$ boson field 

\begin{equation}
W^{\pm}_{\mu} ~=~ \frac{ W^{1}_{\mu} \mp iW^{2}_{\mu} } {\sqrt(2)}
\label{eqn:RealWBosons}
\end{equation}
via exchange of the $W^{\pm}$ bosons and the neutral current interactions from the two neutral gauge boson fields $W^{3}_{\mu}$ and $B_{\mu}$. 

The Lagrangian for the charged-current interaction for the first generation of quarks and leptons is, 

\begin{equation}
\mathcal{L}_{CC} = \frac{g}{2\sqrt{2}} \{ W^{\dagger}_{\mu} [\bar{u}\gamma^{\mu}(1-\gamma_{5})d + \bar{\nu_{e}}\gamma^{\mu}(1-\gamma_{5})e ]~+~h.c \}
\label{eqn:SU2CCLag}
\end{equation}

The $SU(2)_{L}$ charged-current interaction Lagrangian for the next two generations follows the same, establishing the universality of the quark and lepton interactions as a direct consequence of the gauge symmetry.

The neutral-current Lagrangian is given by, 
\begin{equation}
\mathcal{L}_{NC} = \sum_{j}{ \bar{\psi_{j}} \gamma^{\mu} \{ A_{\mu} [ g \frac{\tau_{3}}{2} sin\theta_{W} + g^{'} Y cos\theta_{W} ] + Z_{\mu} [ \frac{\tau_{3}}{2} cos\theta_{W} - g^{'} Y sin\theta_{W}] \} \psi_{j} }
\label{eqn:SU2NCLag}
\end{equation}

where the two neutral gauge fields $Z_{\mu}$ and $A_{\mu}$ associated with $Z$ boson and photon governing the weak neutral and electromagnetic interactions are obtained from an arbitrary linear combination of the $W^{3}_{\mu}$ and $B_{\mu}$ fields as 

\begin{equation}
\begin{pmatrix} A_{\mu} \\ Z_{\mu} \end{pmatrix} =  \begin{pmatrix} cos{\theta_{W}} & sin{\theta_{W}} \\ -sin{\theta_{W}} & cos{\theta_{W}} \end{pmatrix} \begin{pmatrix} B_{\mu} \\ W^{3}_{\mu} \end{pmatrix}
\label{eqn:NeutralGaugeBosons}
\end{equation}


The following condition needs to be imposed to get QED from $A_{\mu}$:

\begin{equation}
g sin\theta_{W} = g^{'} cos\theta_{W} = e ~\& ~  Y= Q - T_{3}
\label{eqn:QEDFromEWk}
\end{equation}
, where $T_{3}=\frac{\tau_{3}}{2}$ is the weak isospin and $\theta_{W}$ is the Weinberg mixing angle, which can be measured experimentally and expressed in terms of the two $SU(2)_{L}$ coupling $g^{'}$ and $U(1)_{Y}$ coupling $g$ as:

\begin{equation}
sin\theta_{W} = \frac{g^{'}}{\sqrt{g^{2} +  g^{'2} }} ~\&~ cos\theta_{W} = \frac{g}{\sqrt{g^{2} +  g^{'2} }}
\label{eqn:WeinbergAngle}
\end{equation}

The Lagrangian in equation \ref{eqn:EWKLagrangian1} describes the electroweak interactions only for massless fermions and massless gauge bosons, which contradicts the experimental observations. The mass origin of the fermions and gauge bosons is discussed in Section \ref{subsubsec:HiggsMech} below. 

\subsubsection{Higgs Mechanism}
\label{subsubsec:HiggsMech}

Massive gauge bosons in the Lagrangian \ref{eqn:EWKLagrangian1} can be accommodated through the Brout-Englert-Higgs (BEH) mechanism, by introducing a complex scalar field $\phi$ in the spinor representation of $SU(2)_{L}$ doublet as \cite{HiggsMechanism},

\begin{equation}
\phi = \begin{pmatrix} \phi^{+} \\ \phi^{0} \end{pmatrix}
\end{equation}

A new term in the SM Lagrangian $\mathcal{L}_{BEH}$ depending on this scalar field can be defined as, 

\begin{equation}
\mathcal{L}_{BEH}  = (D_{\mu} \phi)^{\dagger} ( D^{\mu} \phi) - \mu^2 \phi^{\dagger} \phi + \lambda (\phi^{\dagger} \phi)^2
\label{eqn:LagBEH}
\end{equation}

The first term $(D_{\mu} \phi)^{\dagger} ( D^{\mu} \phi)$ describes the kinematic of the new fields and the BEH potential $V(\phi)$ is given by the second term as, 

\begin{equation}
V(\phi) = \lambda (\phi^{\dagger} \phi)^2 - \mu^2 \phi^{\dagger} \phi
\label{eqn:HiggsPot}
\end{equation}

The term $\lambda (\phi^{\dagger} \phi)^2$ describes the quartic self-interactions of the scalar fields and $\lambda > 0 $ is imposed by the vacuum stability. 

For $\mu^2 > 0$, the scalar field develops a nonzero Vacuum Expectation Value (VEV) which spontaneously breaks the symmetry. Due to the symmetry of $V(\phi)$ an infinite number of degenerate states exists with the potential $v$ only depending on the combination of $\phi^{\dagger}\phi$ 
\cite{PeskinQFT} with minimum energy satisfying $\phi^{\dagger}\phi = \frac{v^2}{2}$. This minimum energy requirement reduces one of the four degrees of freedom of the complex scalar field $\phi$. The three remaining degrees of freedom can be eliminated by a gauge transformation. We can choose $\phi$ by eliminating the upper component and the imaginary part of the lower component of the complex scalar field as,

\begin{equation}
<\phi> = \frac{1}{\sqrt{2}}\begin{pmatrix} 0 \\ v+H(x) \end{pmatrix},~ \hspace{10pt} ~;,~ H(x) = H^{*}(x) 
 \frac{1}{\sqrt{2}} \begin{pmatrix} 0 \\ v \end{pmatrix}
\label{eqn:ScalarExp}
\end{equation}

where the Higgs field ($H$) emerges as the excitation from the vacuum state. This choice of the minimum, spontaneously breaks the gauge symmetry \cite{DESYHiggsLecture}. 

After substituting the $\phi$ in the Lagrangian in equation \ref{eqn:LagBEH}, the kinetic term takes the form

\begin{equation}
\begin{array}{l}
\mathcal{L}_{BEH~Kinetic}  = \frac{\lambda}{2}v^{4} \\
\hspace{25pt}  +\frac{1}{2} \partial_{\mu}H \partial^{\mu}H - \lambda v^{2}H^{2} + \frac{\lambda}{\sqrt{2}} v H^{3} + \frac{\lambda}{8} H^4  \\
\hspace{25pt} + \frac{1}{4} ( v +\frac{1}{\sqrt{2} } H)^2 (W_{\mu}^{1} \hspace{10pt} W_{\mu}^2 \hspace{10pt} W_{\mu}^3 \hspace{10pt} B_{\mu} ) \begin{pmatrix} g^2 & 0 & 0 & 0 \\ 0 & g^2 & 0 & 0 \\  0  & 0 & g^2 & gg^{'} \\ 0  & 0 & gg^{'} & g^{2} \end{pmatrix} \begin{pmatrix} W^{1\mu}\\ W^{2\mu} \\ W^{3\mu} \\ B^{\mu} \end{pmatrix}
\end{array}
\label{eqn:LagBEHKin}
\end{equation}

where, the first line is the vacuum energy density and can be ignored in the case of QFT. The second line describes the triple and quartic self-interactions of the Higgs field as well as the mass term of the real scalar field H as $m_{H} = 2\lambda v^2$. The last line contains the mass term for the vector bosons. 

From equations \ref{eqn:LagBEHKin} and \ref{eqn:RealWBosons} is evident the mass of the two charged vector bosons $W^{\pm}$ is $m_{W}=\frac{1}{2}g^2v^2$. Similarly, from equations \ref{eqn:LagBEHKin} and 
\ref{eqn:NeutralGaugeBosons}, mass of the $Z$ boson is $m_{Z} = \frac{1}{2}(g^2+g^{'})v^2$ and mass of the photon is $m_{\gamma}=0$. 

The initial $SU(2)_{L}$ Lagrangian in equation \ref{eqn:LagBEH} started with four gauge symmetries, which is reduced to a single $U(1)_{Q}$ gauge symmetry associated with the massless vector field in equation \ref{eqn:LagBEHKin}. This phenomenon in the Higgs mechanism is called the Electroweak Symmetry Breaking (EWSB) mechanism. As discussed above, the EWSB mechanism is at the heart of the SM by which the gauge boson gets the mass which also arises the longitudinal polarization of the massive vector bosons. This thesis summarizes a measurement that has an experimental sensitivity to a such important property of the theory.

The last remaining piece in the SM is adding the fermion mass to the Lagrangian. A simple Lagrangian with the fermion mass can be written as, 

\begin{equation}
\mathcal{L}_{mass~fermion} = -m(\bar{\chi_{L}}\psi_{R} + \bar{\psi_{R}}\chi_{L})
\label{eqn:FermMass}
\end{equation}

This term violates $SU(2)_{L}$ gauge symmetry because the left-handed fermions are doublets and the right-handed are singlets. Adding a scalar complex field $\phi =\frac{1}{\sqrt{2}} \begin{pmatrix} 0 \\ v+ H(x) \end{pmatrix}$ in the Lagrangian becomes, 

\begin{equation}
\mathcal{L}_{Yukawa,~\ell} = \frac{G_{\ell}v}{\sqrt{2}} (\bar{\chi_{L}}\psi_{R} + \bar{\psi}_{R}\chi_{L} ) - \frac{G_{\ell}}{\sqrt{2}} (\bar{\chi_{L}}\psi_{R} + \bar{\psi}_{R}\chi_{L} )H
\label{eqn:YukawaLepMass}
\end{equation}

with arbitrary parameters $G_{\ell =e,\mu,\tau}$. The constant in the first term $\frac{G_{\ell}v}{\sqrt{2}}$ represents the mass of the fermions, whereas the second term gives the interaction of fermions with the Higgs field. 

Similarly, the mass terms for quarks follow but including the down-type quarks, the parameters corresponding to $G_{\ell}$ are matrices $G^{ij}_{q}$ for the quark families $i,j$ and up-type and down-type quarks as:

\begin{equation}
\mathcal{L}_{Yukawa,~Q} = -G^{ij}_{d}(\bar{u}_{i} , \bar{d}_{i} )_{L} \phi d_{jR} - G^{ij}_{u}(\bar{u}_{i} , \bar{d}_{i} )_{L} \phi u_{jR} + h.c.
\label{eqn:YukawaQuarkMass}
\end{equation}

The final Standard Model Lagrangian is the sum of the QED (equation \ref{eqn:QEDFullLag}), QCD (equation \ref{eqn:QCDInvLag}), Boson self-interactions (equation \ref{eqn:EWKLagrangian1}), Higgs potential and self-interactions ( equation \ref{eqn:LagBEH}), and the Higgs-fermion Yukawa coupling (equations \ref{eqn:YukawaLepMass} $\&$ \ref{eqn:YukawaQuarkMass}), which in a compact form is written in equation \ref{eqn:SMLagrangian}.

	\section{Limitations of the Standard Model}	
\label{sec:SM_Incomplete}

The predictions of the Standard Model have been experimentally validated by many discoveries since the $20^{th}$ century. The breakthrough discovery of the Higgs boson in 2012 at the LHC validated the last piece of the theory \cite{CMSHiggsDiscovery}$\&$ \cite{ATLASHiggsDiscovery}. Many predicted parameters such as production cross-sections and decay branching ratios for several processes have been measured with high precision. No statistically significant discrepancy from theory has been observed except for the $W^{\pm}$ boson mass measurement from the CDF $II$ Collaboration \cite{CDFWMass}.

Despite the incredible success of the theory, experimental evidence suggests that the theory is incomplete. SM has the following limitations:

\begin{itemize}

\item{SM fails to explain the gravitational force.}

\item{SM only describes $5\%$ of the universe. It fails to explain dark matter whose existence is experimentally supported by astrophysical observations such as galactic rotation curves and gravitational lensing \cite{DMGravitationalLensing}. It also doesn't explain dark energy. }

\item{The CP violation allowed in SM cannot explain the amount of anti-matter asymmetry observed in the universe. }

\item{ The strengths of the four fundamental forces are different by many orders of magnitude. It is not yet understood the hierarchy of such interactions.}
\end{itemize}
 
These limitations suggest that the SM is an effective field theory, only describing an approximation of our universe. Thus, motivating the experimental searches for new physics beyond the Standard Model (BSM).

	\section{Phenomenology of Proton-Proton Collisions }    
\label{sec:Pheno}

The main results discussed in this thesis are differential cross-sections for di-Z boson production in association with two jets in a proton-proton collider at the center of mass energy of $\sqrt{s}=13$ TeV. The differential cross-section measured gives the production probability of two $Z$ bosons and two jets as a function of their kinematic properties resulting from $p$-$p$ interactions at a given LHC collision energy and luminosity.

Protons are composite particles made up of quarks and gluons. Collisions happen between different constituents of the protons, \textit{partons}. The partons carry only a fraction of the total momentum $x_{i}$; thus, a partonic cross-section is $\sqrt{\hat{s}} = \sqrt{sx_1x_2}$. Figure \ref{fig:ColliderPheno} schematically shows the di-Z boson production in association with two jets from $p$-$p$ collision. The parton interaction that produces the physics of interest ($ZZ^*jj$) with large momentum transfer is \textit{hard scattering}. The additional partons of the two protons that interact in the hard interaction process lead to minor energy deposits in the detector referred to as \textit{underlying events}. Since the $p$-$p$ collision happens in bunches, other protons interact, leaving energy deposits in the detector, which are referred to as \textit{pile-up}.

\begin{figure}[!htb]
\centering
    \includegraphics[width=0.7\textwidth] {figures/Theory/ColliderPheno.pdf}\hspace{1cm}
    \caption{Phenomenology of di-Z boson production in association with two jets in a proton-proton collider.}
\label{fig:ColliderPheno}
\end{figure}

The differential cross-section $d\sigma$ for two particles is given by:
\begin{equation}
d\sigma  = \frac{{|\mathcal{M}|}^2}{F} dQ,
\label{eqn:DiffxS}
\end{equation}
where $F$ is the incident flux, and $dQ$ represents the Lorentz invariant phase space factor. The scattering amplitude $\mathcal{M}$ is the matrix element calculated from the Lagrangian density of the SM using a perturbative expansion \cite{QCDForCollider}.

The cross-section of a hard scattering process with two initial-state protons $p_{1}$ and $p_{2}$ producing the final state $X$ is given by:

\begin{equation}
d\sigma_{p_{1}p_{2} \rightarrow X } = \int dx_{1} dx_{2} \sum_{q_{1},q_{2}} f_{q_{1}}(x_{1},\mu_{F})f_{q_{2}}(x_{2},\mu_{F}) d\sigma_{q_{1}q_{2}\rightarrow X } (x_{1},x_{2},\mu_{F},\mu_{R}),
\label{eqn:DifferentialPartonicXS}
\end{equation}
where $q_{1}$ and $q_{2}$ are the partons of the protons, and $d\sigma_{q_{1}q_{2}\rightarrow X } (x_{1},x_{2},\mu_{F},\mu_{R})$ is the partonic cross-section. The functions $f_{q_{1}}(x_{1},\mu_{F})$ and $f_{q_{2}}(x_{2},\mu_{F})$ are the parton distribution functions (PDF) representing the density of the partons q inside a proton carrying the longitudinal momentum fraction $x$.

The partonic cross-section is calculated perturbatively as an expansion in terms of the strong coupling constant $\alpha_{S}$ as,
\begin{equation}
\label{eqn:PartonicXS}
d\sigma_{q_{1}q_{2}\rightarrow X} = \alpha_{S}^{k} \sum_{m=0}^{n} c_{m}\alpha_{S}^{m}.
\end{equation}
The coefficient $c_{m}$ depends on the center-of-mass energy, and theoretical calculations usually contain a finite number of coefficients. Leading order (LO) calculations include one term ($n=0$), whereas next-to-leading order (NLO) and next-to-next-to-leading order (NNLO) contains two ($n=1$) and three ($n=2$) terms, respectively. The theoretical calculations relevant to this thesis are generally calculated at NLO. 

The higher-order terms in the series contain additional virtual loop\footnote{virtual loop corrections consists of radiation of gluon from a quark which is absorbed internally by the same or different quark. Thus, creating a loop of radiation with additional QCD vertex. The particles in virtual loop corrections do not appear in the final state.} corrections and real emissions\footnote{real emission consists of adding an additional QCD vertex through radiation of quarks or gluons. Particles from real emissions appear in the final state.} of quarks and gluons. When calculating scattering amplitudes for processes involving virtual loops beyond the LO, the integration over the loop momentum can result in singularities. These \textit{ultraviolet singularities} are associated with the high momentum particle that participates in the virtual loops. The divergences are controlled via the renormalization procedure, where the singularities are absorbed by the redefinition of the strong coupling constant $\alpha_S$ to keep the predictions finite. The renormalization process is energy-dependent, and the predicted cross-sections from theoretical calculations depend on an energy-dependent \textit{renormalization scale $\mu_{R}$}. Moreover, additional \textit{infrared singularities} arise from real emissions of soft or collinear gluons. The infrared divergences either cancel out as predicted by the Kinoshita-Lee-Nauenberg theorem \cite{SoftCollinear} or are fixed by introducing the \textit{factorization scale $\mu_{F}$}, where the PDFs and fragmentation functions are redefined. Thus, any finite order prediction of cross-section depends on two energy scales $\mu_{R}$ and $\mu_{F}$. The scale dependence is reduced when higher order terms in the perturbative series are included and vanishes when including all perturbative orders. In practice, these scales are fixed to the energy scale of the process being evaluated. The residual scale dependence is taken as uncertainties on the predicted cross-sections, which are estimated by varying the values of $\mu_{R}$ and $\mu_{F}$ in Monte Carlo simulations.

The PDFs used in Equation \ref{eqn:DifferentialPartonicXS} are determined experimentally using data from deep-inelastic-scattering, jet production, and Drell-Yan events \cite{FixedTargetDrellYan} \cite{PDF4LHC}. As shown in Figure \ref{fig:PDFFig}, a PDF of a parton depends on the reference value of the momentum transfer $Q_{0}^2$. The differences are driven by modifications of partons' momenta resulting from the emission of gluons from quarks and the splitting of gluons to $q\bar{q}$ pairs. A PDF at any value of $Q^2$ can be calculated using the PDF at reference scale $Q_{0}^2$. The factorization scale $\mu_{F}$ determines the threshold whether the perturbative corrections modify the PDF or are included in the partonic cross-sections $d\sigma_{q_{1}q_{2}}$ \cite{QCDForCollider}.

\begin{figure}[!htb]
\centering
    \includegraphics[width=1\textwidth] {figures/Theory/PDF.png}\hspace{1cm}
    \caption{ Parton distribution functions $xf_{q}(x,Q^2)$ for reference momentum transfer $Q^2_{0} = 10 ~ GeV^2$ (left) and $Q^2_{0} = 10^4~ GeV^{2}$ (right)\cite{PDFFigure}.}
\label{fig:PDFFig}
\end{figure}

Any particles with a color charge involved in the interaction or produced during the hard scattering radiate gluons, which further emit QCD radiation forming showers of color particles, also known as \textit{parton shower}. During parton showering, the energy of each parton is split among the radiated particles. Below an energy scale named pole of the QCD running coupling ($\lambda_{QCD}$), the bounding potential of the strong force intervenes, and the partons are bound into a colorless state of stable and unstable hadrons. This process is named \textit{hadronization} and leads to the formation of collimated sprays of charged and neutral hadrons in the detector called \textit{jets}. The matrix element generating the hard-scattering process can describe a few hard QCD emissions. However, dedicated parton showering algorithms are used to describe multiple QCD emissions. The parton shower represents an approximate perturbative treatment of higher-order QCD corrections. The parton showering algorithm calculates dominant contributions associated with soft or collinear parton splitting. The hadronization process is simulated using non-perturbative models. There are two phenomenological models for hadronization, string and cluster models. String models are based on the assumption
that the potential energy between two quarks increases linearly as their spatial separation increases. The cluster model is motivated by considering central objects, hadrons, as a color-neutral cluster of quarks.

The theoretical predictions of an event shown in Figure \ref{fig:ColliderPheno} are calculated using Monte Carlo (MC) simulations which include matrix element calculations for hard scattering, the parton showering, the effect of the underlying events, hadronizations, and pile-up. A comprehensive overview of the methods used in MC simulation is discussed in Ref \cite{EventGenerator}.

	\section{ Electroweak Diboson Physics }	
\label{sec:EWKPheno}
\clearpage

\part{\LARGE{Experimental Setup}}
\label{sec:Experiment}

The European Organization for Nuclear Research, CERN, in Geneva, Switzerland, is home to the world's largest particle accelerator, the Large Hadron Collider (LHC). The measurements presented in this thesis correspond to the processes at the frontier of high-energy collisions. The relevant energy scale is only possible through large particle accelerators, giant detectors, and international collaboration. There are currently eight experiments analyzing the data from the LHC, among which ATLAS and CMS are the two largest multipurpose experiments. They analyze the collected data to perform SM precision measurements and search for new physics. This thesis analyzes the data collected by the ATLAS experiment between 2015-2018.

This chapter gives a description of the LHC in Section \ref{sec:LHC}, the ATLAS experiment in Section \ref{sec:ATLAS}, details on physics object reconstruction in Section \ref{sec:ParticleReconstruction}, and plans for future upgrades in Section \ref{sec:FutureUpgrades}. 
	\label{sec:LHC}

\section{ATLAS Detector}
\label{sec:ATLAS}
\section{ATLAS Detector}
\label{sec:ATLAS}

A Toroidal LHC ApparatuS (ATLAS) is a general-purpose detector of LHC that detects events from proton-proton and heavy ion collisions \cite{ATLAS}. It is a $44$ meters long, and $25$ meters wide cylindrical-shaped detector built around LHC Interaction Point 1 \cite{ATLAS}. ATLAS has multiple concentric sub-detectors layered around the beamline, providing forward-backward symmetric coverage. The two proton beams collide at the center of the detector producing outgoing particles from hard scattering, underlying events, and pile-up. The outgoing particles interact with the detector material leaving tracks and energy deposits in several layers of the sub-detectors. 

The sub-detector closest to the beamline is called \textit{Inner Detector (ID)}, which measures the trajectories of the charged particle and plays a crucial role in identifying the physical position of hard-scattering, also known as the \textit{interaction point (IP)}. ID is surrounded by a solenoid magnet that provides a $2$ T magnetic field to bend the particle trajectories for momentum measurements \cite{ATLAS}. After the solenoid magnet lies the \textit{electromagnetic calorimeter (ECAL)} and then the \textit{hadronic calorimeter (HCAL)}, which measure the energy of electromagnetic and hadronic physics objects, respectively. The outermost layer of the ATLAS detector is the \textit{Muon Spectrometer(MS)} that provides a secondary measure of muon trajectories for its momentum measurement. MS is embedded inside a toroidal magnetic field that provides a magnetic field up to 3.5 T \cite{ATLAS}. Figure \ref{fig:ATLAS} shows a schematic of the ATLAS detector with all its sub-detectors.

\begin{figure}
    \centering
    \includegraphics[width=.98\linewidth]{figures/LHC/AtlasDetector.png}
    \caption{ A detailed schematic of the ATLAS detector with all its sub-detectors \cite{ATLAS}.\label{fig:ATLAS}}
\end{figure}

\subsection{ATLAS Coordinate System}
\label{subsec:ATLASCS}

ATLAS measurements use a right-handed coordinate system with the nominal interaction point as the origin. The beamline is along the cylindrical symmetry axis of the detector, which defines the longitudinal \textit{z}-axis. The transverse \textit{xy}-plane is perpendicular to the beam direction, where \textit{x}-axis points to the center of the LHC ring and \textit{y}-axis points upwards towards the surface. Figure \ref{fig:ATLAS_CS} shows a schematic of the ATLAS coordinate system. The angle measured around the beamline in \textit{xy}-plane gives the azimuthal angle $\phi$, whereas the angle measured with respect to the \textit{z}-axis gives the polar angle $\theta$. Transverse momentum ($p_{T}$) is particle's momentum in the \textit{xy}-plane, defined as, 

\begin{equation}
p_{T} = \sqrt{p_{x}^2+p_{y}^2}=p\sin\theta
\label{eqn:pT}
\end{equation}
\textit{Rapidity (y)} defined in terms of a particle's energy ($E$) and momentum ($p$) is a commonly used collider physics quantity that measures whether an outgoing particle is produced perpendicular or parallel to the \textit{z}-axis. Rapidity is defined as, 

\begin{equation}
    y = \frac{1}{2}\ln{ \left( \frac{E+p_{z}}{E-p_{z}} \right) }
    \label{eqn:Rapidity}
\end{equation}
Particles with larger momentum along the \textit{z}-axis have larger values of rapidity, whereas particles with larger momentum values in the transverse plane have smaller values of rapidity. For particles with negligible mass, the rapidity approaches a purely angular variable called \textit{pseudorapidity ($\eta$)} defined as, 

\begin{equation}
    \eta = \frac{1}{2}\ln{ \left( \frac{ |\vec{p}|+p_{z}}{ |\vec{p}| -p_{z}} \right) } = -ln { \left[ \tan \left( \frac{\theta}{2}\right) \right] } 
    \label{eqn:PseudoRapidity}
\end{equation}
Higher values of rapidity and pseudorapidity refer to the forward region of the detector. ATLAS detector has full $2\pi$ coverage in $\phi$ and maximum coverage up to $|\eta| < 4.5$ corresponding to $1.3^{\circ} < \theta < 178.7^{\circ} $ \cite{ATLAS}. 

\begin{figure}
    \centering
    \includegraphics[width=.98\linewidth]{figures/LHC/ATLAS_CoordinateSys.png}
    \caption{ A schematic of the right-handed ATLAS coordinate system \cite{ATLAS_CoordSys}.\label{fig:ATLAS_CS}}
\end{figure}

\subsection{Inner Detector}
\label{subsec:ID}
The inner detector is the innermost sub-detector of ATLAS and is responsible for tracking charged particles' trajectories and identifying the interaction point of the hard scatter. Closest to the interaction point is the Insertable B-Layer (IBL) \cite{ATLAS_IBL}, which was installed during the long-upgrade shutdown between Run-1 and Run-2 to meet the requirements for competent tracking at higher pile-up. The IBL is highly granular, consisting of roughly 12 million silicon pixel sensors with a size of $50\times 250 ~\mu m^2$ \cite{ATLAS_IBL}. IBL is located $3.3$ cm from the beamline and can reconstruct tracks within the pseudorapidity range of $|\eta|<2.5$ \cite{ATLAS_IBL}. 

Three layers of silicon-pixel detectors with $1,744$ pixel sensors, each comprising $47,232$ pixels of size $50\times 400 ~\mu m^2$ surround the IBL \cite{ID_Pixel}. The slightly larger pixel size is adequate for the pile-up at a distance larger than $5$ cm from the interaction point. These pixel layers were also present during the Run-1 data-taking period and provided coverage up to $|\eta|<2.5$ with a spatial resolution of tracks between $5$ and 12 $\mu$m \cite{ID_Pixel}. Surrounding the pixel layers is the Semiconductor Tracker (SCT) consisting of five layers of silicon microstrip detectors with a mean strip pitch of $80 ~\mu m$ in the barrel region and varying pitch of  $57-94 ~\mu m$ in the end-cap regions \cite{ID_Strips}. 

At a distance about $50$ cm from the beamline lies the outermost layer of the ATLAS inner detector, the Transition Radiation Tracker (TRT), with $370,000$ straw tubes with a diameter of $4$ mm \cite{ID_TRT}. Each TRT straw tube is filled with a Xenon-based gas mixture and consists of $31~\mu m$ diameter tungsten wires \cite{ID_TRT}. A charged particle passing through different layers of ID leaves a track via ionization. 

Figure \ref{fig:ATLAS_ID} schematically shows different parts of the inner detector and their distances from the interaction point.

\begin{figure}
    \centering
    \includegraphics[width=.98\linewidth]{figures/LHC/ATLAS_InnerDetector.jpg}
    \caption{ A schematic of the inner detector of ATLAS showing the IBL, pixel detectors, SCT, and TRT \cite{ID_Align_Run2}.\label{fig:ATLAS_ID}}
\end{figure}

\subsection{Calorimeters}
\label{subsec:Cal}

ATLAS has two calorimeters, electromagnetic and hadronic, designed to measure the energy of charged and neutral particles up to the range of $|\eta|\leq4.9$ \cite{ATLAS}. When interacting with a material, an electron loses its energy by photon emission, which could produce a pair of $e^{+}e^{-}$, which could again radiate a photon, creating an electromagnetic shower in the detector. Similarly, the hadronic particles also result in a shower of particles through multiple scattering. The calorimeters measure the energy of the particles by reconstructing the electromagnetic and hadronic showers. The calorimeters are designed to absorb all particles except muons and neutrinos. Therefore, motivated by the need to prevent \textit{punch-through}\footnote{particles' probabilities of passing through the calorimeters} effect, materials with high radiation length ($X_{0}$) and high interaction length ($\lambda$) are chosen to construct the calorimeters.

Outside the solenoid magnet surrounding the ID is the accordion-shaped electromagnetic calorimeter consisting of an alternate layer of lead absorber plates and highly granular liquid-argon (LAr) cells to precisely measure the energies of electrons and photons. It comprise of barrel section in $|\eta| < 1.475$ range and two end-caps in $1.375 < |\eta| < 3.2$ range \cite{ATLAS_ECAL}. The calorimeter's central region ($|\eta| < 2.5$) is designed to identify electrons and photons with high precision.

The hadronic calorimeter surrounds the ECAL and consists of a steel absorber and active scintillator tiles in the $|\eta| < 1.7$ range. In the end-caps range of $1.5 < |\eta| < 3.2$, it consists of a copper absorber and active LAr detectors. The forward region ranging from $3.2 < |\eta| < 4$ comprises the tungsten absorber followed by active LAr detectors \cite{ATLAS_HCAL}. 

Figure \ref{fig:ATLAS_Cals} schematically shows the layout of ATLAS calorimeters. 

\begin{figure}
    \centering
    \includegraphics[width=.98\linewidth]{figures/LHC/ATLAS_CALO.jpeg}
    \caption{ A schematic of electromagnetic and hadronic calorimeters In ATLAS \cite{ATLAS}.\label{fig:ATLAS_Cals}}
\end{figure}

\subsection{Muon Spectrometer}
\label{subsec:MS}
In ATLAS, muons are deeply-penetrating charged particles that leave minimum ionizing deposits in the calorimeter. The muon spectrometer, the outermost part of the ATLAS detector, tracks trajectories of muons deflected in $0.5$ magnetic field provided by the superconducting toroidal magnets, giving an additional measure of muon's momentum \cite{ATLAS}. The MS tracks muon with $p_{T} > 3$ GeV in $|\eta| < 2.7$ range \cite{ATLAS}. As shown in Figure \ref{fig:ATLAS_MS}, the muon spectrometer comprises four types of detectors; first, the three stations of Monitored Drift Tubes (MDT) in $|\eta| < 2.0$ region followed by the Cathode Strip Chambers (CSC) in  $2.0 < |\eta| < 2.7$ region \cite{ATLAS}. The other two detectors are the Resistive Plate Chambers (RPC) in $|\eta| < 1.05$ and the Thin-gap Chambers (TGC) beyond $|\eta| = 1.05$ comprising the trigger system in MS \cite{ATLAS}. 

\begin{figure}
    \centering
    \includegraphics[width=.98\linewidth]{figures/LHC/ATLAS_MS.jpeg}
    \caption{ A schematic of different components of the muon spectrometer in ATLAS \cite{ATLAS}.\label{fig:ATLAS_MS}}
\end{figure}

\section{ Particle Reconstruction} 
\label{sec:ParticleReconstruction}
\subsection{Electrons}
\label{subsec:ParticleRecon_Elec}

\subsection{Muons}
\label{subsec:ParticleRecon_Muon}

\subsection{Jets}
\label{subsec:ParticleRecon_Jets}

\section{Future Upgrades}
\label{sec:FutureUpgrades}
\section{Future Upgrades}
\label{sec:FutureUpgrades}

\subsection{High-Luminosity LHC}
\label{subsec:HLLHC}
The planned High-Luminosity Large Hadron Collider (HL-LHC) is expected to operate starting in mid-2029. The primary goals of the HL-LHC are to collect large quantities of high-quality data needed to study rare SM processes such as Higgs self-interaction, Higgs couplings to lighter particles, the longitudinal component of vector boson scattering processes, and to extend the direct BSM searches beyond the current reach of LHC. The HL-LHC upgrade aims to increase the center-of-mass energy of proton-proton collisions to $\sqrt{s}=14$ TeV and the instantaneous luminosity up to $\mathcal L = 7.5 \times 10^{34} cm^{-2}s^{-1}$ \cite{HLLHC}. Figure \ref{fig:HLLHC} shows the complete operation of LHC starting in 2011 to the planned decade-long HL-LHC program. 

\begin{figure}[!htb]
    \centering
    \includegraphics[width=.95\linewidth]{figures/LHC/HLLHCPlan.jpeg}
    \caption{ Timeline of LHC operation starting from 2011 to the planned HL-LHC upgrade. Taken from \small{https://hilumilhc.web.cern.ch/content/hl-lhc-project}.\label{fig:HLLHC}}
\end{figure}
\normalsize

\subsection{ATLAS Upgrades}
\label{subsec:ATLASUpgrade}
The higher center-of-mass energy collisions and about $200$ interactions per bunch crossing at the HL-LHC gives rise to several detector challenges, such as higher detector occupancy, harsher radiation conditions, and higher particle fluxes \cite{HLLHC}. The ATLAS detector will upgrade several sub-systems to meet the challenges of the HL-LHC. The most significant upgrade is replacing the current ID with all-Silicon Inner Tracking (ITk) detector \cite{HLLHC}. Other upgrades include the muon system upgrades, such as the replacement of some MDT chambers in the inner barrel region \cite{HLLHCMuon}, the trigger and data acquisition system upgrade to meet challenges from higher detector occupancy \cite{HLLHCTrigger}, as well as upgrading the electronics of several other sub-systems \cite{HLLHC}. A new High Granularity Timing Detector (HGTD) will also be inserted in the end-cap regions to supplement the tracking system \cite{HLLHC}.

The ITk consists of Silicon pixel and strip detectors to increase granularity and radiation hardness with less material in the detector. Figure \ref{fig:ITKLayout} shows the ITk layout with $5$ inner layers of pixel detector and four outer layers of strips detector. The tracking for ITk is extended in the forward region up to $|\eta| < 4.0$ \cite{ITkStripsTDR}. 

\begin{figure}[!htb]
    \centering
    \includegraphics[angle=270,width=.7\linewidth]{figures/LHC/ITKLayout.pdf}
    \caption{ Schematic layout of ITK \cite{ITkPixelTDR}.\label{fig:ITKLayout}}
\end{figure}

At the HL-LHC, the ATLAS experiment is expected to record at least ten times more data than Run-2, making the precision measurements of the rare vector boson scattering process crucial. Identifying and reconstructing the two jets is one of the significant sources of experimental systematic uncertainties in the VBS measurements. Compared to Run-2, at the HL-LHC, the pile-up jet rejection efficiency for the forward jets is expected to increase dramatically due to extended $\eta$ coverage from the ITk \cite{HLLHC_JetTrack}. Moreover, the timing information from the HGTD in the HL-LHC is expected to further improve the forward pile-up jet rejection efficiency up to approximately $30\%$ \cite{HLLHC_HGTD}. Therefore, the HL-LHC program is critical to thoroughly probe the VBS measurements with extremely small cross-sections and two jets in the forward regions.

	\section{ATLAS Detector}
\label{sec:ATLAS}

A Toroidal LHC ApparatuS (ATLAS) is a general-purpose detector of LHC that detects events from proton-proton and heavy ion collisions \cite{ATLAS}. It is a $44$ meters long, and $25$ meters wide cylindrical-shaped detector built around LHC Interaction Point 1 \cite{ATLAS}. ATLAS has multiple concentric sub-detectors layered around the beamline, providing forward-backward symmetric coverage. The two proton beams collide at the center of the detector producing outgoing particles from hard scattering, underlying events, and pile-up. The outgoing particles interact with the detector material leaving tracks and energy deposits in several layers of the sub-detectors. 

The sub-detector closest to the beamline is called \textit{Inner Detector (ID)}, which measures the trajectories of the charged particle and plays a crucial role in identifying the physical position of hard-scattering, also known as the \textit{interaction point (IP)}. ID is surrounded by a solenoid magnet that provides a $2$ T magnetic field to bend the particle trajectories for momentum measurements \cite{ATLAS}. After the solenoid magnet lies the \textit{electromagnetic calorimeter (ECAL)} and then the \textit{hadronic calorimeter (HCAL)}, which measure the energy of electromagnetic and hadronic physics objects, respectively. The outermost layer of the ATLAS detector is the \textit{Muon Spectrometer(MS)} that provides a secondary measure of muon trajectories for its momentum measurement. MS is embedded inside a toroidal magnetic field that provides a magnetic field up to 3.5 T \cite{ATLAS}. Figure \ref{fig:ATLAS} shows a schematic of the ATLAS detector with all its sub-detectors.

\begin{figure}
    \centering
    \includegraphics[width=.98\linewidth]{figures/LHC/AtlasDetector.png}
    \caption{ A detailed schematic of the ATLAS detector with all its sub-detectors \cite{ATLAS}.\label{fig:ATLAS}}
\end{figure}

\subsection{ATLAS Coordinate System}
\label{subsec:ATLASCS}

ATLAS measurements use a right-handed coordinate system with the nominal interaction point as the origin. The beamline is along the cylindrical symmetry axis of the detector, which defines the longitudinal \textit{z}-axis. The transverse \textit{xy}-plane is perpendicular to the beam direction, where \textit{x}-axis points to the center of the LHC ring and \textit{y}-axis points upwards towards the surface. Figure \ref{fig:ATLAS_CS} shows a schematic of the ATLAS coordinate system. The angle measured around the beamline in \textit{xy}-plane gives the azimuthal angle $\phi$, whereas the angle measured with respect to the \textit{z}-axis gives the polar angle $\theta$. Transverse momentum ($p_{T}$) is particle's momentum in the \textit{xy}-plane, defined as, 

\begin{equation}
p_{T} = \sqrt{p_{x}^2+p_{y}^2}=p\sin\theta
\label{eqn:pT}
\end{equation}
\textit{Rapidity (y)} defined in terms of a particle's energy ($E$) and momentum ($p$) is a commonly used collider physics quantity that measures whether an outgoing particle is produced perpendicular or parallel to the \textit{z}-axis. Rapidity is defined as, 

\begin{equation}
    y = \frac{1}{2}\ln{ \left( \frac{E+p_{z}}{E-p_{z}} \right) }
    \label{eqn:Rapidity}
\end{equation}
Particles with larger momentum along the \textit{z}-axis have larger values of rapidity, whereas particles with larger momentum values in the transverse plane have smaller values of rapidity. For particles with negligible mass, the rapidity approaches a purely angular variable called \textit{pseudorapidity ($\eta$)} defined as, 

\begin{equation}
    \eta = \frac{1}{2}\ln{ \left( \frac{ |\vec{p}|+p_{z}}{ |\vec{p}| -p_{z}} \right) } = -ln { \left[ \tan \left( \frac{\theta}{2}\right) \right] } 
    \label{eqn:PseudoRapidity}
\end{equation}
Higher values of rapidity and pseudorapidity refer to the forward region of the detector. ATLAS detector has full $2\pi$ coverage in $\phi$ and maximum coverage up to $|\eta| < 4.5$ corresponding to $1.3^{\circ} < \theta < 178.7^{\circ} $ \cite{ATLAS}. 

\begin{figure}
    \centering
    \includegraphics[width=.98\linewidth]{figures/LHC/ATLAS_CoordinateSys.png}
    \caption{ A schematic of the right-handed ATLAS coordinate system \cite{ATLAS_CoordSys}.\label{fig:ATLAS_CS}}
\end{figure}

\subsection{Inner Detector}
\label{subsec:ID}
The inner detector is the innermost sub-detector of ATLAS and is responsible for tracking charged particles' trajectories and identifying the interaction point of the hard scatter. Closest to the interaction point is the Insertable B-Layer (IBL) \cite{ATLAS_IBL}, which was installed during the long-upgrade shutdown between Run-1 and Run-2 to meet the requirements for competent tracking at higher pile-up. The IBL is highly granular, consisting of roughly 12 million silicon pixel sensors with a size of $50\times 250 ~\mu m^2$ \cite{ATLAS_IBL}. IBL is located $3.3$ cm from the beamline and can reconstruct tracks within the pseudorapidity range of $|\eta|<2.5$ \cite{ATLAS_IBL}. 

Three layers of silicon-pixel detectors with $1,744$ pixel sensors, each comprising $47,232$ pixels of size $50\times 400 ~\mu m^2$ surround the IBL \cite{ID_Pixel}. The slightly larger pixel size is adequate for the pile-up at a distance larger than $5$ cm from the interaction point. These pixel layers were also present during the Run-1 data-taking period and provided coverage up to $|\eta|<2.5$ with a spatial resolution of tracks between $5$ and 12 $\mu$m \cite{ID_Pixel}. Surrounding the pixel layers is the Semiconductor Tracker (SCT) consisting of five layers of silicon microstrip detectors with a mean strip pitch of $80 ~\mu m$ in the barrel region and varying pitch of  $57-94 ~\mu m$ in the end-cap regions \cite{ID_Strips}. 

At a distance about $50$ cm from the beamline lies the outermost layer of the ATLAS inner detector, the Transition Radiation Tracker (TRT), with $370,000$ straw tubes with a diameter of $4$ mm \cite{ID_TRT}. Each TRT straw tube is filled with a Xenon-based gas mixture and consists of $31~\mu m$ diameter tungsten wires \cite{ID_TRT}. A charged particle passing through different layers of ID leaves a track via ionization. 

Figure \ref{fig:ATLAS_ID} schematically shows different parts of the inner detector and their distances from the interaction point.

\begin{figure}
    \centering
    \includegraphics[width=.98\linewidth]{figures/LHC/ATLAS_InnerDetector.jpg}
    \caption{ A schematic of the inner detector of ATLAS showing the IBL, pixel detectors, SCT, and TRT \cite{ID_Align_Run2}.\label{fig:ATLAS_ID}}
\end{figure}

\subsection{Calorimeters}
\label{subsec:Cal}

ATLAS has two calorimeters, electromagnetic and hadronic, designed to measure the energy of charged and neutral particles up to the range of $|\eta|\leq4.9$ \cite{ATLAS}. When interacting with a material, an electron loses its energy by photon emission, which could produce a pair of $e^{+}e^{-}$, which could again radiate a photon, creating an electromagnetic shower in the detector. Similarly, the hadronic particles also result in a shower of particles through multiple scattering. The calorimeters measure the energy of the particles by reconstructing the electromagnetic and hadronic showers. The calorimeters are designed to absorb all particles except muons and neutrinos. Therefore, motivated by the need to prevent \textit{punch-through}\footnote{particles' probabilities of passing through the calorimeters} effect, materials with high radiation length ($X_{0}$) and high interaction length ($\lambda$) are chosen to construct the calorimeters.

Outside the solenoid magnet surrounding the ID is the accordion-shaped electromagnetic calorimeter consisting of an alternate layer of lead absorber plates and highly granular liquid-argon (LAr) cells to precisely measure the energies of electrons and photons. It comprise of barrel section in $|\eta| < 1.475$ range and two end-caps in $1.375 < |\eta| < 3.2$ range \cite{ATLAS_ECAL}. The calorimeter's central region ($|\eta| < 2.5$) is designed to identify electrons and photons with high precision.

The hadronic calorimeter surrounds the ECAL and consists of a steel absorber and active scintillator tiles in the $|\eta| < 1.7$ range. In the end-caps range of $1.5 < |\eta| < 3.2$, it consists of a copper absorber and active LAr detectors. The forward region ranging from $3.2 < |\eta| < 4$ comprises the tungsten absorber followed by active LAr detectors \cite{ATLAS_HCAL}. 

Figure \ref{fig:ATLAS_Cals} schematically shows the layout of ATLAS calorimeters. 

\begin{figure}
    \centering
    \includegraphics[width=.98\linewidth]{figures/LHC/ATLAS_CALO.jpeg}
    \caption{ A schematic of electromagnetic and hadronic calorimeters In ATLAS \cite{ATLAS}.\label{fig:ATLAS_Cals}}
\end{figure}

\subsection{Muon Spectrometer}
\label{subsec:MS}
In ATLAS, muons are deeply-penetrating charged particles that leave minimum ionizing deposits in the calorimeter. The muon spectrometer, the outermost part of the ATLAS detector, tracks trajectories of muons deflected in $0.5$ magnetic field provided by the superconducting toroidal magnets, giving an additional measure of muon's momentum \cite{ATLAS}. The MS tracks muon with $p_{T} > 3$ GeV in $|\eta| < 2.7$ range \cite{ATLAS}. As shown in Figure \ref{fig:ATLAS_MS}, the muon spectrometer comprises four types of detectors; first, the three stations of Monitored Drift Tubes (MDT) in $|\eta| < 2.0$ region followed by the Cathode Strip Chambers (CSC) in  $2.0 < |\eta| < 2.7$ region \cite{ATLAS}. The other two detectors are the Resistive Plate Chambers (RPC) in $|\eta| < 1.05$ and the Thin-gap Chambers (TGC) beyond $|\eta| = 1.05$ comprising the trigger system in MS \cite{ATLAS}. 

\begin{figure}
    \centering
    \includegraphics[width=.98\linewidth]{figures/LHC/ATLAS_MS.jpeg}
    \caption{ A schematic of different components of the muon spectrometer in ATLAS \cite{ATLAS}.\label{fig:ATLAS_MS}}
\end{figure}
	\subsection{Electrons}
\label{subsec:ParticleRecon_Elec}

\subsection{Muons}
\label{subsec:ParticleRecon_Muon}

\subsection{Jets}
\label{subsec:ParticleRecon_Jets}
	\section{Future Upgrades}
\label{sec:FutureUpgrades}

\subsection{High-Luminosity LHC}
\label{subsec:HLLHC}
The planned High-Luminosity Large Hadron Collider (HL-LHC) is expected to operate starting in mid-2029. The primary goals of the HL-LHC are to collect large quantities of high-quality data needed to study rare SM processes such as Higgs self-interaction, Higgs couplings to lighter particles, the longitudinal component of vector boson scattering processes, and to extend the direct BSM searches beyond the current reach of LHC. The HL-LHC upgrade aims to increase the center-of-mass energy of proton-proton collisions to $\sqrt{s}=14$ TeV and the instantaneous luminosity up to $\mathcal L = 7.5 \times 10^{34} cm^{-2}s^{-1}$ \cite{HLLHC}. Figure \ref{fig:HLLHC} shows the complete operation of LHC starting in 2011 to the planned decade-long HL-LHC program. 

\begin{figure}[!htb]
    \centering
    \includegraphics[width=.95\linewidth]{figures/LHC/HLLHCPlan.jpeg}
    \caption{ Timeline of LHC operation starting from 2011 to the planned HL-LHC upgrade. Taken from \small{https://hilumilhc.web.cern.ch/content/hl-lhc-project}.\label{fig:HLLHC}}
\end{figure}
\normalsize

\subsection{ATLAS Upgrades}
\label{subsec:ATLASUpgrade}
The higher center-of-mass energy collisions and about $200$ interactions per bunch crossing at the HL-LHC gives rise to several detector challenges, such as higher detector occupancy, harsher radiation conditions, and higher particle fluxes \cite{HLLHC}. The ATLAS detector will upgrade several sub-systems to meet the challenges of the HL-LHC. The most significant upgrade is replacing the current ID with all-Silicon Inner Tracking (ITk) detector \cite{HLLHC}. Other upgrades include the muon system upgrades, such as the replacement of some MDT chambers in the inner barrel region \cite{HLLHCMuon}, the trigger and data acquisition system upgrade to meet challenges from higher detector occupancy \cite{HLLHCTrigger}, as well as upgrading the electronics of several other sub-systems \cite{HLLHC}. A new High Granularity Timing Detector (HGTD) will also be inserted in the end-cap regions to supplement the tracking system \cite{HLLHC}.

The ITk consists of Silicon pixel and strip detectors to increase granularity and radiation hardness with less material in the detector. Figure \ref{fig:ITKLayout} shows the ITk layout with $5$ inner layers of pixel detector and four outer layers of strips detector. The tracking for ITk is extended in the forward region up to $|\eta| < 4.0$ \cite{ITkStripsTDR}. 

\begin{figure}[!htb]
    \centering
    \includegraphics[angle=270,width=.7\linewidth]{figures/LHC/ITKLayout.pdf}
    \caption{ Schematic layout of ITK \cite{ITkPixelTDR}.\label{fig:ITKLayout}}
\end{figure}

At the HL-LHC, the ATLAS experiment is expected to record at least ten times more data than Run-2, making the precision measurements of the rare vector boson scattering process crucial. Identifying and reconstructing the two jets is one of the significant sources of experimental systematic uncertainties in the VBS measurements. Compared to Run-2, at the HL-LHC, the pile-up jet rejection efficiency for the forward jets is expected to increase dramatically due to extended $\eta$ coverage from the ITk \cite{HLLHC_JetTrack}. Moreover, the timing information from the HGTD in the HL-LHC is expected to further improve the forward pile-up jet rejection efficiency up to approximately $30\%$ \cite{HLLHC_HGTD}. Therefore, the HL-LHC program is critical to thoroughly probe the VBS measurements with extremely small cross-sections and two jets in the forward regions.
\clearpage

\part {\LARGE{Analysis Overview}}
\label{sec:AnalysisOverview}

\section{Goals}
\label{sec:Analysis_Goals}

The primary goal of the analysis is to measure the differential cross-sections of the kinematic observables sensitive to the EWK $ZZjj \rightarrow 4\ell jj$ production mode. The differential cross-sections measured in VBS-enhanced phase space are used in the precision study of the SM $4\ell jj$ production and constrain the effects of BSM physics. For simpler re-interpretation in the future without ATLAS detector simulations, the differential cross-sections are measured at a particle level using an unfolding technique, which corrects the detector effects. The details of the unfolding to extrapolate the particle-level yield from detector-level yield will be discussed in Section \ref{sec:Unfolding}. The unfolded cross-sections shown in Section \ref{sec:DifferentialxS} are then used to constrain the effect of BSM in a model-independent framework using the Effective Field Theory (EFT) approach, which will be discussed in Section \ref{sec:EFT}. 
  

	\section{Fiducial Phase Space}
\label{sec:FidSel}

The ATLAS detector has limited acceptance in the physical phase space, and the selected objects and events are reconstructed within this acceptance phase space. Thus, a fiducial phase space imitating the detector acceptance is defined using physics objects at the particle level to measure the unfolded differential cross-sections. The particle-level signal events are essential to derive the inputs needed to correct the detector effects. Moreover, the measured unfolded cross-sections are compared to the SM-predicted particle-level cross-sections evaluated from these events. Thus, it is essential to carefully select the particle-level signal events in the defined fiducial phase space. This section summarizes the kinematic requirements defining the fiducial phase space of the analysis and the selection of the particle-level signal events. 

The kinematic selections for the fiducial phase space are close to the detector-level object and event selections. The kinematic requirements applied are motivated by the nature of the electroweak production of $pp\rightarrow ZZ^* ( \rightarrow 4\ell) jj$ $[\ell = e,~\mu]$, where the two SF-OC lepton pairs from two $Z$ bosons are produced centrally with respect to the highly energetic dijet. Moreover, the fiducial phase space contains no leptons from tau decays. Both fiducial-level electrons and muons are required to be dressed. The dressing procedure accounts for the energy losses of leptons through photon emissions via bremsstrahlung. Dressed leptons are constructed by adding the four-momenta of nearby photons within the lepton's small $\Delta R < 0.1$ cone. Several kinematic cuts summarized in Table \ref{tab:FidObjectCut} are applied individually to the muons, electrons, and jets to ensure the selected particle-level objects fall within the detector's acceptance before defining the events. Motivated by the discussion of physics object reconstruction in Section \ref{sec:ParticleReconstruction}, each electron is required to have $p_{T} > 7$ GeV and $|\eta| < 2.47$, whereas the muons satisfy $p_{T} > 5$ GeV and $|\eta| < 2.7$. 

\begin{table}[!htbp]
    \centering
    \caption{Details of the kinematic pre-selection applied to the particle-level electrons, muons, and jets. The required kinematic cuts are applied to the dressed leptons.
    \label{tab:FidObjectCut}}
    \begin{tabular}{|| l || c | c | c ||}
        \hline
        Selections      & Electrons             &       Muons        &          Jets            \\
        \hline\hline
        $\Pt~$          & $> 7$ GeV             &       $ >5$ GeV    &      $>30$GeV        \\
        \hline 
        $|\eta|$            &  $< 2.47  $           &       $ < 2.7 $        &      $ < 4.5$            \\
        \hline
    \end{tabular}
\end{table}  

Event quadruplets are formed from the events with at least four leptons by requiring two SF-OC lepton pairs, with leading and sub-leading lepton $p_{T}>20$ GeV and angular separation between any two leptons to satisfy $\Delta R > 0.05$. The invariant mass of any SF-OC lepton pair is required to satisfy $m_{\ell \ell } > 5$ GeV. These particle-level selections have similar motivation to those defined for the detector-level requirements in Section \ref{sec:EventSel}. In any event with more than two SF-OC lepton pairs, the quadruplet is formed by choosing the two pairs that minimize the distance to the $Z$ resonance pole. Once the quadruplet is formed, the leading-lepton pair is defined as the one with a higher absolute rapidity value, i.e., $|y_{ij}|$. Finally, an additional criterion on the invariant mass of the quadruplet of $m_{4\ell} > 130$ GeV is imposed. 

Similarly, the event dijet is constructed from the two leading jets with the opposite sign of pseudo-rapidity ($\eta$) to imitate the detector-level VBS dijet production, where jets are reconstructed on the opposite side of the detector. Similar to detector-level, the particle-level jets are required to satisfy $|n| < 4.5$, $p_{T,~leading~jet} > 40$ GeV, and $p_{T,~sub-leading~jet} > 30$ GeV. The dijet is required to have a significant rapidity separation of $|\Delta y_{jj}| > 2$ and high invariant mass of $m_{jj} > 300$ GeV to resemble dijet produced in electroweak $ZZ^*(\rightarrow 4 \ell) jj$ production. Table \ref{tab:QuadDijetFidCut} summarizes the requirements to select the particle-level quadruplet and dijet in an event.           
    
\begin{table}[!htbp]
    \caption{Details of the kinematic selections applied to form a particle-level quadruplet and a particle-level dijet in the fiducial volume. 
    \label{tab:QuadDijetFidCut}}
    \begin{tabular}{|| l || c ||}
        \hline
        Selections              &           Cut \\
        \hline\hline
        Lepton Kinematics       & $P_{T,~leading~lepton} > 20 $ GeV\\
                                & $P_{T,~sub-leading~lepton} > 20 $ GeV\\
        \hline 
        Pair Requirement        & $\Delta R_{\ell i,\ell     j} > 0.05 $\\
                                & SF-OC with $\mll > 5$ GeV\\
        \hline
        Quadruplet Requirement  & $2$ pair candidates with smallest $|\mOneTwo  - m_{Z} | + |\mThreeFour    - m_{Z} |$  \\
                                & Leading pair: pair with highest $|y_{ij}|$\\
                                & Sub-leading pair: pair with lowest $|y_{ij}|$\\
                                & $\mFourL > 130 $ GeV\\
        \hline
        Di-jet Requirement      & $p_{T,~leading~jet} > 40$ GeV \\
                                & $|\Delta y_{jj}| > 2 $ \\ 
                                & $m_{jj} > 300$ GeV    \\
        \hline
    \end{tabular}
\end{table}

	\section{Reconstruction Selection}
\label{sec:ObjReconstruction}

This section summarizes the detector-level phase space selections applied to three physics objects, electrons, muons, and jets used in the measurement. Each physics object of the analysis has two categories: \textit{baseline} and \textit{signal} objects. Physics objects satisfying a set of kinematic selections or looser identification criteria are categorized as \textit{baseline} whereas, the baseline leptons that pass either stricter kinematic selections or additional isolation and track-to-vertex association (TTVA) requirements are \textit{signal}.

\subsection{Electrons}
\label{subsec:ElecRecon}

As discussed in Section \ref{subsec:ParticleRecon_Elec}, electrons are reconstructed by matching the inner detector track (ID) to an energy cluster in the electromagnetic calorimeter. Baseline electron objects are required to satisfy the kinematic selections of $\Pt~ > 7$ GeV $ \&~ |\eta| < 2.47$ and a loose likelihood identification of working point \textit{LHVeryLoose}. To avoid the electrons from pileup, a loose vertex association requirement of $|z_{0}\sin\theta| < 0.5 $ mm and an overlap removal discussed in section \ref{subsec:OR} is applied to the baseline electron candidates.

Signal electrons are required to pass a more stringent loose likelihood identification, \textit{LHLooseBL}, which requires at least one hit in the innermost layer of the pixel detector. The signal electrons are distinguished by requiring the baseline electrons to have impact parameter requirements of $d0/ \dZeroSig < 5$ and an isolation working point identification of \textit{LooseVarRad}. Table \ref{tab:Electron_RecoSel} summarizes the several selections imposed to define the baseline and signal electrons.

\begin{table}[ht]
	\centering
		\caption{Definition of the baseline and signal electrons.\label{tab:Electron_RecoSel}}
		\begin{tabular}{|| l || c | c ||}
		\hline
		Selection Category & \textbf{Baseline} & \textbf{Signal} \\
		\hline\hline
		Kinematic cuts & $p_{T}~ > 7$ GeV & $ p_{T}~ > 7$ GeV \\
		            & $|\eta| < 2.47$  &  $|\eta| < 2.47$\\
		\hline  
		Identification & LHVeryLoose & LHLooseBL \\
		\hline 
		Vertex Association & $|z_{0}\sin\theta| < 0.5$ mm & $|z_{0}\sin\theta|< 0.5$ mm\\
		\hline
		Overlap removal & Lepton-favored & Lepton-favored\\
		\hline
		Isolation Working Point & $-$ & LooseVarRad\\
		\hline 
		Impact Parameters & $-$ & $d_{0}/ \sigma_{d_{0}} < 5$ \\
		\hline
	\end{tabular}
\end{table}

\subsection{Muons}
\label{subsec:MuonRecon}
As discussed in section \ref{subsec:ParticleRecon_Muon}, muons are reconstructed in multiple ways based on information from the inner detector (ID), the muon spectrometer (MS), and the calorimeters. All baseline muons are required to satisfy $ |\eta| < 2.7 $, $p_{T} > 5$ GeV, a loose impact parameter requirements of $|z_{0}\sin\theta| < 0.5 $ mm, lepton-favoring overlap removal and \textit{Loose} identification working point. The signal muons are identified by requiring additional isolation identification of \textit{PflowLooseVarRad} and TTVA requirements of $d_{0}/\sigma_{d_{0}} < 3$. Table \ref{tab:muon_baseline_signal} summarizes baseline and signal muons selection requirements.

\begin{table}[ht]
	\centering
		\caption{Definition of the baseline and signal muons.\label{tab:muon_baseline_signal}}
		\begin{tabular}{|| l || c | c ||}
		\hline
		Selection Category & \textbf{Baseline} & \textbf{Signal} \\
		\hline\hline
		Kinematic cuts & $p_{T}~ > 5$ GeV & $p_{T}~ > 5$ GeV \\
					& Calo-tagged $ p_{T}~ > 15$ GeV & Calo-tagged $ p_{T}~ > 15$ GeV \\
		      & $|\eta| < 2.7$ & $|\eta| < 2.7$\\
		\hline
		Identification & Loose & Loose \\
		\hline 
		Vertex Association & $|z_{0}\sin\theta| < 0.5$ mm & $|z_{0}\sin\theta|< 0.5$ mm\\
		\hline
		Overlap removal & Lepton-favored & Lepton-favored\\
		\hline
		Isolation Working Point & $-$ & PflowLooseVarRad\\
		\hline 
		Impact Parameters & $-$ & $d_{0}/\sigma_{d_{0}} < 3$ \\
		\hline
	\end{tabular}
\end{table}

\subsection{Jets}
\label{subsec:JetRecon}
Jets are reconstructed with the particle flow anti$-K_{T}$ clustering algorithm using a radius parameter of $R = 0.4$ as discussed in section \ref{subsec:ParticleRecon_Jets}. The jets reconstructed using the particle flow algorithm are required to satisfy $p_{T}~ > 15$ GeV, $ |\eta| < 4.5 $ kinematic cuts, and the lepton-favored overlap removal to be classified as baseline jets. Baseline jets satisfying the \textit{Tight} working point of the jet to the vertex tagger tool are classified as signal jets. \textit{Jet-vertex-tagger (JVT)} is applied to the baseline jets with $ |\eta| < 2.4 $ whereas the \textit{forward-jet-vertex-tagger (fJVT)} tool is applied to the baseline jets with $ |\eta| > 2.5 $. Table \ref{tab:jets} summarizes the details of baseline and signal jets selection. 

\begin{table}[ht]
	\centering
	\caption{Definition of the baseline and signal jets.\label{tab:jets}}
		\begin{tabular}{|| l || c | c ||}
		\hline
		Selection Category & \textbf{Baseline} & \textbf{Signal} \\
		\hline\hline
		Kinematic cuts & $\Pt~ > 30$ GeV & $\Pt~ > 30$ GeV \\
			 & $|\eta| < 4.5$ & $|\eta| < 4.5$\\
		\hline 
		Identification & AntiKt4EMPFlow & AntiKt4EMPFlow\\
		\hline
		Overlap removal & Lepton-favored & Lepton-favored\\
		\hline
		Jet-Vertex-Tagger & $-$ & $ |\eta| < 2.4 $ JVT ("Tight")\\
				& $-$ & $|\eta| > 2.5 $ fJVT ("Tight")\\
		\hline
	\end{tabular}
\end{table}

\subsection{Overlap Removal}
\label{subsec:OR}

An \textit{overlap removal} procedure is applied to remove the physics objects reconstructed from the same detector signal. The measurement uses a lepton-favored overlap removal which selects leptons over jets. Overlap removal is an iterative process in which only objects surviving all previous steps are used in the subsequent steps. Table \ref{tab:overlap_removal} summarizes the overlap removal steps, where the $\Delta R$ is the angular separation between objects calculated using rapidity.

\begin{table}[ht]
	\centering
		\caption{Overlap removal used in the analysis. An object removed in one step does not enter into the subsequent step. \label{tab:overlap_removal}}
		\begin{tabular}{|| l || c | c ||}
		\hline
		Remove Object & Accept Object & Overlap Criteria \\
		\hline\hline
		Electron & Electron & Share a track or have overlapping calorimeter cluster.\\
				&		& Keep electron with higher $p_{T}~$\\
		\hline
		Muon & Electron & Share ID track, and the muon is calo-tagged\\
		\hline
		Electron & Muon & Share ID track\\
		\hline
		Jet & Electron & $\Delta R_{e-jet} < 0.2$ \\
		\hline 
		Jet & Muon & $\Delta R_{\mu-jet} < 0.2/$ghost-associated and $N_{jet~tracks} < 3$\\
		\hline
	\end{tabular}
\end{table}
	\section{Trigger}
\label{sec:Trigger}

Due to the presence of four fully reconstructed leptons in the final state, the data events and detector-level MC events are preselected using a logical OR of different single and double-lepton triggers. The trigger menu varies according to the data-taking run periods to reflect the changes in the high-level trigger system, which are required to cope with increasing data rates. Additionally, trigger matching is required for the selected events. The trigger matching selects a subset of preselected events in which at least one lepton of the quadruplet is matched to one of the fired triggers. Table \ref{tab:Trigger} shows the trigger menu used by the analysis per different data periods using either electrons, muons, or mixed electron-muon triggers. The \textit{HLT\_*} string specifies the high-level trigger menu used where the "e" or "mu" substring specifies the type of object used to fire the trigger, and the attached number specifies the minimum $p_{T}$ threshold for the object. The substring \textit{lh*} attached to \textit{HLT\_*} specifies the identification working point for the electrons, and the \textit{ivar*} specifies the isolation working point used for either object. The string \textit{L1*} indicates the use of either calorimeters or MS L1 trigger, and the string \textit{noL1} suggests the absence of L1 triggers. 

\begin{table}
    \centering
    \begin{tabular}{| l | c | c |}
    \hline 
    Period & Leptons & Triggers \\
    \hline
    \multirow{10}{*} {2015} & \multirow{4}{*} {Electron} &  HLT$\_$e24$\_$lhmedium$\_$L1EM20VH  \\
                        &          &  HLT$\_$e60$\_$lhmedium \\
                        &    & HLT$\_$e120$\_$lhloose \\
                        & &  HLT$\_$2e12$\_$lhvloose$\_$L12EM10VH\\\cline{2-3}

                        & \multirow{4}{*} {Muon} & HLT$\_$mu20$\_$iloose$\_$L1MU15 \\
                        & & HLT$\_$mu50 \\
                        & & HLT$\_$2mu10 \\
                        & & HLT$\_$mu18$\_$mu8noL1 \\\cline{2-3}
                        & \multirow{2}{*}{Mixed} & HLT$\_$e7$\_$lhmedium$\_$mu24 \\
                        & & HLT$\_$e17$\_$lhloose$\_$mu14 \\
    \hline
    \multirow{10}{*} {2016} & \multirow{4}{*} {Electron} & HLT$\_$e26$\_$lhtight$\_$nod0$\_$ivarloose \\
                        &          & HLT$\_$e60$\_$lhmedium$\_$nod0 \\
                        & & HLT$\_$e140$\_$lhloose$\_$nod0\\
                        & &  HLT$\_$2e17$\_$lhvloose$\_$nod0 \\\cline{2-3}
                        & \multirow{4}{*} {Muon} & HLT$\_$mu26$\_$ivarmedium \\
                        & & HLT$\_$mu50 \\
                        & & HLT$\_$2mu14 \\
                        & & HLT$\_$mu22$\_$mu8noL1 \\\cline{2-3}
                        & \multirow{2}{*}{Mixed} & HLT$\_$e7$\_$lhmedium$\_$nod0$\_$mu24  \\
                        & & HLT$\_$e17$\_$lhloose$\_$nod0$\_$mu14 \\
    \hline
    \multirow{10}{*} {2017} & \multirow{4}{*} {Electron} &  HLT$\_$e26$\_$lhtight$\_$nod0$\_$ivarloose \\
                        & &  HLT$\_$e60$\_$lhmedium$\_$nod0 \\
                        & & HLT$\_$e140$\_$lhloose$\_$nod0 \\
                        &          & HLT$\_$2e24$\_$lhvloose$\_$nod0 \\\cline{2-3}
                        & \multirow{4}{*} {Muon} & HLT$\_$mu26$\_$ivarmedium \\
                        & & HLT$\_$mu50 \\
                        & & HLT$\_$2mu14 \\
                        & & HLT$\_$mu22$\_$mu8noL1  \\\cline{2-3}
                        & \multirow{2}{*}{Mixed} &  HLT$\_$e17$\_$lhloose$\_$nod0$\_$mu14  \\
                        & & HLT$\_$e26$\_$lhmedium$\_$nod0$\_$mu8noL1 \\
    \hline
    \multirow{10}{*} {2018} & \multirow{4}{*} {Electron} & HLT$\_$e26$\_$lhtight$\_$nod0$\_$ivarloose \\
                        &          & HLT$\_$e60$\_$lhmedium$\_$nod0  \\
                        &          & HLT$\_$e140$\_$lhloose$\_$nod0 \\
                        &          &  HLT$\_$2e24$\_$lhvloose$\_$nod0 \\\cline{2-3}
                        & \multirow{4}{*} {Muon} & HLT$\_$mu26$\_$ivarmedium \\
                        & & HLT$\_$mu50  \\
                        & & HLT$\_$2mu14 \\
                        & &  HLT$\_$mu22$\_$mu8noL1 \\\cline{2-3}
                        & \multirow{2}{*}{Mixed} & HLT$\_$e17$\_$lhloose$\_$nod0$\_$mu14 \\
                        & &  HLT$\_$e26$\_$lhmedium$\_$nod0$\_$mu8noL1  \\
    \hline
    \end{tabular}
    \caption{Trigger menu used in the analysis for event preselection \label{tab:Trigger}}
\end{table}

The trigger efficiency of MC is defined as a ratio of events passing the logical OR selection of the triggers to the number of events passing reconstruction level pre-trigger selection. The trigger efficiency for MC events could differ from that for the data. Thus, trigger efficiency scale factors are applied to MC events to account for the differences. The scale factors are defined as a fraction of trigger efficiency for MC to that of data and retrieved from the ATLAS supported tool \textit{TrigGlobalEffciencyCorrectionTool}\footnote{https://gitlab.cern.ch/atlas/athena/tree/21.2/Trigger/TrigAnalysis/TrigGlobalEfficiencyCorrection}. Figure \ref{fig:Trigger} shows the efficiency of trigger selection in events with a signal quadruplet and a dijet as a function of $m_{4\ell}$. Both trigger and trigger matching efficiencies are $100\%$ over the entire spectrum. 
\begin{figure}
    \centering
    \includegraphics[width=.8\linewidth]{figures/AnalysisOverview/TriggerEfficiency.pdf}
    \caption{ Trigger efficiency as a function of $m_{4\ell}$ in events with a quadruplet and a dijet.\label{fig:Trigger}}
\end{figure}
	\section{Event Selection}
\label{sec:EventSel}

A $ZZ^*(\rightarrow 4\ell) jj$ event at the detector-level consists of a lepton quadruplet formed by two SF-OC lepton pairs from each $Z$ boson decay and a dijet from the initial state partons. The quadruplet is formed in events with four prompt leptons (electrons or muons), where the leading and sub-leading leptons satisfy $p_{T} > 20$ GeV to ensure a high trigger efficiency. Due to momentum conservation, the prompt leptons are separated from each other. Therefore, all prompt leptons in an event must have $\Delta R > 0.05$ to reduce contributions from mis-reconstruction while keeping leptons from possible boosted production scenarios. Additionally, each SF-OC lepton pair's invariant mass is required to satisfy $m_{\ell \ell } > 5$ GeV to suppress the contamination from lower resonance backgrounds. A quadruplet is formed from the two SF-OC lepton pairs whose invariant masses are closest and next closest to the mass of the Z-boson ($m_{Z}$). Based on these requirements, the quadruplets can be of the following three types:

\begin{itemize}
\item{$4e$: events with two $e^{+}e^{-}$ pairs.}
\item{$4\mu$: events with two $\mu^{+}\mu^{-}$ pairs.}
\item{$2e2\mu$ or $2\mu2e$: events where one of the pair is $e^{+}e^{-}$ and other is $\mu^{+}\mu^{-}$}
\end{itemize}

Once the quadruplet is formed, the leading-lepton pair is defined as the one with a higher absolute rapidity value, i.e., $|y_{ij}|$. This additional identification is motivated by consistently defining the leading and sub-leading pairs at both particle and detector levels to reduce the resolution-induced bin migrations, which need to be corrected by the unfolding procedure. The quadruplets with all four leptons passing the signal lepton criteria of the TTVA and isolation are the \textit{signal quadruplet} defining the signal region. The quadruplets with one or more lepton failing either the isolation or TTVA requirement are the \textit{non-signal quadruplets} and are used in the non-prompt background estimation. The invariant mass of the quadruplet ($m_{4\ell}$) must be greater than 130 GeV to exclude the events from on-shell $H\rightarrow ZZ^*$ production, which are measured extensively by ATLAS analyses focused on Higgs measurements.

The final state jets in electroweak production of $ZZ^*(\rightarrow 4\ell) jj$ come from the initial state quarks on the opposite sides of the interaction point. Thus, the dijet is selected by requiring two signal jets defined in Section \ref{subsec:JetRecon} from the opposite side of the detector, i.e., $\eta_{leading~jet} \times \eta_{sub-leading~jet} < 0$. To maximize the probability of selecting an event from EWK $ZZ^*(\rightarrow 4\ell)jj$ production, a requirement of significant rapidity difference between the jets of $\Delta y_{jj}> 2 $ and a large invariant mass of $m_{jj} > 300 $ GeV are imposed on the dijet selection. 

\subsection{Signal Region}
\label{subsec:SignalRegion}
The two $Z$ bosons in the electroweak production of $ZZ^*jj$ are produced centrally with respect to the dijet. Thus, the signal region of the analysis is defined based on the centrality ($\zeta$) of the di-$Z$boson production in an event. Centrality depends on the rapidity of the quadruplet and the rapidity of the dijet as:
\begin{equation}
    \zeta~=~\frac{|y_{quadruplet}~-~ 0.5*(y_{leading~jet}~+~y_{sub-leading~jet})| }{|y_{leading~jet}~-~y_{sub-leading~jet}|}.
    \label{eq:centr}
\end{equation}

Figure \ref{fig:centrality_a} shows the predicted MC distribution of centrality for the three main production modes of $ZZ^*(\rightarrow 4 \ell)jj$. The significance of the EWK component over the inclusive parton initiated and $gg$-loop initiated QCD production is defined as, 
\begin{equation}
    s=\frac{N_{EWK}}{\sqrt{N_{QCD}^{(qq)}+N_{QCD}^{(gg)}}},
    \label{eqn:EWKSignificance}
\end{equation}
where $N_{EWK}$, $N_{QCD}^{(qq)}$ and $N_{QCD}^{(gg)}$ are the numbers of events from electroweak, parton-initiated QCD and gluon-loop initiated QCD productions, respectively. The chosen cut value on the centrality maximizes the EWK significance while maintaining a good selection efficiency of EWK events. Figure in \ref{fig:centrality_b} shows the efficiency and significance for various cut values of centrality.  

A VBS-Enhanced signal region is defined based on events with a quadruplet, a dijet, and $\zeta<0.4$. The low value of the centrality and the requirements for a signal dijet ensures that the events in this signal region originate in a more significant fraction from the electroweak production of $ZZ^*(\rightarrow 4 \ell) jj$. A VBS-Suppressed control region is also defined based on events with a quadruplet, a dijet, and $\zeta>0.4$. These events mainly originate from the QCD production of $ZZ^*(\rightarrow 4 \ell) jj$ and are used to optimize the analysis strategies. Table \ref{tab:EventSelection} summarizes all selections applied to select $ZZ^*(\rightarrow 4\ell)jj$ detector-level events.

\begin{figure}[!htbp]
\begin{subfigure}{.48\textwidth}
  \centering
  \includegraphics[width=.95\linewidth]{figures/AnalysisOverview/centrality_Dist.pdf}  
  \caption{Yields of EWK and QCD production.}
  \label{fig:centrality_a}
\end{subfigure}
\begin{subfigure}{.48\textwidth}
  \centering
  \includegraphics[width=.9\linewidth]{figures/AnalysisOverview/centrality_Cut.pdf}  \\
  \caption{Selection efficiency and EWK significance. }
  \label{fig:centrality_b}
\end{subfigure}
\caption{Centrality dependence for yield, EWK selection efficiency, and EWK significance. }
\end{figure}

\begin{table}[!htbp]
    \centering
        \caption{Details of event selection.\label{tab:EventSelection}}
        \begin{tabular}{|| l || c | c ||}
        \hline
        Event Selection         & Cut                   & Requirement                                                       \\
        \hline\hline
        Event               & Trigger                   &  Fire at least one lepton trigger                                     \\
        Preselection                & Vertex                    & At least one vertex with $2$ or more tracks                               \\
        \hline  
                    & Lepton Kinematics         & $p_{T} > 20$ GeV for two leading leptons                     \\
                    & Lepton Separation         & $\Delta R_{ij} > 0.05$ between leptons in quadruplet      \\
        Quadruplet  & Pair Requirement          & Two SF-OC lepton pairs                                            \\
        Selection   &                       & $m_{\ell\ell} > 5$ GeV                                    \\
                    & Minimal $\Delta m_{Z}$    & quadruplet with smallest $|m_{12} - m_{Z} | + |m_{34} - m_{Z} |$\\
                    &                       & Leading Pair: pair with highest $|y_{ij}|$                        \\
                    & ZZ Mass               & $m_{4\ell} > 130 $ GeV                                            \\
        \hline  
        Quadruplet          & Signal Quadruplet         & Quadruplet with all \textbf{signal leptons}                           \\
        Categorisation          & non-signal Quadruplet         & Quadruplet with $\geq 1$ \textbf{baseline-non-signal lepton}          \\
        \hline  
                    & Different Detector Sides      & $\eta_{leading~jet} \times \eta_{sub-leading~jet} < 0 $          \\
        Dijet       & Rapidity Separation       & $ \Delta y_{jj}> 2 $                                              \\
        Selection   & Leading Jet $p_{T}$   &   $p_{T,~leading~jet} > 40$ GeV               \\
                    & Dijet Mass                & $m_{jj} > 300 $ GeV                                                   \\
                    & Dijet         & Both jets required to pass either JVT or FJVT                             \\
        \hline  
                            
        Event               & VBS-Enhanced Region       & signal quadruplet $\&$ dijet and centrality ($\zeta) < 0.4 $              \\
        Categorisation          & VBS-Suppressed Region     & signal quadruplet $\&$ dijet and centrality ($\zeta) > 0.4$               \\
        
        \hline
    \end{tabular}
\end{table}

Figure \ref{fig:EventDisplayZZjj} illustrates a typical signal event with two $Z$-bosons produced in association with two jets. The event display corresponds to an event recorded during Run Number $340368$ of the 2017 data-taking period. The two light-yellow cones on two opposite sides of the detector with a large rapidity gap represent the reconstructed dijet of the event with an invariant mass of $m_{jj} = 2228$ GeV. In this event, one of the SF-OC lepton pairs is formed from $e^+e^-$ representing the $Z\rightarrow e^+e^-$ decay, and the other is formed from $\mu^+\mu^-$ corresponding to the $Z\rightarrow \mu^+\mu^-$ decay, which are represented respectively by the pairs of green and red tracks. The invariant mass of the four leptons is $m_{4\ell} = 605$ GeV. The significant rapidity separation between the two jet cones on the opposite sides of the ATLAS detector and centrally produced two $Z$ bosons defines the characteristic feature of the EWK production of $ZZ(\rightarrow 4\ell) jj$. 

\begin{figure}[!htbp]
\centering
\includegraphics[width=.9\linewidth]{figures/AnalysisOverview/ZZjjEventDisplay.png}  
\caption{Event display of a candidate $pp \rightarrow ZZjj \rightarrow e^+e^-\mu^+\mu^- jj $ recorded by the ATLAS experiment in Run-$2$ $2017$ data-taking period. \label{fig:EventDisplayZZjj} \cite{ATLASZZjj}.}
\end{figure}
	\section{Datasets and Monte Carlo Simulation}
\label{sec:DataSetAndMonteCarlo}

\subsection{LHC Dataset}
\label{subsec:Dataset}

The measurement uses the LHC collision data, the ATLAS Run-$2$ dataset collected by the ATLAS experiment during its operation in $2015$, $2016$, $2017$, and $2018$. This dataset corresponds to proton-proton collisions at the center-of-mass energy of $\sqrt{s} = 13$ TeV and total integrated luminosity of $139 \pm 2.4$ fb$^{-1}$ measured by the LUCID-2 detector \cite{ATLASLuminosityDetector}\cite{ATLASRun2IntegratedLumi}. The LUCID-2 detector is a Cherenkov detector with 16 photo-multipliers (PMT) that provides counts of particles from each colliding bunch. The counts integrated over time periods of about $60$ seconds are called luminosity blocks (LB). The instantaneous luminosity is given by 
\begin{equation}
  \mathcal{L} = \frac{R}{\sigma_{vis}},
\end{equation}
where $R$ is the event rate and $\sigma_{vis}$ is the calibration constant measured during the LHC's special runs, which took place at the beginning of each year of data taking. The calibration constant accounts for the non-linear response of the LUCID-2 detector. The uncertainty on the integrated luminosity is obtained from the combination of the measurements of each year of the LHC run.

Each data-taking run period of the LHC is further divided into sub-periods of one to three weeks that vary in beam and detector conditions. The dataset used in physics analyses is required to satisfy a series of data quality checks discussed in detail in Ref \cite{ATLASRun2DataTaking}. The data passing these requirements collectively form a Good Run List (GRL) and consists of several LB. Figure \ref{fig:InstLuminosity} shows the total integrated luminosity delivered by LHC in the green distribution, recorded by the ATLAS experiment in the yellow distribution and part of the GRL in the blue distribution. The plateaus correspond to the end-of-year shutdowns of LHC, and the slopes correspond to the increasing instantaneous luminosity in different data-taking periods. 

\begin{figure}[!htb]
\centering
\includegraphics[width=.8\linewidth]{figures/AnalysisOverview/IntegratedLumiRun2.pdf}  
  \caption{Total integrated luminosity collected during data taking period in Run-$2$ \cite{ATLASRun2DataTaking}. }
\label{fig:InstLuminosity}
\end{figure}

\subsection{Monte Carlo Samples }
\label{subsec:MCSamples}

As briefly mentioned in Section \ref{sec:Pheno}, the $pp \rightarrow ZZ^* (\rightarrow 4\ell) jj$ events are simulated by MC generators. The first step in generating SM prediction is \textit{event generation} which incorporates the matrix element calculations for the hard-scatter $pp \rightarrow ZZ^* (\rightarrow 4\ell) jj$ process, the parton showering, hadronization, and the effect of the underlying events. The generated events are then \textit{simulated} to interact with the ATLAS material using the Geant4 simulation toolkit following the description in Ref \cite{GEANT4}. The energy deposits of the simulated events in the detectors are then \textit{digitized} where the simulated signal hits are overlayed with cavern background events hits and additional hits from soft QCD events to simulate the effect of the pile-up. Finally, the events are \textit{reconstructed} using the same procedure to reconstruct data objects using detector geometry corresponding to the data-taking period. The final simulations are available in terms of \textit{physics derivations}, which are used in the final analysis. Figure \ref{fig:MCGenerationSchematic} shows a schematic overview of the MC generation. 
\begin{figure}[!htb]
\centering
\includegraphics[width=.3\linewidth]{figures/AnalysisOverview/MCSchematic.png}  
  \caption{Various steps in MC sample generation.}
\label{fig:MCGenerationSchematic}
\end{figure}

Each physics process is simulated using different generation campaigns corresponding to the different conditions of Run-2 ATLAS data-taking periods. As shown in Figure \ref{fig:PileupDiffRuns}, the pile-up distribution is different for the different data-taking periods. The MC-generated events are modified to correctly simulate the effect of pile-up distribution to simulate that of the data. 

\begin{figure}[!htb]
  \centering
  \includegraphics[width=.8\linewidth]{figures/AnalysisOverview/mu_ProfileRun2.pdf}
  \caption{Pile-up distributions in different Run-2 data-taking period.\label{fig:PileupDiffRuns} \cite{ATLASRun2DataTaking}}
\end{figure}

\subsubsection{Signal Samples}
\label{subsubsec:SigSamples}
As discussed in Section \ref{sec:EWKPheno}, two types of interactions, QCD and EWK, give us $pp \rightarrow ZZ^*(\rightarrow 4 \ell) jj$ final state. The two types of QCD process, quark induced $qqZZ$ $[qq \rightarrow ZZ^*(\rightarrow 4 \ell) jj]$ and gluon induced $ggZZ$ $[gg \rightarrow ZZ^* (\rightarrow 4\ell) ]jj$ are simulated using the \textsc{Sherpa} $2.2.2$ MC generator. The parton initiated $qqZZ$ samples corresponding to Figure \ref{fig:ZZjjFeynmanDiag_QCD_qq} are generated with NLO accuracy in QCD up to one additional parton emission and LO accuracy for up to three additional partons emission. The loop-induced $ggZZ$ samples emerging at NNLO in $\alpha_{S}$ corresponding to Figure \ref{fig:ZZjjFeynmanDiag_QCD_gg} are generated using LO-accurate matrix elements for up to one additional parton emission \cite{EventGenWithSherpa}. The generator uses an NNPDF3.0NNLO PDF set evaluated using different measurements from several experiments, such as deep-inelastic inclusive cross-sections measurement from HERA-II, the combined charm data from HERA, jet production, vector boson rapidity and transverse momentum measurements from ATLAS, CMS and LHCb, total cross sections of top quark pair production from ATLAS and CMS and W+c data from CMS \cite{PDFForRunII}. Parton showering is done by \textsc{Sherpa}'s internal algorithm based on Catani–Seymour dipole factorization matrix element \cite{SherpaPS}. The matrix element calculations are matched and merged using the $ME+PS@NLO$ prescription \cite{PSMatching}. The matching process requires the hard jets to match the parton-level quarks and gluons from matrix elements, whereas the merging allows the MC generators to merge two or more low-mass jets. 

An alternative \textsc{MadGraph5} samples produced at NLO accuracy for up to one additional parton emission and LO accuracy for up to three additional parton emission \cite{MADGRAPHNLO} are also used in the measurement for the parton induced $qqZZ$ samples. The generator uses the A14NNPDF23LO PDF set, and the ME is interfaced with \textsc{Pythia8} for parton showering, merging, and matching \cite{Pythia8}. 

The EWK production $qqZZjj$ $[qq \rightarrow ZZ^{*}(\rightarrow 4 \ell) jj]$ is simulated using a \textsc{PowhegV2} generator using an MSTW2008 PDF set with NLO accuracy in QCD and interfaced with \textsc{Pythia8} for parton showering and hadronization \cite{PowhegV2}. An alternative sample at LO accuracy is also used in the measurement from \textsc{MadGraph5} with A14NNPDF23LO PDF set and \textsc{Pythia8} showering \cite{MADGRAPHNLO}. The \textsc{PowhegV2} NLO prediction of electroweak $qqZZjj$ does not contain the contribution from electroweak triboson $VZZ$ processes where two vector bosons decay leptonically and one decay hadronically. The contribution from these electroweak triboson processes is predicted using the {Sherpa} $2.2.2$ MC generator at LO accuracy for up to two additional parton emissions and added to the \textsc{PowhegV2} predictions. Table \ref{tab:SigMC} summarizes the signal MC used in the measurement. 

\begin{table}[!htbp]
\footnotesize
\centering
\begin{tabular}{l l c c c }
\hline\hline
Process & Description & Generator  & PDF & Accuracy\\
\hline \hline
QCD $qqZZ$ &        &        &       &   \\
\multirow{2}{*}{ $q\bar{q} \rightarrow ZZ^{*}( \rightarrow 4\ell) jj$ } 
                    & \multirow{2}{*}{inclusive} & \textsc{Sherpa}$2.2.2$ & NNPDF3.0NNLO & \multirow{2}{*} {$0,1 j @NLO + 2,3 j @LO $} \\ 
        &  & \textsc{MadGraph} & A14NNPDF23LO & \\
        &       &        &       &   \\
\hline
QCD $ggZZ$ loop&        &        &       &   \\
 $gg \rightarrow ZZ^{*} (\rightarrow 4\ell) jj$ &  $m_{4 \ell } > 130$ GeV  & \textsc{Sherpa}$2.2.2$ & NNPDF3.0NNLO & $0,1 j @LO $ \\

&       &        &       &   \\

\hline 
EWK $qqZZjj$ &      &        &       &   \\
\multirow{2}{*}{ $q\bar{q} \rightarrow ZZ^{*}( \rightarrow 4\ell) jj$ } 
                    & \multirow{2}{*}{$m_{4\ell} > 130$ GeV } & \textsc{PowhegV2} & MSTW2008 &  $\ge 2 j$ (EWK) @ NLO QCD \\ 
        &   & \textsc{MadGraph} & A14NNPDF23LO & $\ge 2 j$ (EWK) @LO \\
        &       &        &       &   \\
\hline 
EWK $VZZ$ & & & & \\
$q\bar{q} \rightarrow VZZ^{*} \rightarrow 4\ell jj$ &       &   \textsc{Sherpa}$2.2.2$   &  NNPDF3.0NNLO     & $1,2j @LO$    \\
\hline\hline

\end{tabular}
\normalsize
\caption{List of signal MC samples used in the analysis. Each process consists of three different generation campaigns corresponding to the data-taking conditions of the ATLAS Run2 data-taking periods.\\ \label{tab:SigMC}}
\end{table}

\subsubsection{Background Samples}
\label{subsubsec:BkgSamples}

In addition to the QCD and EWK production discussed above, two other processes, triboson ($WWZ, ~WZZ, ~ZZZ$) and $Z$-bosons production in association with a top quark pair ($t\bar{t}Z$), also contributes to the $ 4\ell jj$ final state. The triboson processes are modeled with \textsc{Sherpa}$2.2.2$ generator at NLO accuracy in QCD for zero or one additional parton emissions and LO accuracy for up to two additional parton emissions. The triboson samples only include the fully leptonic decays of the vector bosons. Therefore, there is no overlap between the background triboson and the signal EWK $qqZZjj$ samples. The $t\bar{t}Z$ processes are modeled by \textsc{Sherpa}$2.2.0$ generator at LO accuracy with up to one additional parton emission using the MEPS@LO set-up \cite{Sherpa220}. The same algorithms as in the QCD $qqZZ$ sample generation are used for parton showering, matching, and merging. The MC simulation of the triboson and $t\bar{t}Z$ samples are subtracted directly from the data. Table \ref{tab:BkgMC} summarizes the details of these samples. 

\begin{table}[!htbp]
\footnotesize
\centering
\begin{tabular}{l l c c c }
\hline\hline
Process & Description & Generator  & PDF & Accuracy\\
\hline \hline
 &      &        &       &   \\
 $pp \rightarrow W^{(*)}W^{(*)}Z^{(*)} \rightarrow 4\ell 2\nu $  & \multirow{3}{*}{inclusive} & \textsc{Sherpa}$2.2.2$ & \multirow{3}{*}{NNPDF3.0NNLO} & \multirow{3}{*}{$0,1 j @NLO + 2 j @LO $} \\ 
 
$pp \rightarrow W^{(*)}Z^{(*)}Z^{(*)} \rightarrow 5\ell 1\nu$  &  & \textsc{Sherpa}$2.2.2$ &   &  \\ 
$pp \rightarrow Z^{(*)} Z^{(*)} Z^{(*)} \rightarrow 6\ell $ &  & \textsc{Sherpa}$2.2.2$ &  &  \\ 
        
\hline 
&       &        &       &   \\
$pp \rightarrow t\bar{t}+Z(\rightarrow 2\ell)$ & $m_{ll} > 5$ GeV & \textsc{Sherpa}$2.2.0$ & NNPDF3.0NNLO & LO \\

\hline\hline
\end{tabular}
\normalsize
\caption{List of background MC samples used in the analysis. Each process consists of three different generation campaigns corresponding to the data-taking conditions of the ATLAS Run2 data-taking periods.\\ \label{tab:BkgMC}}
\end{table}

\subsubsection{Samples for Non-prompt Background}
\label{subsubsec:FakeBkgSamples}
In addition to the triboson and $t\bar{t}$Z samples, the analysis has additional backgrounds coming from events with one or more non-prompt or fake leptons. These non-prompt backgrounds are estimated using a data-driven method discussed in detail in Section \ref{subsec:FakeBackground}. MC samples are used to develop and validate the data-driven non-prompt background estimation procedure. Three sources of events could contribute as a source for non-prompt background events. The first type of events is from a Z-boson production in association with jets $pp \rightarrow Z^{*} (\rightarrow 2\ell) +jets$, which is simulated for both three or more leptons using \textsc{Sherpa}$2.2.1$. The subdominant process is events from $t\bar{t}\rightarrow 2\ell$ production in which both top quarks decay semileptonically, which is simulated with \textsc{Powheg+Pythia8} and uses the A14NNPDF23LO PDF set \cite{PowhegPythia}. The third type of non-prompt backgrounds arises from the WZ production in which both bosons decay leptonically $pp \rightarrow WZ \rightarrow 3 \ell 1\nu $ and is simulated using \textsc{Sherpa}$2.2.2$. Table \ref{tab:FakeBkgMC} summarizes the different processes and MC generators used in various studies related to the data-driven fake factor method to estimate the non-prompt background. A simulated non-prompt background event in the $4\ell jj$ final state could comprise an event with a quadruplet formed from one or more fake leptons from a non-prompt source, such as misidentification and other remaining leptons from these three physics processes.

\begin{table}[!htbp]
\footnotesize
\centering
\begin{tabular}{l l c c c }
\hline\hline
Process & Description & Generator  & PDF & Accuracy\\
\hline \hline
 &      &        &       &   \\
 $pp \rightarrow Z^{*} (\rightarrow 2e)+jets $  & \multirow{3}{*}{inclusive} & \multirow{3}{*}{\textsc{Sherpa}$2.2.2$} & \multirow{3}{*}{NNPDF3.0NNLO} & \multirow{3}{*}{$NLO+2j,LO+4j $} \\ 
 
$pp \rightarrow Z^{*} (\rightarrow 2\mu) +jets $  &  &  &   &  \\ 
$pp \rightarrow Z^{*} (\rightarrow 2\tau) +jets $ &  &  &  &  \\ 
        
\hline 
&       &        &       &   \\
$pp \rightarrow t\bar{t} \rightarrow 2\ell + X $ & inclusive & \textsc{Powheg+Pythia8} & A14NNPDF23LO & LO \\
\hline 
&       &        &       &   \\
$pp \rightarrow WZ \rightarrow 3 \ell 1\nu $ & inclusive & \textsc{Sherpa}$2.2.2$ & NNPDF3.0NNLO & $NLO + 1j, LO+3j $\\
\hline\hline

\end{tabular}
\normalsize
\caption{List of MC samples used in the estimation and validation of the data-driven non-prompt background estimation.\\ \label{tab:FakeBkgMC}}
\end{table}

\subsection{Event Weights}
\label{subsec:EventWt}

The predictions from the MC generators are often generated with a higher effective luminosity than the data to reduce the statistical uncertainties in simulations. Therefore, the raw predictions of the MC are completely unscaled and cannot be compared to the data recorded by the detector directly. Different weights need to be accounted for to scale each generated MC event.

The MC-generated events beyond LO have a generator event weight which is needed once a sufficient number of events are simulated to attain the correct cross-section distribution. Some generated events have negative weight to account for the cancellations caused by interference that arises during matrix element calculation. The generator event weight also accounts for the Sudakow form factors associated with QCD emissions in parton showering. Each MC-generated event first needs to be scaled based on its generator event weight and normalized by the total sum of all the generated event weights. Once the event weight is multiplied by the cross-section and the integrated luminosity of the data, the distribution is correctly normalized. 

A higher-order cross-section calculation might be available for some physics processes. Such predictions are re-weighted by scaling the event weight using the available \textit{k-factor}. Moreover, the simulations of certain physics processes are computationally intensive, and MC generators might impose kinematic filters to simulate these processes. The efficiency of such filters, $\epsilon_{filt}$, are also considered in the event weight. 

A set of corrections in the event weight are related to detector measurements. As discussed in Section \ref{sec:ParticleReconstruction}, the \textit{scale factors (SF)} correct the efficiencies in MC related to the reconstruction, identification, isolation, and trigger to match that of the measured data. These scale factors are applied to the event weight. Finally, the pile-up re-weighting corrects the MC event weights to match the distribution of the average number of interactions per bunch crossing observed in the data and is also considered in the event weight.

Therefore, each event weight in MC can be represented as, 
\begin{equation}
w_{event} = \frac{\sigma~\cdot~\text{k-factor}\cdot~~\epsilon_{filt}~\cdot w_{generator} \cdot {L_{data} \cdot w_{reco}}}{\sum{w_{generator}}},
\label{eqn:EventWeight}
\end{equation}
where $\sigma$ is the cross-section of the process, $w_{generator}$ is the generator event weight, $L_{data}$ is the integrated luminosity from data, and $w_{reco}$ accounts for all detector measurement related re-weighting in detector-level predictions.
	\section{Definition of Measured Observables}
\label{sec:Obs}

The primary results of the thesis are the unfolded differential cross-sections of the following eleven different kinematic observables:

\begin{itemize}

\item{	$m_{4\ell}$: invariant mass of the four-leptons (or $2$ $Z$-bosons)}
\item{ 	$m_{jj}$:  invariant mass of the dijet}
\item{	$p_{T,4\ell}$: transverse momentum of the four-leptons }
\item{	$p_{T, jj}$: transverse momentum of the dijet }
\item{	$p_{T,4\ell jj}$: transverse momentum of the four-leptons and the dijet }
\item{	$s_{T,4\ell jj}$: scalar transverse momentum of the four-leptons and the dijet }
\item{ 	$\Delta \phi _{jj}^{signed}$: difference in the azimuthal angle between the two jets in the dijet, ordered according to their rapidity,i.e. 
\begin{align*}
	\Delta \phi _{jj}^{signed} = 
	\begin{cases}
	\phi(j_1)-\phi(j_2) & \text{if $y_{j_1} > y_{j_2}$}\\
	\phi(j_2)-\phi(j_1) & \text{otherwise}
	\end{cases} 
\end{align*}
}
\item{ $\Delta y_{jj}$: the absolute value of rapidity difference between the leading and the sub-leading jets in the dijet}
\item{ $\zeta$: centrality of the system}
\item{ $\cos \theta^{*}_{\ell 1 \ell 2}$: cosine of the decay angle of the negative lepton of the leading pair in the pair's rest frame as shown by Figure \ref{fig:costhetaFrameOfRef}}
\item{ $\cos \theta^{*}_{\ell 3 \ell 4}$: cosine of the decay angle of the negative lepton of the sub-leading pair in the pair's rest frame as shown by Figure \ref{fig:costhetaFrameOfRef} }

\end{itemize}

\begin{figure}
\centering
\includegraphics[width=.8\linewidth]{figures/AnalysisOverview/costhetaFrameOfRef.png}
\caption{Figure showing the decay angle $\theta^{*}_{\ell 1 \ell 2}$ ( $\theta^{*}_{\ell 3 \ell 4}$ ) of the negative lepton in the primary (secondary) pair's rest frame.\label{fig:costhetaFrameOfRef}\cite{AngularFrameDef}.}
\end{figure}

\clearpage

\part {\LARGE{Analysis Strategy}}
\label{sec:Analysis Strategy}

\section{Background}
\label{sec:Bkg}
\subsection{Data Driven Estimate of Fake Background}
\label{subsec:FakeBackground}

\subsubsection{Lepton Composition}
\label{subsubsec:LepComp}

\subsubsection{Control Regions}
\label{subsubsec:CR}

\subsubsection{Fake Efficiency}
\label{subsubsec:FakeEff}

\subsubsection{Method Validation}
\label{subsubsec:Validation}

\subsubsection{Signal Region Estimation}
\label{subsubsec:SREstimation}

\section{Unfolding}
\label{sec:Unfolding}
The main results of the thesis are differential cross-section measurements at the particle level. The inclusive detector level cross-section for a given physics process $p_{1}p_{2}\rightarrow X$ is, 

\begin{equation}
    \sigma ^{detector~level}_{p_{1}p_{2}\rightarrow X} = A \times \epsilon \times \sigma ^{particle-level}_{p_{1}p_{2}\rightarrow X}
    \label{eqn:InclusiveXS}
\end{equation}

where $\sigma ^{particle-level}_{p_{1}p_{2}\rightarrow X}$ is the \textit{true} cross-section of the physics process predicted by the theory. The physical layout of the ATLAS detector does not cover all areas of the phase space. $A$ accounts for the limited acceptance of the ATLAS detector. Several parts of the detector have several reconstruction efficiencies, which are accounted for by the factor $\epsilon$. The detector level cross-section is measured experimentally in terms of the number of particles in a given final state (N) and integrated Luminosity $L$ as $\sigma ^{detector~level}_{p_{1}p_{2}\rightarrow X} = \frac{N}{L}$. The \textit{true} particle level inclusive cross-section can be estimated by correcting for detector acceptance and detector efficiency for the measured cross-section $\sigma ^{detector~level}_{p_{1}p_{2}\rightarrow X}$.

For differential cross-sections where the cross-section is measured in different bins of the kinematic observables, additional correction is needed to correct the resolution-induced migration between nearby bins. 

This Chapter discusses the unfolding technique in detail. Section \ref{subsec:UnfoldingOverview} gives an Overview on the unfolding algorithm, whereas Section \ref{subsec:UnfoldingValidation} validates the unfolding method. Section \ref{subsec:Bias} discusses the bias from unfolding and the attempts to optimize the bias.  

\subsection{Method Overview}
\label{subsec:UnfoldingOverview}
The analysis uses an \textit{iterative Bayesian unfolding} algorithm based on Baye's theorem \cite{BayesianUnfolding} \cite{Improved_BayesianUnfolding} using ATLAS-supported \textit{RooUnfold} package \cite{RooUnfold}. Bayes' theorem formulates a mathematical relation to obtain a probability of an effect $E$ caused by several independent causes $C_{i}$, given the initial probability of the causes $P(C_{i})$ and the conditional probability of the $i-th$ cause to produce the effect $P(E|C_{i})$ as, 

\begin{equation}
P(C_{i}|E) = \frac{ P(E|C_{i}) . P(C_{i}) } { \sum_{j}{ P(E|C_{j}).P(C_{j}) } }
\label{eqn:BayesTheorem}
\end{equation}

The obtained probability depends on the prior probability of the cause and the conditional probability of cause and effect. The prior dependency is reduced by using an iterative technique, where the outcome of the previous iteration is used as a prior for the subsequent iteration.

For a single iteration, the algorithm can be summarized as, 

\begin{equation}
    U_{i} = \frac{1}{ \epsilon_{i} } \times \sum^{reco~bins}_{j}{ (R_j -F_j ) . f_{i} . \frac{M_{ji} T_{i}}{ \sum_{k}^{truth~bins}{M_{jk} T_{k}}} } 
    \label{eqn:BayesianUnfolding}
\end{equation}

where $U_{i}$ is the unfolded yield in the target bin $i$, $T_{i}$ is the predicted truth level yield in particle bin $i$, $R_{j}$ is the observed detector level yield in reco bin $j$ and $F_{j}$ is the subtracted detector level reducible background yield. $M_{ij}$ is the migration matrix element from particle level bin $j$ to detector level bin $i$. 

Based on the discussion, conceptually, three corrections from the SM MC prediction need to be applied to estimate the unfolded yield. The three unfolding inputs are 

\begin{itemize}
    \item{\textit{\textbf{Reconstruction efficiency ($\epsilon$):}} The reconstruction efficiency accounts for the limited acceptance and efficiency of the detector. Technically, it is defined as a fraction of events that pass both detector and fiducial level selection to the events passing only the fiducial level selection. }
    
    \item{\textit{\textbf{Fiducial fraction ($f$):}} The fiducial fraction accounts for events that are outside the fiducial region at the particle level, which due to limited detector resolution entered in the measured distribution. An example of such an event would be a signal $4\ell+jj$ event where one of the jets originates from pile-up instead of hard-scatter. Technically, it is defined as a fraction of events that pass both detector and fiducial level selection to the events passing only the detector level selection. }
    
    \item{\textit{\textbf{Migration matrix ($M_{ij}$):}} The migration matrix is a two-dimensional matrix that accounts for events migrated from particle level bin $j$ to detector level bin $i$. The migration matrix corrects the probability of bin migration. It is measured in MC by comparing particle and detector levels distributions for events that pass both fiducial and detector-level selections. Bin migrations result from resolution effects and smearing of the reconstructed distributions. The diagonal component of the migration matrix is related to the \textit{fiducial purity}, which corresponds to the fraction of detector-level events that originate from the same bin at the particle level.}
\end{itemize}

Figure \ref{fig:UnfoldingInputs} show all three unfolding inputs along with the purity as a function of $m_{jj}$. The reconstruction efficiency is less than $50\%$ caused by the poor jet reconstruction efficiency. The fiducial fraction and purity is smaller in lower bins of $m_{jj}$, which mainly corresponds to contribution from pileup jets faking the event selection. The normalized migration matrix shown in the second plot with the particle level prediction in $y-axis$ and the detector level prediction in $x-axis$ is diagonal.

\begin{figure}[htb]
    \centering
    \begin{subfigure}{.48\textwidth}
        \centering
        \includegraphics[width=.9\linewidth]{figures/Analysis/Unfolding/efficiencies_VBS_Enhanced.pdf}
        \caption{ reconstruction efficiency (red), fiducial fraction (yellow) and purity (blue). }
    \end{subfigure}
    \begin{subfigure}{.48\textwidth}
        \centering
        \includegraphics[width=.9\linewidth]{figures/Analysis/Unfolding/migration_matrix_VBS_Enhanced.pdf}
        \caption{migration matrix}
    \end{subfigure}
    \caption{ Unfolding inputs from SM MC as a function of $m_{jj}$.\textcolor{red}{remake first plot with ATLAS Label and stability} \label{fig:UnfoldingInputs}}
\end{figure}

\subsection{Binning for Unfolding}
\label{subsec:Binning}
Choosing optimal binning to perform the unfolding procedure for all kinematic observables effectively is imperative. Two factors drive the choice of binning; first, the necessity to have large enough bin statistics to maintain the Gaussian approximation while preserving the shape of the differential distributions, and second, the necessity to minimize large bin migrations and statistical uncertainties from unfolding. Therefore, each bin must have at least $15$ events in the VBS-Suppressed region and at least $20$ events in the VBS-Enhanced signal region. 

To maintain a good performance of the unfolding, each bin for the kinematic observable has at least $70\%$ purity except for $p_{T,4\ell jj}$ where at least $50\%$ purity is required. Moreover, for each observable, every bin width must be equal to or greater than the resolution of the same bin. The resolution in each particle-level bin is evaluated from MC by comparing the difference of particle and detector level yield for events that pass both fiducial- and detector-level event selection. The difference is fitted using Gaussian approximation, and twice the resulting standard deviation is taken as the resolution. Table \ref{tab:binning} shows the final bin choices for all the kinematic observables used in differential cross-section measurement.
.

\begin{table}
    \caption{Binning for all unfolded observables in VBS-Enhanced and suppressed regions. \label{tab:binning}}
    \begin{center}
    \begin{tabular}{ | c | c | c | }
    \hline
    Observable & Region & Binning \\
    \hline \hline
    \multirow{4}{*}{ $m_{jj}$ [GeV] } &  &  \\
        & VBS-Enhanced & [300, 400, 530, 720, 1080, 3280] \\
    & VBS-Suppressed & [300, 410, 600, 178] \\
    & &\\
    \hline
    \multirow{4}{*}{ $m_{4\ell}$ [GeV] } &  &  \\
        & VBS-Enhanced & [130, 210, 250, 304, 400, 1130] \\
    & VBS-Suppressed & [130, 226, 304, 752] \\
    & &\\
    \hline
    \multirow{4}{*}{ $p_{T,4\ell}$ [GeV] } &  &  \\
    & VBS-Enhanced & [0, 50, 80, 116, 174, 512] \\
    & VBS-Suppressed & [0, 76, 140, 424]\\
    & &\\
    \hline
    \multirow{4}{*}{ $p_{T,jj}$ [GeV] } &  &  \\
    & VBS-Enhanced & [0, 52, 82, 116, 172, 524] \\
    & VBS-Suppressed & [0, 80, 146, 448]\\
    & &\\
    \hline
    \multirow{4}{*}{ $p_{T,4\ell jj}$ [GeV] } &  &  \\
    & VBS-Enhanced & [0, 20, 42, 64, 298] \\
    & VBS-Suppressed & [0, 36, 70, 254]\\
    & &\\
    \hline
    \multirow{4}{*}{ $s_{T,4\ell jj}$ [GeV] } &  &  \\
    & VBS-Enhanced & [70, 240, 320, 420, 580, 1410] \\
    & VBS-Suppressed & [70, 330, 500, 1210]\\
    & &\\
    \hline
    \multirow{4}{*}{ $|\Delta y_{jj}|$ } &  &  \\
    & VBS-Enhanced & [2, 3.08, 3.74, 4.32, 5.06, 7.4] \\
    & VBS-Suppressed & [2, 2.94, 3.78, 5.4]\\
    & &\\
    \hline
    \multirow{4}{*}{ $\Delta \phi_{jj}^{signed}$ } &  &  \\
    & VBS-Enhanced & [$-\pi$, -2.1, 0, 2.1, $\pi$] \\
    & VBS-Suppressed & [$-\pi$,0,$\pi$] \\
    & & \\
    \hline
    \multirow{4}{*}{ $cos \theta_{\ell i\ell j}^{\ast}$ } &  &  \\
    & VBS-Enhanced & [-1, -0.5, 0, 0.5, 1] \\
    & VBS-Suppressed & [-1, 0, 1]\\
    & & \\
    \hline
    \multirow{4}{*}{ $\zeta$ } &  &  \\
    & VBS-Enhanced &[0, 0.06, 0.12, 0.18, 0.26, 0.4] \\
    & VBS-Suppressed & [0.4, 0.5, 0.64, 1.02]\\
    & &\\
    \hline
    \end{tabular}
    \end{center}
\end{table}

\subsection{Method Validation}
\label{subsec:UnfoldingValidation}
The unfolding method is validated using three different tests.

\subsubsection{MC Closure Test}
\label{subsubsec:MCClosure}

The first validation of the unfolding technique is with the SM MC. An SM-predicted detector level distribution for a kinematic observable is unfolded using the unfolding inputs from the same MC. Figure \ref{fig:unfolding_technical_closure} shows an example of the MC-based closure test for $m_{jj}$ in the VBS-Enhanced region. The blue detector-level MC prediction is unfolded using the inputs from the same MC, and the resulting black unfolded distribution is compared with the red particle-level prediction. Since both detector-level prediction and unfolding inputs are from the same MC, a perfect closure between the unfolded and particle-level distribution is observed.

\begin{figure}[!htb]
\centering
\includegraphics[width=.6\textwidth]{figures/Analysis/Unfolding/technical_closure_VBS_Enhanced.pdf}
\caption{MC technical closure test of the unfolding procedure for $m_{jj}$. The detector-level MC distribution (in blue) is unfolded with the nominal SM unfolding inputs and compared to the particle-level distribution (in red) from the same MC. A perfect closure between unfolded and particle level distribution is observed\label{fig:unfolding_technical_closure}}
\end{figure}

\subsubsection{Injection Test}
\label{subsubsec:InjectionTest}
The analysis uses a model-independent EFT approach discussed in Section \ref{sec:EFT} to constrain the effect of BSM physics. Therefore, it is essential to test the ability of the unfolding algorithm to uncover the accurate particle-level prediction from data containing BSM physics via injection test. In an injection test, a BSM physics contribution is added to the SM detector-level prediction, unfolded with the nominal SM unfolding inputs, and compared with the BSM-added particle-level distribution. Figure \ref{fig:Dim8cont} shows an injection test for $m_{jj}$ in the VBS-Enhanced region where a BSM contribution (green distribution) is added to the SM MC. The BSM contribution is from linear and quadratic contributions of an $FT0$ EFT operator. Figure \ref{fig:InjectTestResult} shows the result of the injection test. The BSM-added detector-level MC prediction (blue) is unfolded (black) using nominal SM MC unfolding inputs and compared against the BSM-added particle-level distribution (red). A small non-closure of about $5\%$ in the last bin of $m_{jj}$ is observed, which is well within the uncertainties of the unfolded distribution.

\textcolor{red}{Note to self: perhaps it makes sense to discuss EFT theory motivation in theory section?}

\begin{figure}[htb]
    \centering
    \begin{subfigure}{.48\textwidth}
        \centering
        \includegraphics[width=.9\linewidth]{figures/Analysis/Unfolding/injection_test_FT0_quad_mjj_detectorPred.pdf}
        \caption{ Detector level MC prediction with contribution from dimension$-8$ $FT0$ EFT operator. \label{fig:Dim8cont} }
    \end{subfigure}
    \begin{subfigure}{.48\textwidth}
        \centering
        \includegraphics[width=.9\linewidth]{figures/Analysis/Unfolding/injection_test_FT0_quad_mjj.pdf}
        \caption{Unfolded SM+EFT MC detector-level distribution with response matrix from SM MC. \label{fig:InjectTestResult}}
    \end{subfigure}
    \caption{ Injection test with  dimension$-8$ $FT0$ EFT operator. \textcolor{red}{remake plots with ATLAS Label} \label{fig:injection_test_FT0_quad}}
\end{figure}

\subsubsection{Physics Variation}
FFrom the previous ATLAS electroweak $ZZjj$ analysis, a slight enhancement on the central value of the EWk $ZZjj$ cross-section was measured \cite{ATLASZZjj}. The final unfolding validation tested the ability of the algorithm to recover the actual shape of particle-level distribution if a physics process cross-section was different from the SM prediction. First, as shown by figure \ref{fig:unfolding_xsec_var_QCD}, the cross-section for parton-initiated QCD $qqZZjj$ is varied by a factor equal to the total statistical uncertainty on data in the VBS-Suppressed region $\pm  15\%$. The varied detector-level distribution is then unfolded using the nominal SM MC unfolding inputs and compared with the varied fiducial level prediction. Figure \ref{fig:unfolding_xsec_var_EWK} shows the same test where the $EWK qqZZjj$ cross-section is varied by $\pm 11\%$ based on the enhanced cross-section observed in the previous measurement. In both cases, a non-closure of about $1\%$ is observed, well below the uncertainties from unfolding.

\begin{figure}[htb]
    \centering
    \begin{subfigure}{.48\textwidth}
        \centering
        \includegraphics[width=.9\linewidth]{figures/Analysis/Unfolding/QCD_xsec_variation.pdf}
        \caption{ QCD cross-section is varied by $\pm  15\%$ \label{fig:unfolding_xsec_var_QCD} }
    \end{subfigure}
    \begin{subfigure}{.48\textwidth}
        \centering
        \includegraphics[width=.9\linewidth]{figures/Analysis/Unfolding/EWK_xsec_variation.pdf}
        \caption{ EWK cross-section is varied by $\pm 11\%$ \label{fig:unfolding_xsec_var_EWK}}
    \end{subfigure}
    \caption{ Unfolding validation using physics variation where parton-initiated QCD (left) or the EWK process cross-sections are varied. \label{fig:unfolding_xsec_var}}
\end{figure}

\subsection{Bias and Optimization}
\label{subsec:Bias}

The unfolded procedure relies on a prior value depending on the SM MC which naturally biases the unfolded cross-sections. With each iteration of unfolding, the algorithm improves the knowledge of the prior, thus, reducing the unfolding bias. However, with increasing number of iterations, the repeated bin migrations amplifies the statistical fluctuations in data, resulting in larger values of statistical uncertainties. Therefore, a finite number of iteration is chosen and the resulting unfolding bias is taken as the systematic uncertainty for the measurement. 

The unfolding bias is evaluated by the \textit{data-driven closure test}, where a pseudo dataset is developed utilizing the ratio of observed data and SM-predicted detector-level yield. First, for each observable the data and MC ratio is smoothed using Friedman's Super Smoother technique \cite{FriedmanSmoother}, fixing the end points to the value of ratio in the first and last bins. A reweighing function for each observable is developed to reweigh the SM fiducial- and detector-level yields. The reweighed detector-level signal-yield is then unfolded with the nominal unfolding inputs from SM and compared with the reweighed fiducial-level yield to get the final unfolding bias. Figure \ref{fig:unfolding_ddclosure} shows step-by-step procedure for the data-driven closure test. As shown by the ratio panel of figure \ref{fig:ddclosure_FinalBias}, unfolding bias of order $10\%$ is observed. 

\begin{figure}[htb]
    \centering
    \begin{subfigure}{.48\textwidth}
        \centering
        \includegraphics[width=.9\linewidth]{figures/Analysis/Unfolding/DDClosure_VBS_Suppressed_Ratio.pdf}
        \caption{ Data and MC for $m_{jj}$ \label{fig:ddclosure_DataMC}}
    \end{subfigure}
    \begin{subfigure}{.48\textwidth}
        \centering
        \includegraphics[width=.9\linewidth]{figures/Analysis/Unfolding/DDClosure_VBS_Suppressed_SmoothRatio.pdf}
        \caption{Smoothed ratio of Data and MC. \label{fig:ddclosure_DataMCSmooth} }
    \end{subfigure}\\
    \begin{subfigure}{.48\textwidth}
        \centering
        \includegraphics[width=.9\linewidth]{figures/Analysis/Unfolding/DDClosure_VBS_Suppressed_Reweighted.pdf}
        \caption{ Nominal SM (red) detector level yield and reweighted-detector level yield(green). \label{fig:ddclosure_DataMCReweighted} }
    \end{subfigure}
    \begin{subfigure}{.48\textwidth}
        \centering
        \includegraphics[width=.9\linewidth]{figures/Analysis/Unfolding/DDClosure_VBS_Suppressed_Bias.pdf}
        \caption{Unfolding bias. \label{fig:ddclosure_FinalBias} }
    \end{subfigure}
    \caption{ A step-by-step overview of the data driven closure test to get the unfolding bias.  \textcolor{red}{remake plots with ATLAS Label} \label{fig:unfolding_ddclosure}}
\end{figure}

The bias observed in figure \ref{fig:ddclosure_FinalBias} is obtained by using one number of iteration for unfolding. With a goal to reduce the unfolding bias, the data-driven closure test was repeated for several number of iterations. The resulting unfolding bias and systematic uncertainties up to $4$ iterations are shown in figure \ref{fig:BiasStatUnc}. As expected the unfolding bias decreases whereas the statistical uncertainty increases with the higher number of iteration. To balance between the statistical uncertainty and bias uncertainty, one number of iteration is chosen as optimal choice for the measurement.

\begin{figure}[htb]
    \centering
    \begin{subfigure}{.49\textwidth}
        \centering
        \includegraphics[width=.9\linewidth]{figures/Analysis/Unfolding/UnfoldingBiasIteration.pdf}
       % \caption{ Unfolding bias \label{fig:UnfoldingBiasIteration} }
    \end{subfigure}
    \begin{subfigure}{.49\textwidth}
        \centering
        \includegraphics[width=.9\linewidth]{figures/Analysis/Unfolding/StatUnc_Sup.pdf}
        %\caption{ Statistical uncertainty \label{fig:UnfoldingStatUnc}}
    \end{subfigure}
    \caption{ Unfolding bias (left) and statistical uncertainty (right) with up to $4$ unfolding iterations as a function of $m_{jj}$ in VBS-Suppressed region. \label{fig:BiasStatUnc}}
\end{figure}

Unfolding bias is the largest source of the systematic uncertainty of the analysis and is studied in detail using a MC-driven toy studies to understand the source. The observed large bias is from detector-level pileup jets at lower $p_{T}$ or higher $\eta$ that are not part of the fiducial phase space. The jet-vertex-tagger and forward-jet-vertex-tagger has lower efficiency to select the hard scattering jets at lower $p_{T}$ or higher $\eta$, thus resulting in more \textit{fiducial-fake-event} contamination. The additional MC-based studies on the unfolding bias are summarized in Appendix \ref{Appendix:Unfolding_bias}. 


\section{Uncertainties on the Measurement}
\label{sec:Uncertainties}
The differential cross-section measurements discussed in this thesis are affected by three sources of systematic uncertainties, experimental sources, theoretical sources, and intrinsic systematics related to the unfolding process. The statistical uncertainty of the measurements is dominant as data statistics limit the cross-section measurements. This section discusses the source of theoretical, experimental, unfolding uncertainties and propagation of the statistical uncertainties to the unfolded cross-sections. 

\subsection{Theoretical Uncertainties}
\label{subsec:TheoryUnc}

The following sources of theoretical uncertainties are considered in the measurement.

\begin{itemize}
\item{\textbf{Uncertainties on QCD Scale:} As discussed in Section \ref{sec:Pheno}, the theoretical predictions of cross-sections depend on the factorization scale ($\mu_{F}$) and renormalization scale ($\mu_{R}$) \cite{QCDScaleAndPDFUnc}. To account for this dependence, a QCD scale uncertainty is evaluated by scaling $\mu_{F}$ and $\mu_{R}$ independently using on-the-fly variations provided by the MC generators. The variations constitute of six-point variations of $\mu_{F}$ and $\mu_{R}$ from $-50\%$ or $+100\%$ around their nominal values of 1, such that \{$\mu_R = 0.5, \mu_F = 0.5$\}, \{$\mu_R = 0.5, \mu_F = 1.0$\}, \{$\mu_R = 1.0, \mu_F = 0.5$\}, \{$\mu_R = 1.0, \mu_F = 2.0$\}, \{$\mu_R = 2.0, \mu_F = 1.0$\}, and \{$\mu_R = 2.0, \mu_F = 2.0$\}. The final uncertainty is evaluated as the absolute envelope of the six variations. The QCD scale uncertainties are evaluated for $qqZZ$, $ggZZ$, and $EWK qqZZjj$ samples. \textcolor{red}{to add intermediate systematics plot} %Figure \ref{fig:QCDScale} shows different variations and the final envelope for QCD Scale uncertainties in the VBS Enhanced region as a function of $m_{jj}$. 

% \begin{figure}
%     \centering
%     \includegraphics[width=.8\linewidth]{figures/AnalysisOverview/mu_ProfileRun2.pdf}
%     \caption{ Evaluation of the fiducial level QCD factorization and renormalization scale uncertainties band for $qqZZ$ sample. \label{fig:QCDScale}}
% \end{figure}
}
\item{\textbf{Uncertainties on PDF $\&$ $\alpha_{S}$:} The cross-sections also depend on the choice of the PDF used by the MC generators. Thus, the PDF uncertainty for Sherpa and \textsc{MadGraph5} samples that use NNPDF3.0NNLO is evaluated using the prescription described in Ref \cite{PDFForRunII} using on-the-fly variation weights. The PDF variations include a set of 100 internal variations, two additional variations from the nominal PDF reweighted to the alternative MMHT2014nnlo \cite{MMHT2014PDFs} $\&$ CT14nnlo \cite{CT14nnlo} PDF sets and variations of the strong coupling constant by $\pm0.0001$ where the nominal value of $\alpha_{S}$ is $0.118$. The total uncertainty is taken as the absolute envelope of all standard deviations of $100$ internal variations, the two alternate PDF variations, added in quadrature with the envelope of the $\alpha_{S}$ variations, 
\begin{equation}
    \sigma_{PDF}^{NNPDF3.0NNLO} = \sqrt{ [ max (\sigma_{std.~dev.~int.}, |\sigma_{MMHT2014nnlo}| , |\sigma_{CT14nnlo}|~)]^2 + \sigma_{\alpha_S}^2 }
\end{equation}

The PDF uncertainty is evaluated for $qqZZ$, $ggZZ$ and \textsc{MadGraph5} $EWK qqZZjj$ samples. \textcolor{red}{to add intermediate systematics plot}  %Figure \ref{fig:PDFScale} shows different variations and the final envelope for QCD Scale uncertainties in the VBS Enhanced region as a function of $m_{jj}$. 

% \begin{figure}
%     \centering
%     \includegraphics[width=.8\linewidth]{figures/AnalysisOverview/mu_ProfileRun2.pdf}
%     \caption{ Evaluation of the fiducial level QCD factorization and renormalization scale uncertainties band for $qqZZ$ sample. \label{fig:PDFScale}}
% \end{figure}

The electroweak $EWK~qqZZjj$ samples generated by \textsc{POWHEG-V2} do not have on-the-fly variations to evaluate the PDF uncertainty. Therefore, PDF uncertainty from the \textsc{MadGraph5} sample is taken as the PDF uncertainty for \textsc{POWHEG-V2} $EWK~qqZZjj$ samples.
}

\item{\textbf{Uncertainties on $gg\rightarrow ZZ^{(\ast)}$ NLO Corrections:}
The uncertainty is related to the NLO QCD k-factor applied to the $ggZZ$ sample \cite{ggZZNLOUnc}. The NLO QCD k-factors applied are evaluated differentially as a function of the $m_{4\ell}$. 
}

\item{\textbf{$t\bar{t}V$ $\&$ $VVV$ cross-sections:}
The experimental uncertainties on recently published cross-section measurements of the $ttV$~\cite{ATLAS_ttV} and $WZZ$~\cite{ATLAS_VVV} processes by ATLAS are propagated for the analysis. On the entire $ttV$ process, a flat conservative variation of $15\%$ is applied, taken from the cross-section measurement of $ttZ$. Similarly, for $VVV$ conservative $10\%$ variation taken from the cross-section measurement of $WWZ$ is applied to the whole $VVV$ samples.
}

\end{itemize}

As shown above, the theoretical uncertainties are process specific and are evaluated separately for each MC sample. The theory uncertainties need to be propagated to the unfolded cross-section measurements. For each theory uncertainty, variation-applied particle and detector level yields are built by substituting the varied distribution for the selected process instead of the nominal one. The variation-applied detector level yield is unfolded using the unfolding inputs from nominal SM predictions. The difference between the unfolded result to the variation-applied truth MC yields gives systematic uncertainty for each variation. In general, the theoretical variations significantly affect the predicted particle-level and detector-level yields; however, they have a negligible impact on the shape of the distributions. Since the variation is applied to both detector and particle level yields, the resulting uncertainties from theory systematics on the unfolded cross-sections are small.

\subsection{Experimental Uncertainties}
\label{subsec:ExpUnc}
The experimental uncertainties arise from the measurement of the energy and momentum scales of the reconstructed objects and the uncertainties on object reconstruction, identification, and selection efficiencies. The following category summarizes the sources of experimental uncertainties,

\textbf{Jet Related Uncertainties: }
The analysis requires two jets in the final state. Therefore, jet reconstruction and selection uncertainties are the measurement's most significant source of systematic experimental uncertainties. 

\begin{itemize}
    \item{\textbf{Jet Reconstruction Uncertainty:}
    The jet-related uncertainties associated with reconstruction and different steps of calibration discussed in Section \ref{subsec:ParticleRecon_Jets} are provided by ATLAS-supported tool $\textit{JetUncertainties}\footnote{https://twiki.cern.ch/twiki/bin/view/AtlasProtected/JetUncertainties}$. The tool provides several configurations for jet-related uncertainties adjusted to the various needs of several analyses. The measurement in this thesis uses $\textit{GlobalReduction\_FullJER}$ configuration with a total of $36$ uncertainties, each with upward and downward components, corresponding to $36\times2$ variations, $20\times 2$ variations are related to JES, and $13\times2$ to JER. $6\times 2$ of the $36 \times 2$ variations are related to the $\eta$ inter-calibration procedure, $4\times 2$ to the pile-up energy subtraction procedure, and $8 \times 2$ to the in-situ calibration of jets. Additional $1\times 2$ variations arise separately from the flavor composition, flavor response, a single particle response at high $p_{T}$, and possible punch-through effects.
    }
    \item{\textbf{JVT $\&$ fJVT Uncertainties:} Additional sets of jet uncertainties ($1\times 2$) arising from the efficiencies of jet vertex selections, JVT, and fJVT cut requirements are also considered in the analysis. }
\end{itemize}

An envelope of the $13$ JER uncertainty added in quadrature to an envelope of each of the other sources gives the final impact of jet-related uncertainties. 

\textbf{Lepton Related Uncertainties: } The following categories define the lepton-related uncertainties in the analysis
\begin{itemize}
    \item{\textbf{Electron Efficiencies:}
    The electron efficiency uncertainty consists of uncertainties on the trigger, identification, reconstruction,
and isolation efficiencies of electrons. These uncertainties are provided by an ATLAS-supported
tool $\textit{ElectronEffciencyCorrection} \footnote{https://gitlab.cern.ch/atlas/athena/-/tree/21.2/PhysicsAnalysis/ElectronPhotonID/ElectronEfficiencyCorrection} $. There are a total of $61$ nuisance parameters related to electron efficiencies, each with upward and downward components corresponding to $61\times2$ variations. $34\times2$ out of $61$ is related to uncertainties in identification efficiency, $25\times2$ related to the reconstruction efficiencies, and a single nuisance parameter ($1\times2$) each from the isolation efficiency and trigger efficiency scale factors.    
    }
    \item{\textbf{Muon Efficiencies:} Similar to the electrons, muon efficiency uncertainty consists of variations on the trigger, identification, reconstruction, and isolation efficiencies of muons, which are provided by another ATLAS-supported tool $\textit{MuonEfficiencyCorrections} \footnote{21https://gitlab.cern.ch/atlas/athena/-/tree/21.2/PhysicsAnalysis/MuonID/MuonIDAnalysis/MuonEfficiencyCorrections} $. In total, there are $10\times2$ nuisance parameters, sets of two ($2\times2$) variations corresponding to trigger efficiency scale factors, sets of four ($4\times2$) related to the identification and reconstructed efficiency, two sets of two ($2\times 2$) each corresponding to the isolation efficiency and track-to-vertex association efficiency. 
    }
    \item{\textbf{Electron Scale $\&$ Resolution:} The electron scale and resolution uncertainty is accounted for by two sets of nuisance parameters corresponding $2\times2$ variations. 

    }
    \item{\textbf{Muon Scale $\&$ Resolution:} For muons resolution and scale uncertainties, there are $5$ sets of nuisance parameters, $2\times2$ corresponding to the muon momentum resolution as measured separately by the Inner Detector and the Muon Spectrometer. One set of nuisance parameters ($1\times 2$) corresponds to the uncertainties on the muon momentum scale, and two sets of $2\times 2$ are associated with the uncertainties in Sagitta correction.}
\end{itemize}

\textbf{Other Experimental Uncertainties: }
\begin{itemize}
    \item{\textbf{Pileup Reweighting:} As discussed in Section \ref{subsec:EventWt}, the MC predictions are reweighted to match the pile-up profile of data. A single $1\times2$ nuisance parameter accounts for upward and downward variations in the factors used for pile-up reweighting. }
    \item{\textbf{Luminosity:} As discussed in Section \ref{subsec:Dataset}, the uncertainty in the collected integrated luminosity of $139 fb^{-1}$ is $\pm1.7\%$, which is applied as a flat variation to both particle and detector level yields. } 
\end{itemize}

The experimental uncertainties affect all detector-level MC predictions and the estimate of the fake backgrounds. The experimental uncertainties need to be propagated to the unfolded cross-sections. For each systematic variation, a detector-level signal ($qqZZ+ggZZ+EWK~qqZZjj$) and background ($ttV+VVV$) distribution, a variation applied prediction is built. The variation is also applied to the fake background estimate. The variation-applied background MC and fake backgrounds are subtracted from the variation-applied total MC prediction and then unfolded using the unfolding inputs from the nominal SM prediction. The individual systematic uncertainty corresponds to the difference between the variation-applied and nominal unfolded distributions for each variation.  

\subsection{Unfolding Uncertainties}
\label{subsec:UnfoldingUnc}
The following two uncertainties are intrinsic to the unfolding process itself and are included in the uncertainties for the unfolded differential cross-sections.

\begin{itemize}
    \item{\textbf{Unfolding Bias:} The unfolding bias estimated using the data-driven method discussed in Section \ref{subsec:Bias} is an inherent bias of the unfolding method and the biggest source of the systematic uncertainty for the measurement.
    }

    \item{\textbf{QCD $qqZZ$ Modeling Uncertainty:} There are known differences between different generators driven by differences in parton shower and hadronization. Therefore, the second source of unfolding systematics is required to account for the differences in the unfolding input modeling for the dominant $qqZZ$ process. To avoid double-counting of the unfolding method covered by the data-driven uncertainties, an alternative $qqZZ$ sample predicted by \textsc{MadGraph5} is first reweighted to match the nominal-\textsc{Sherpa} lineshape. The relative difference in reweighted detector-level distribution is unfolded using the inputs from nominal-\textsc{Sherpa} and compared with the reweighted-\textsc{MadGraph5} particle level distribution. The relative difference between these two distributions is taken as modeling systematic uncertainty. Figure \ref{fig:QCDModelUnc} shows the estimation of the modeling uncertainty for $m_{jj}$ in the VBS-Enhanced region. The ratio panel of the right plot shows the QCD modeling uncertainties, which range from $2-4\%$ varying in different bins.
    
    \begin{figure}
        \centering
        \includegraphics[width=.48\linewidth]{figures/Analysis/Systematics/QCDmodel_Dist.pdf}
        \includegraphics[width=.48\linewidth]{figures/Analysis/Systematics/QCDmodel_Unc.pdf}
        \caption{The left plot shows three distributions of $m_{jj}$ in VBS-Enhanced region at detector-level, red corresponding to SM predictions where $qqZZ$ is taken from \textsc{Sherpa}, yellow shows the same but $qqZZ$ is taken from \textsc{MadGraph5} and black shows the reweighted-\textsc{MadGraph5} distribution to match the \textsc{Sherpa} lineshape. The right plot shows reweighted-\textsc{MadGraph5} detector-level (blue) distribution, which is unfolded (black) using unfolding inputs from nominal-\textsc{Sherpa} and compared to the particle-level reweighted-\textsc{MadGraph5} distribution (red). The ratio panel of the right plot showing the ratio between reweighted truth-level \textsc{MadGraph5} and reweighted unfolded-level \textsc{MadGraph5} gives the QCD modeling uncertainties. \label{fig:QCDModelUnc}}
 \end{figure}  

    }
\end{itemize}

\subsection{Background Uncertainties}
\label{subsec:BkgUnc}
There are additional sources of uncertainties from the data-driven estimate of the fake background. The statistical and systematic uncertainties on the fake efficiency discussed in Section \ref{subsubsec:FakeEff}, estimated in the combined control region, are propagated to the final unfolded cross-section yield. First, the variation-applied fake background is calculated and subtracted from the nominal detector-level prediction for each variation. The subtracted altered distribution is then unfolded with nominal unfolding inputs. The difference between the altered-unfolded distribution and the nominal-unfolded distribution gives the impact of the background uncertainties on the unfolded cross-section measurements.

\subsection{Breakdown of the Systematic Uncertainties }
\label{subsec:SysUncBreakdown}
Tables \ref{tab:systematics_mjj_VBS_Suppressed} and \ref{tab:systematics_mjj_VBS_Enhanced} show the impact of systematic uncertainties in VBS-Suppressed and VBS-Enhanced region respectively for each bin of $m_{jj}$. In both regions and most bins, the unfolding bias is the dominant source of systematic uncertainty, followed by the jet systematics. Figure \ref{fig:systematics_mjj} shows the same systematic uncertainties schematically. Figure \ref{fig:jet_systematics_mjj} shows the impact of different categories of the jet systematic uncertainties. In most bins of $m_{jj}$, the dominant jet uncertainties are from the pile-up energy correction step in the jet calibration. The uncertainties from jet eta-dependent calibration and jet energy resolution are also significant. Overall, the jet reconstruction uncertainties have about $8-9\%$ effect on each bin of the unfolded cross-sections.

\begin{table}
\centering
\begin{tabular}{|l || c | c | c | }
\hline 
Bin $m_{jj}$ [GeV] & [300, 410) & [410, 600) & [600, 1780)\\
\hline 
QCD MC modelling & 1.4 & 0.3 & \textbf{6.6 }\\
Jet & \textbf{7.8} & \textbf{6.6} & \textbf{6.3 }\\
Trigger & 0.32 & 0.07 & 0.081 \\
Leptons & 1.8 & 1.2 & 1.2 \\
PRW & 0.39 & 0.062 & 0.21\\
Theory ($qqZZ$) & 2 & 2.4 & 2.1 \\
Theory (EWK $qqZZjj$) & 0.017 & 0.01 & 0.037 \\
Theory ($ggZZ$) & 0.3 & 0.51 & 0.64 \\
MC Bkg. (ttV+VVV) & 1.6 & 1.7 & 1.6 \\
Fake Bkg. (stat + syst) & 3 & 2.3 & 1.8 \\
Luminosity & 1.7 & 1.7 & 1.7 \\
Data-Driven Closure & \textbf{12} & \textbf{6.3} & \textbf{7.1}\\
\hline
Total & 15 & 10 & 12 \\
\hline
\end{tabular}
\caption{Breakdown of the relative systematic uncertainties ($\%$) for each bin of $\mjj$ in the VBS-Suppressed region. \label{tab:systematics_mjj_VBS_Suppressed}}
\end{table}

\begin{table}
\centering
\begin{tabular}{| l || c | c | c | c | c | }
\hline \hline
Bin $m_{jj}$ [GeV] & [300, 400) & [400, 530) & [530, 720) & [720, 1080) & [1080, 3280)\\
\hline
QCD MC modelling & 2.6 & 0.27 & 0.39 & 1.3 & 3.6\\
Jet & \textbf{7.4} & \textbf{7.6} & \textbf{8.9} & \textbf{8.5} & \textbf{8.9}\\
Trigger & 0.061 & 0.078 & 0.083 & 0.053 & 0.049\\
Leptons & 1.1 & 1.1 & 1.1 & 1.1 & 1.1\\
PRW & 0.38 & 0.58 & 0.79 & 0.83 & 0.59\\
Theory ($qqZZ$) & 2.7 & 2.3 & 2.6 & 1.9 & 0.85\\
Theory (EWK $qqZZjj$) & 0.074 & 0.054 & 0.065 & 0.15 & 0.89\\
Theory ($ggZZ$) & 0.48 & 0.3 & 0.34 & 0.36 & 1.1\\
MC Bkg. (ttV+VVV) & 2.7 & 2.6 & 2.4 & 1.8 & 1.2\\
Fake Bkg. (stat + syst) & 2.4 & 2.5 & 2.5 & 1.7 & 1.5\\
Luminosity & 1.7 & 1.7 & 1.7 & 1.7 & 1.7\\
\hline
Total & 9.3 & 9 & 10 & 9.4 & 10\\
\hline
\end{tabular}
\caption{Breakdown of the relative systematic uncertainties ($\%$) for each bin of $\mjj$ in the VBS-Enhanced region. \textcolor{red}{to update including data-driven closure test. } \label{tab:systematics_mjj_VBS_Enhanced}}
\end{table}

\begin{figure}[!htb]
\centering
\includegraphics[width=.49\linewidth]{figures/Analysis/Systematics/systematics_VBS_Suppressed.pdf}
\includegraphics[width=.49\linewidth]{figures/Analysis/Systematics/systematics_VBS_Enhanced.pdf}
\caption{Systematic uncertainties as a function of $\mjj$ in the VBS-Suppressed region (left) and the VBS-Enhanced region (right). \textcolor{red}{update with ATLAS labels and data-driven closure test}}  \label{fig:systematics_mjj}
\end{figure}

\begin{figure}[!htb]
\centering
\includegraphics[width=.49\linewidth]{figures/Analysis/Systematics/jet_systematics_VBS_Suppressed.pdf}
\includegraphics[width=.49\linewidth]{figures/Analysis/Systematics/jet_systematics_VBS_Enhanced.pdf}
\caption{Impact of different jet systematic uncertainties on the unfolded distribution of $\mjj$ in the VBS-Suppressed region (left) and the VBS-Enhanced region (right). \textcolor{red}{update with ATLAS labels and add data-driven closure test}  \label{fig:jet_systematics_mjj}}
\end{figure}

\subsection{Statistical Uncertainties}
\label{subsubsec:StatUnc}
The statistical uncertainty from the reconstructed data needs to be propagated to the estimated unfolded yield. Equation \ref{eqn:BayesianUnfolding} gives the unfolded yield for a target bin i with a single iteration. As the background subtracted detector yields are filled event by event, the reconstruction distribution is uncorrelated. However, as shown by the equation, an unfolded yield in one single bin depends on all detector-level bins due to the resolution effects via bin migration. Therefore, the statistical uncertainty on the unfolded yield is a combination of the uncertainties in detector-level bins and uncertainties on the migration probabilities, which takes the covariance between the detector-level bins into account. The statistical uncertainties at the unfolded level are evaluated by the \textit{RooUnfold} package, which propagates both of these uncertainties.


\clearpage

\part {\LARGE{Results}}
\label{sec:Results}

This chapter presents the main results of this thesis. Section \ref{sec:DetectorLevel_Measurement} presents the detector-level measurements of the eleven kinematic observables introduced in Section \ref{sec:Obs}, whereas Section \ref{sec:DifferentialxS} presents the unfolded differential cross-sections. Section \ref{sec:EFT} reinterprets the unfolded cross-sections shown in Section \ref{sec:DifferentialxS} to constrain parameters of physics beyond the SM affecting the quartic gauge vertices of electroweak $ZZjj$ production. 

\section{Detector Level Measurements }
\label{sec:DetectorLevel_Measurement}

Figure \ref{fig:reco_VBS_Enhanced_a} shows the measured detector-level data and predicted detector-level yields for six kinematic observables:
\begin{itemize}
    \item{ invariant mass of the dijet system $[m_{jj}]$},
    \item{ invariant mass of the two $Z$ bosons $[m_{4\ell}]$},
    \item{ transverse momentum of the two tagging jets $[p_{T,jj}]$},
    \item{ transverse momentum of the two $Z$ bosons $[p_{T,4\ell}]$},
    \item{ transverse momentum of the two $Z$ bosons and two tagging jets $[p_{T,4\ell jj}]$}, and 
    \item{ scalar transverse momentum of the two $Z$ bosons and dijet $[s_{T,4\ell jj}]$}.
\end{itemize}
Similarly, Figure \ref{fig:reco_VBS_Enhanced_b} shows the detector-level data and predictions for remaining kinematic observables:
\begin{itemize}
\item{ cosine of the decay angle of the negative lepton of the leading (sub-leading) pair in the pair's rest frame $[\cos \theta^{*}_{\ell 1 (3) \ell 2 (4)}]$},
\item{ signed difference between the azimuthal angle of two jets $[\Delta \phi _{jj}^{signed}]$},
\item{ rapidity difference between two jets $[\Delta y_{jj}]$}, and 
\item{ centrality of the system $[\zeta]$}.
\end{itemize}
For each of these distributions, the measured data (black dot) is compared with the state-of-the-art SM predictions, where the two QCD signals $qqZZ$ (red) and $ggZZ$ (blue), and the two MC predicted backgrounds, $VVV$ (yellow) and $ttV$ (purple) are \textsc{Sherpa} predictions. The contribution from non-prompt backgrounds (light pink) is estimated using the data-driven method. The $qqZZjj$ electroweak signal is obtained from \textsc{PowhegV2}, and the electroweak production of triboson and two jets ($VZZjj$) is obtained from \textsc{Sherpa}. The vertical-solid error bars on the data points represent the statistical uncertainty in the measured data, and the dashed black band in each bin represents the impact of the total theoretical and experimental uncertainties on the predicted detector-level yields. The impact on the bin yield from the total systematic uncertainties and the statistical precision each ranges from $15$ to $20\%$, depending on the bins and the distributions. 

The lower panels in these distributions show the ratio of data yields to the total SM predicted yields, which are compatible within the total uncertainties. Some discrepancies are observed for some distributions. For instance, the yield is underpredicted in the second and the third bin of $m_{jj}$ and $m_{4l}$, respectively. However, these differences are statistically insignificant. Moreover, a slight but statistically insignificant asymmetry is also observed in the measured distribution of $\Delta \phi _{jj}^{signed}$, which is expected to be symmetric in the SM. 

A simple chi-squared per degree of freedom ($\chi^2/NDF$) is estimated to quantify whether the measured data agree with the SM prediction. As the data yield is event counts, the $\chi^2/NDF$ is computed using the residual difference in each bin between the unweighted data yield and the weighted MC prediction yield. The respective statistical and systematic uncertainties are also considered in the computation of the $\chi^2/NDF$. The reported values of $\chi^2/NDF$ for each distribution in Figures \ref{fig:reco_VBS_Enhanced_a} and \ref{fig:reco_VBS_Enhanced_b} show statistically good agreement between the measured data and SM predictions. Most values of $\chi^2/NDF$ are smaller than one suggesting the systematic uncertainties could be overestimated in these distributions. The experimental uncertainties related to physics object reconstruction are estimated in a different phase space than the analysis's, which could contribute to the overestimation of these systematic uncertainties. 

\begin{figure}[!htbp]
    \centering
    \begin{subfigure}{.49\textwidth}
        \centering
        \includegraphics[width=.98\linewidth]{figures/Results/RecoDist_VBSEnhanced/reco_mjj_SR.pdf}
        \caption{ \footnotesize{$m_{jj}$}: $\chi^2/NDF = 0.43$ }
    \end{subfigure}
    \begin{subfigure}{.49\textwidth}
        \centering
        \includegraphics[width=.98\linewidth]{figures/Results/RecoDist_VBSEnhanced/reco_m4l_SR.pdf}
        \caption{ \footnotesize{$m_{4\ell}$ }: $\chi^2/NDF = 0.49$ }
    \end{subfigure}\\
    \begin{subfigure}{.49\textwidth}
        \centering
        \includegraphics[width=.98\linewidth]{figures/Results/RecoDist_VBSEnhanced/reco_ptjj_SR.pdf}
        \caption{ \footnotesize{$p_{T,jj}$}: $\chi^2/NDF = 0.30$ }
    \end{subfigure}
    \begin{subfigure}{.49\textwidth}
        \centering
        \includegraphics[width=.98\linewidth]{figures/Results/RecoDist_VBSEnhanced/reco_pt4l_SR.pdf}
        \caption{ \footnotesize{$p_{T,4\ell}$ }: $\chi^2/NDF = 0.15$ }
    \end{subfigure}\\
    \begin{subfigure}{.49\textwidth}
        \centering
        \includegraphics[width=.98\linewidth]{figures/Results/RecoDist_VBSEnhanced/reco_ptzzjj_SR.pdf}
        \caption{ \footnotesize{$p_{T,4\ell jj}$}: $\chi^2/NDF = 0.70$ }
    \end{subfigure}
    \begin{subfigure}{.49\textwidth}
        \centering
        \includegraphics[width=.98\linewidth]{figures/Results/RecoDist_VBSEnhanced/reco_stzzjj_SR.pdf}
        \caption{ \footnotesize{$s_{T, 4\ell jj}$ }: $\chi^2/NDF = 0.24$ }
    \end{subfigure}
    \caption{Detector level distributions in the VBS-Enhanced region.}  \label{fig:reco_VBS_Enhanced_a}
\end{figure}

\begin{figure}[!htbp]
    \centering
    \begin{subfigure}{.49\textwidth}
        \centering
        \includegraphics[width=.98\linewidth]{figures/Results/RecoDist_VBSEnhanced/reco_cosThetaStar1_SR.pdf}
        \caption{ \footnotesize{$\cos \theta^{*}_{\ell 1 \ell 2}$}: $\chi^2/NDF = 0.51$ }
    \end{subfigure}
    \begin{subfigure}{.49\textwidth}
        \centering
        \includegraphics[width=.98\linewidth]{figures/Results/RecoDist_VBSEnhanced/reco_cosThetaStar3_SR.pdf}
        \caption{ \footnotesize{$\cos \theta^{*}_{\ell 3 \ell 4}$ }: $\chi^2/NDF = 0.61$ }
    \end{subfigure}\\
    \begin{subfigure}{.49\textwidth}
        \centering
        \includegraphics[width=.98\linewidth]{figures/Results/RecoDist_VBSEnhanced/reco_dphi_SR.pdf}
        \caption{ \footnotesize{$\Delta \phi _{jj}^{signed}$ }: $\chi^2/NDF = 1.69$ }
    \end{subfigure}
    \begin{subfigure}{.49\textwidth}
        \centering
        \includegraphics[width=.98\linewidth]{figures/Results/RecoDist_VBSEnhanced/reco_dy_SR.pdf}
        \caption{ \footnotesize{$\Delta y_{jj}$ }: $\chi^2/NDF = 0.26$ }
    \end{subfigure}\\
    \begin{subfigure}{.49\textwidth}
        \centering
        \includegraphics[width=.98\linewidth]{figures/Results/RecoDist_VBSEnhanced/reco_centrality_SR.pdf}
        \caption{ \footnotesize{$\zeta$ }: $\chi^2/NDF = 0.50$ }
    \end{subfigure}
    \caption{Detector level distributions in the VBS-Enhanced region.}  \label{fig:reco_VBS_Enhanced_b}
\end{figure}
\section{ Unfolded Differential Cross-sections }
\label{sec:DifferentialxS}

The background-subtracted data yield is unfolded using the iterative Bayesian Unfolding discussed in Section \ref{sec:Unfolding}. The unfolded differential cross-sections, the main results of this thesis, are obtained by multiplying the inverse of integrated luminosity to the background-subtracted unfolded yield in each bin. The unfolded differential cross-sections for eleven kinematic observables are shown in figures \ref{fig:unfolded_xs_VBS_Enhanced_a} and \ref{fig:unfolded_xs_VBS_Enhanced_b}. 

Each distribution of unfolded differential cross-sections in black is compared to two different state-of-the-art particle-level SM predictions in red and blue. The red distribution represents the SM particle level differential cross-sections where the QCD $qqZZjj$ contribution is predicted by \textsc{Sherpa} generator.
Similarly, in the blue distribution, the QCD $qqZZjj$ contribution is predicted by \textsc{Madgraph} generator. The light-red and light-blue bands are the fiducial level systematics on the particle-level yield, respectively, by \textsc{Sherpa} and \textsc{Madgraph}. The vertical-solid line on the unfolded data points is the statistical uncertainty, whereas the black band represents the total systematic uncertainties from theoretical, experimental, and unfolding sources. The unfolded cross-sections are limited by statistical precision in all distributions and all bins. For some distributions, one or two events are found in the overflow bin resulting from bin migrations. These events are added to the content of the last bin of the unfolded distribution. 

Generally, for all distributions, the data is well modeled by the MC simulations within $2\sigma$ of the uncertainty band. Two different p-values are determined by comparing unfolded cross-sections to the two predicted cross-sections to quantify the agreement between the experimentally measured unfolded and the SM-predicted cross-sections. The p-value is calculated based on a technique discussed in Ref\cite{pValueStat} by taking the residual and uncertainties of two weighted histograms. For all kinematic observables, the reported p-values obtained by comparing to either generator are more significant than $0.05$. Therefore, in the analyzed LHC Run-2 dataset, for the $ZZ (\rightarrow 4) \ell jj$ process, all differential cross-sections in the VBS-Enhanced region are concluded to agree with the SM predictions. 

\begin{figure}[!htb]
    \centering
    \begin{subfigure}{.48\textwidth}
        \centering
        \includegraphics[width=.98\linewidth]{figures/Results/CrossSection_VBSEnhanced/xs_mjj_SR.pdf}
        \caption{ \footnotesize{$m_{jj}$: p-value(Sherpa=0.95 $\&$ MG=0.91)}}
    \end{subfigure}
    \begin{subfigure}{.48\textwidth}
        \centering
        \includegraphics[width=.98\linewidth]{figures/Results/CrossSection_VBSEnhanced/xs_m4l_SR.pdf}
        \caption{ \footnotesize{$m_{4\ell}$: p-value(Sherpa=0.84 $\&$ MG=0.45)} }
    \end{subfigure}\\
    \begin{subfigure}{.48\textwidth}
        \centering
        \includegraphics[width=.98\linewidth]{figures/Results/CrossSection_VBSEnhanced/xs_ptjj_SR.pdf}
        \caption{ \footnotesize{$p_{T,jj}$: p-value(Sherpa=0.97 $\&$ MG=0.85)}}
    \end{subfigure}
    \begin{subfigure}{.48\textwidth}
        \centering
        \includegraphics[width=.98\linewidth]{figures/Results/CrossSection_VBSEnhanced/xs_pt4l_SR.pdf}
        \caption{ \footnotesize{$p_{T,4\ell}$: p-value(Sherpa=0.98 $\&$ MG=0.94)}}
    \end{subfigure}\\
    \begin{subfigure}{.48\textwidth}
        \centering
        \includegraphics[width=.98\linewidth]{figures/Results/CrossSection_VBSEnhanced/xs_ptzzjj_SR.pdf}
        \caption{ \footnotesize{$p_{T,4\ell jj}$: p-value(Sherpa=0.92 $\&$ MG=0.80)}}
    \end{subfigure}
    \begin{subfigure}{.48\textwidth}
        \centering
        \includegraphics[width=.98\linewidth]{figures/Results/CrossSection_VBSEnhanced/xs_stzzjj_SR.pdf}
        \caption{ \footnotesize{$s_{T, 4\ell jj}$: p-value(Sherpa=0.94 $\&$ MG=0.83)}}
    \end{subfigure}
    \caption{Unfolded differential cross-sections in the VBS-Enhanced region.} \label{fig:unfolded_xs_VBS_Enhanced_a}
\end{figure}

\begin{figure}[!htb]
    \centering
    \begin{subfigure}{.49\textwidth}
        \centering
        \includegraphics[width=.98\linewidth]{figures/Results/CrossSection_VBSEnhanced/xs_cosThetaStar1_SR.pdf}
        \caption{ \footnotesize{$\cos \theta^{*}_{\ell 1 \ell 2}$: p-value(Sherpa=0.80 $\&$ MG=0.55)}}
    \end{subfigure}
    \begin{subfigure}{.49\textwidth}
        \centering
        \includegraphics[width=.98\linewidth]{figures/Results/CrossSection_VBSEnhanced/xs_cosThetaStar3_SR.pdf}
        \caption{ \footnotesize{$\cos \theta^{*}_{\ell 3 \ell 4}$: p-value(Sherpa=0.71 $\&$ MG=0.30)} }
    \end{subfigure}\\
    \begin{subfigure}{.49\textwidth}
        \centering
        \includegraphics[width=.98\linewidth]{figures/Results/CrossSection_VBSEnhanced/xs_dphi_SR.pdf}
        \caption{ \footnotesize{$\Delta \phi _{jj}^{signed}$: p-value(Sherpa=0.28 $\&$ MG=0.18)} }
    \end{subfigure}
    \begin{subfigure}{.49\textwidth}
        \centering
        \includegraphics[width=.98\linewidth]{figures/Results/CrossSection_VBSEnhanced/xs_dy_SR.pdf}
        \caption{ \footnotesize{$\Delta y_{jj}$: p-value(Sherpa=0.96 $\&$ MG=0.93)} }
    \end{subfigure}\\
    \begin{subfigure}{.49\textwidth}
        \centering
        \includegraphics[width=.98\linewidth]{figures/Results/CrossSection_VBSEnhanced/xs_centrality_SR.pdf}
        \caption{ \footnotesize{$\zeta$: p-value(Sherpa=0.87 $\&$ MG=0.73)} }
    \end{subfigure}
    \caption{Unfolded differential cross-sections in the VBS-Enhanced region.}  \label{fig:unfolded_xs_VBS_Enhanced_b}
\end{figure}
\section{Effective Field Theory ReInterpretation}
\label{sec:EFT}

\clearpage

\part {\LARGE{Conclusion}}
\label{sec:Conclusion}


\clearpage

% \label{sec:Outlook}

\section{Run-3}
\label{sec:Run3}
\input{sections/Run3}

\section{High Luminosity LHC}
\label{sec:HLLHC}
\input{sections/HLLHC}

% 	\include{sections/Run3}
% 	\include{sections/HLLHC}
% \clearpage

\addcontentsline{toc}{part}{References}
{\singlespacing
\renewcommand{\refname}{References}
\bibliography{biblio}
\bibliographystyle{unsrt}
\clearpage
}
\begin{appendices}
    \part*{Appendices}
	\section{Personal Contribution}
\label{Appendix:Contribution}

\subsection{Contribution to $ZZ(\rightarrow 4\ell)jj$ Measurement} 
\label{subsec:ZZjjContr}

The measurement presented in this thesis is only possible from the effort of the entire analysis team. I am one of the two primary analyzers who have contributed to several measurement areas from its initial formation to the current stage of finalization and publication. I contributed significantly to defining the analysis phase space discussed in Chapter $IV$. The phase space defined in the preceding ATLAS analysis \cite{ATLASZZjj}, which observed the electroweak $ZZjj$ process was not optimal for differential measurement. Therefore, I worked on loosening the kinematic selections to increase the acceptance, defining the isolation working point to maintain optimal signal-selection and background-rejection probabilities, and establishing novel pair sorting strategy to reduce the bin migration in unfolding. Moreover, I was responsible for maintaining the main analysis framework, which implements the latest recommendations from combined performance groups for physics object reconstruction discussed in Chapter $III$, and applies most of the kinematic cuts for object and event selection discussed in Chapter $IV$. This framework applies all the scale factors and event weights in MC events, and selects a few hundred relevant physics events that passes all kinematic selections for both data and MC. Additionally, I estimated the fake backgrounds of the analysis using the background estimation technique discussed in Section \ref{sec:Bkg} and assisted in selecting relevant systematics discussed in Section \ref{sec:Uncertainties} of Chapter $V$. I also validated the novel next-to-leading-order \textsc{Powheg} $qqZZjj$ MC sample used as the primary sample for the electroweak production of $ZZ(\rightarrow 4\ell)jj$. 

\subsection{Contribution to the ATLAS Experiment}
\label{subsec:ATLASContr}
I have been a member of the ATLAS collaboration since $2017$ and have contributed to the three critical areas of the experiment; detector development, detector performance, and physics analyses.

\textbf{Detector Developement:}
I spent my first year as a graduate student working at Brookhaven National Lab, where I contributed to the prototype development of the all-silicon inner tracker for the high luminosity LHC. During this period, I assembled the first three prototype staves, the detector units of the ITk strip barrel region detector, using a semi-automated loading setup consisting of an Aeoretch robotic arm, cameras, and an alignment system. I developed a $3D$-printed tooling pins used in the alignment of loading the sensors on staves. I led comprehensive thermal and mechanical tests of the first prototype, the thermo-mechanical stave using IR imaging and laser metrology, respectively. These tests validated the cooling system designs of the ITk stave core structure and the stability of the mechanical design, giving the green light to the production of $200$ out of $400$ strip-staves needed to build the barrel region of the inner tracker. During this year, I learned the fundamentals of particle physics detectors, their development, and their operations.  

\textbf{Physics Analysis:}
Apart from the $ZZ(\rightarrow 4\ell)jj$ analysis presented in this thesis, I have worked on two other ATLAS analyses involving four leptons in the final state, analyzing the complete Run-2 datasets. After my Ph.D. candidacy and ATLAS qualification, I joined the inclusive four-lepton measurement analysis team, whose goal was to inclusively measure the unfolded cross-sections of the Standard Model four-lepton process. In a year as a part of the team, I worked on finding the suitable lepton isolation working points, studying the impact of including electrons and muons originating from tau leptons in the unfolded differential cross-sections, and the most precise measurement to date of the branching ratio of $ Z \to 4\ell$ with full Run-2 dataset.

Since early 2021, I have been a crucial part of the on-shell $ZZ$ CP and polarization analysis team, where the main goal is to extract the fraction of two $Z$ bosons simultaneously longitudinally polarized and search for additional CP violation. Like the $ZZ(\rightarrow 4\ell)jj$ analysis, I have contributed to background estimation, phase space optimization, and event selection. Additionally, I have contributed to deriving the unfolded cross-sections with all relevant systematic and statistical uncertainties used in CP violation searches using an effective field theory approach. 

\textbf{Detector Performance: }
The training of a particle physicist is incomplete without understanding the detector's performance. Therefore, in 2021 I joined the Tracking Combined Performance group of ATLAS and have contributed to the validation of the performance of the Run-3 tracking reconstruction software in both early Run-3 data and different types of MC simulation. 

Similar to $ZZ(\rightarrow 4\ell)jj$ measurement, most ATLAS physics analyses use the vertex with the highest value of the sum-squared of track's transverse momenta $(\sum_{tracks}{p_{T}^2})$ as the hard scattering vertex of the measurement. However, in processes with softer leptons and invisible tracks (including photons), $\sum_{tracks}{p_{T}^2}$ is inadequate to identify the hard scatter vertex. Therefore, I am currently working on developing an alternative algorithm that is suitable for a variety of different physics processes. 

The experience I have gained in different areas of particle physics has significantly shaped my discussion of the measurement presented in this document.  

\section{Additional Study on Unfolding Bias}
\label{Appendix:Unfolding_bias}
As discussed in Section \ref{subsec:Bias}, the bias from the unfolding process is the largest source of the systematic uncertainty for the measurement. Additional studies were conducted to understand the underlying source of this bias. Figure \ref{fig:UnfoldingBias_mjj_VBSEnhanced} shows the unfolding bias and statistical uncertainty on the unfolded yield as a function of the increasing number of unfolding iterations for each bin of $m_{jj}$ in the VBS-Enhanced region. The bias is evaluated using the MC-toy-based method introduced in Section \ref{subsec:Bias}. The total uncertainty shown in the black distribution is always the smallest for a single iteration, further assuring the choice of iteration is optimal. As expected, the bias decreases and the statistical uncertainty increases with the increasing number of unfolding iterations.

\begin{figure}[htb]
    \centering
    \begin{subfigure}{.48\textwidth}
        \centering
        \includegraphics[width=.9\linewidth]{figures/Analysis/Unfolding/unfoldingbias/unfolding_bias_stat_unc_mjj_VBSEnh_bin1.pdf}
        \caption{ bin 1}
    \end{subfigure}
    \begin{subfigure}{.48\textwidth}
        \centering
        \includegraphics[width=.9\linewidth]{figures/Analysis/Unfolding/unfoldingbias/unfolding_bias_stat_unc_mjj_VBSEnh_bin2.pdf}
        \caption{bin 2 }
    \end{subfigure}\\
    \begin{subfigure}{.48\textwidth}
        \centering
        \includegraphics[width=.9\linewidth]{figures/Analysis/Unfolding/unfoldingbias/unfolding_bias_stat_unc_mjj_VBSEnh_bin3.pdf}
        \caption{ bin 3 }
    \end{subfigure}
    \begin{subfigure}{.48\textwidth}
        \centering
        \includegraphics[width=.9\linewidth]{figures/Analysis/Unfolding/unfoldingbias/unfolding_bias_stat_unc_mjj_VBSEnh_bin4.pdf}
        \caption{bin 4 }
    \end{subfigure}
    \begin{subfigure}{.48\textwidth}
        \centering
        \includegraphics[width=.9\linewidth]{figures/Analysis/Unfolding/unfoldingbias/unfolding_bias_stat_unc_mjj_VBSEnh_bin5.pdf}
        \caption{bin 5 }
    \end{subfigure}
    \caption{ MC-toy-based unfolding bias and statistical uncertainty as a function of several numbers of iterations in each bin of $m_{jj}$ distribution in the VBS-Enhanced region. \label{fig:UnfoldingBias_mjj_VBSEnhanced}}
\end{figure}

The unfolding bias is expected to converge to a value of zero with a higher number of iterations. However, as observed in figure \ref{fig:UnfoldingBias_mjj_VBSEnhanced}, the rate of convergence of unfolding bias is lower, suggesting that the fiducial fakes present in the detector level distributions are not fully corrected by the unfolding method. The fiducial fraction, as shown by figure \ref{fig:UnfoldingInputs}, is usually between $60-80\%$ in this measurement. To confirm that a high fraction of fiducial fakes causes the unfolding bias, these are subtracted manually from the MC predictions of nominal and toy distributions. The MC-toy-based unfolding bias estimate is repeated and figure \ref{fig:UnfoldingBias_mjj_VBSEnhanced_noFakes} shows the resulting unfolding bias in each bin of $m_{jj}$ in the VBS-Enhanced region. Compared to the nominal unfolding bias shown in figure \ref{fig:UnfoldingBias_mjj_VBSEnhanced}, figure \ref{fig:UnfoldingBias_mjj_VBSEnhanced_noFakes} has a smaller bias in each bin. Moreover, the bias converges to zero at a higher rate.

\begin{figure}[htb]
    \centering
    \begin{subfigure}{.48\textwidth}
        \centering
        \includegraphics[width=.9\linewidth]{figures/Analysis/Unfolding/unfoldingbias/unfolding_bias_mjj_noFakes_VBSEnh_bin1.pdf}
        \caption{ bin 1}
    \end{subfigure}
    \begin{subfigure}{.48\textwidth}
        \centering
        \includegraphics[width=.9\linewidth]{figures/Analysis/Unfolding/unfoldingbias/unfolding_bias_mjj_noFakes_VBSEnh_bin2.pdf}
        \caption{bin 2 }
    \end{subfigure}\\
    \begin{subfigure}{.48\textwidth}
        \centering
        \includegraphics[width=.9\linewidth]{figures/Analysis/Unfolding/unfoldingbias/unfolding_bias_mjj_noFakes_VBSEnh_bin3.pdf}
        \caption{ bin 3 }
    \end{subfigure}
    \begin{subfigure}{.48\textwidth}
        \centering
        \includegraphics[width=.9\linewidth]{figures/Analysis/Unfolding/unfoldingbias/unfolding_bias_mjj_noFakes_VBSEnh_bin4.pdf}
        \caption{bin 4 }
    \end{subfigure}
    \begin{subfigure}{.48\textwidth}
        \centering
        \includegraphics[width=.9\linewidth]{figures/Analysis/Unfolding/unfoldingbias/unfolding_bias_mjj_noFakes_VBSEnh_bin5.pdf}
        \caption{bin 5 }
    \end{subfigure}
    \caption{ MC-toy-based unfolding bias in each bin of $m_{jj}$ in the VBS-Enhanced region using Gaussian toys after subtracting the contribution of the fiducial fake events from both nominal and toy MC predictions.\label{fig:UnfoldingBias_mjj_VBSEnhanced_noFakes}}
\end{figure}

The differential measurements of the $ZZ(\rightarrow 4\ell)jj$ process are statistically limited, so it is impossible to directly subtract the fiducial fakes from data without increasing both statistical and systematic uncertainties from the fiducial fake estimate. However, it is imperative to understand the origin and topology of the fiducial fake events to reduce their impact without degrading the unfolding procedure's performance. Figure \ref{fig:fake_fraction_jets} shows the fake fraction, the fraction of detector-level events passing detector-level selection but failing the particle-level selection as a function of $p_{T}$ (left) and $\eta$ (right) of the leading and the sub-leading jets. More significant fractions of fakes are observed in low-$p_{T}$ and high $\eta$ region, which is likely related to the worse resolution of jet reconstruction in low-$p_{T}$ and smaller efficiency of fJVT tagging in forward regions. The large fraction of fiducial fakes is understood to originate either from migrations outside the fiducial volume due to jet resolution effects or from wrongfully selecting events with pile-up jets. More stringer kinematic selections were applied to the leading and sub-leading jets in an attempt to reduce the bias, but this resulted in the degradation of unfolding performance due to low statistics. Therefore, the nominal bias shown in Section \ref{subsec:UnfoldingUnc} was deemed optimal. 

\begin{figure}[htb]
    \centering
    \begin{subfigure}{.48\textwidth}
        \centering
        \includegraphics[width=.9\linewidth]{figures/Analysis/Unfolding/FakeFraction_pt_jets.pdf}
        \caption{ $p_{T}$}
    \end{subfigure}
    \begin{subfigure}{.48\textwidth}
        \centering
        \includegraphics[width=.9\linewidth]{figures/Analysis/Unfolding/FakeFraction_eta_jets.pdf}
        \caption{$\eta$}
    \end{subfigure}
    \caption{ Fraction of fiducial fake events as a function of $p_{T}~\&~\eta$ of the leading and the sub-leading jets in the VBS-Enhanced region.\label{fig:fake_fraction_jets}}
\end{figure}

\section{VBS Suppressed Region}
\label{Appendix:VBSSupRegion}

\subsection{Systematics}
\label{appendix:VBSSupSys}

\begin{table}
    \centering
    \begin{tabular}{|l || c | c | c | }
    \hline 
    Bin $m_{jj}$ [GeV] & [300, 410) & [410, 600) & [600, 1780)\\
    \hline 
    QCD MC modelling & 2 & 0.16 & \textbf{7.1}\\
    Jet & \textbf{7.3} & \textbf{6.8} & \textbf{6.2 }\\
    Trigger & 0.03 & 0.06 & 0.08 \\
    Leptons & 1.4 & 1.5 & 1.6 \\
    PRW & 0.39 & 0.06 & 0.16\\
    Theory ($qqZZ$) & 1.9 & 2.3 & 2.1 \\
    Theory (EWK $qqZZjj$) & 0.02 & 0.02 & 0.04 \\
    Theory ($ggZZ$) & 0.13 & 0.22 & 0.28 \\
    MC Bkg. (ttV+VVV) & 1.6 & 1.7 & 1.6 \\
    Fake Bkg. (stat + syst) & 3.4 & 2.9 & 2.6 \\
    Luminosity & 1.5 & 1.4 & 1.4 \\
    Unfolding Bias & \textbf{12.1} & \textbf{11.0} & \textbf{6.1}\\
    \hline
    Total & 14.7 & 13.6 & 12.0 \\
    \hline
    \end{tabular}
    \caption{Breakdown of the relative systematic uncertainties ($\%$) for each bin of $\mjj$ in the VBS-Suppressed region. \label{tab:systematics_mjj_VBS_Suppressed}}
\end{table}

\begin{figure}[!htb]
    \centering
    \begin{subfigure}{.48\textwidth}
        \centering
        \includegraphics[width=.9\linewidth]{figures/Analysis/Systematics/systematics_VBS_Suppressed.pdf}
        \caption{ Statistical and systematic uncertainties affecting the differential cross-sections measurements. \label{fig:sys_mjj_VBS_Suppressed_total}}
    \end{subfigure}
    \begin{subfigure}{.48\textwidth}
        \centering
        \includegraphics[width=.9\linewidth]{figures/Analysis/Systematics/jet_systematics_VBS_Suppressed.pdf}
        \caption{Breakdown of the total jet related uncertainties. \label{fig:sys_mjj_VBS_Suppressed_jet} }
    \end{subfigure}
    \caption{Uncertainties as a function of $\mjj$ in the VBS-Suppressed region.}  \label{fig:sys_mjj_VBS_Suppressed}
\end{figure}

\subsection{Detector Level Measurements}
\label{appendix:VBSSupReco}

\begin{figure}[!htb]
    \centering
    \begin{subfigure}{.49\textwidth}
        \centering
        \includegraphics[width=.98\linewidth]{figures/Results/RecoDist_VBSSuppressed/reco_mjj_CR.pdf}
        \caption{ \footnotesize{$m_{jj}$}: $\chi^2/NDF = 1.78$ }
    \end{subfigure}
    \begin{subfigure}{.49\textwidth}
        \centering
        \includegraphics[width=.98\linewidth]{figures/Results/RecoDist_VBSSuppressed/reco_m4l_CR.pdf}
        \caption{ \footnotesize{$m_{4\ell}$ }: $\chi^2/NDF = 0.88$ }
    \end{subfigure}\\
    \begin{subfigure}{.49\textwidth}
        \centering
        \includegraphics[width=.98\linewidth]{figures/Results/RecoDist_VBSSuppressed/reco_ptjj_CR.pdf}
        \caption{ \footnotesize{$p_{T,jj}$}: $\chi^2/NDF = 2.0$ }
    \end{subfigure}
    \begin{subfigure}{.49\textwidth}
        \centering
        \includegraphics[width=.98\linewidth]{figures/Results/RecoDist_VBSSuppressed/reco_pt4l_CR.pdf}
        \caption{ \footnotesize{$p_{T,4\ell}$ }: $\chi^2/NDF = 0.20$ }
    \end{subfigure}\\
    \begin{subfigure}{.49\textwidth}
        \centering
        \includegraphics[width=.98\linewidth]{figures/Results/RecoDist_VBSSuppressed/reco_ptzzjj_CR.pdf}
        \caption{ \footnotesize{$p_{T,4\ell jj}$}: $\chi^2/NDF = 0.15$ }
    \end{subfigure}
    \begin{subfigure}{.49\textwidth}
        \centering
        \includegraphics[width=.98\linewidth]{figures/Results/RecoDist_VBSSuppressed/reco_stzzjj_CR.pdf}
        \caption{ \footnotesize{$s_{T, 4\ell jj}$ }: $\chi^2/NDF = 0.58$ }
    \end{subfigure}
    \caption{Detector level distributions in the VBS-Suppressed region.}  \label{fig:reco_VBS_Suppressed_a}
\end{figure}

\begin{figure}[!htb]
    \centering
    \begin{subfigure}{.49\textwidth}
        \centering
        \includegraphics[width=.98\linewidth]{figures/Results/RecoDist_VBSSuppressed/reco_cosThetaStar1_CR.pdf}
        \caption{ \footnotesize{$\cos \theta^{*}_{\ell 1 \ell 2}$}: $\chi^2/NDF = 0.01$ }
    \end{subfigure}
    \begin{subfigure}{.49\textwidth}
        \centering
        \includegraphics[width=.98\linewidth]{figures/Results/RecoDist_VBSSuppressed/reco_cosThetaStar3_CR.pdf}
        \caption{ \footnotesize{$\cos \theta^{*}_{\ell 3 \ell 4}$ }: $\chi^2/NDF = 0.60$ }
    \end{subfigure}\\
    \begin{subfigure}{.49\textwidth}
        \centering
        \includegraphics[width=.98\linewidth]{figures/Results/RecoDist_VBSSuppressed/reco_dphi_CR.pdf}
        \caption{ \footnotesize{$\Delta \phi _{jj}^{signed}$ }: $\chi^2/NDF = 1.26$ }
    \end{subfigure}
    \begin{subfigure}{.49\textwidth}
        \centering
        \includegraphics[width=.98\linewidth]{figures/Results/RecoDist_VBSSuppressed/reco_dy_CR.pdf}
        \caption{ \footnotesize{$\Delta y_{jj}$ }: $\chi^2/NDF = 0.55$ }
    \end{subfigure}\\
    \begin{subfigure}{.49\textwidth}
        \centering
        \includegraphics[width=.98\linewidth]{figures/Results/RecoDist_VBSSuppressed/reco_centrality_CR.pdf}
        \caption{ \footnotesize{$\zeta$ }: $\chi^2/NDF = 0.60$ }
    \end{subfigure}
    \caption{Detector level distributions in the VBS-Suppressed region.}  \label{fig:reco_VBS_Suppressed_b}
\end{figure}

\subsection{Unfolded Cross-sections}
\label{appendix:VBSSupUnfolded}

\begin{figure}[!htb]
    \centering
    \begin{subfigure}{.49\textwidth}
        \centering
        \includegraphics[width=.98\linewidth]{figures/Results/CrossSection_VBSSuppressed/xs_mjj_CR.pdf}
        \caption{ \footnotesize{$m_{jj}$}: p-value(Sherpa=0.15 $\&$ MG=0.08)}
    \end{subfigure}
    \begin{subfigure}{.49\textwidth}
        \centering
        \includegraphics[width=.98\linewidth]{figures/Results/CrossSection_VBSSuppressed/xs_m4l_CR.pdf}
        \caption{ \footnotesize{$m_{4\ell}$ }: p-value(Sherpa=0.40 $\&$ MG=0.15)}
    \end{subfigure}\\
    \begin{subfigure}{.49\textwidth}
        \centering
        \includegraphics[width=.98\linewidth]{figures/Results/CrossSection_VBSSuppressed/xs_ptjj_CR.pdf}
        \caption{ \footnotesize{$p_{T,jj}$}: p-value(Sherpa=0.24 $\&$ MG=0.02)}
    \end{subfigure}
    \begin{subfigure}{.49\textwidth}
        \centering
        \includegraphics[width=.98\linewidth]{figures/Results/CrossSection_VBSSuppressed/xs_pt4l_CR.pdf}
        \caption{ \footnotesize{$p_{T,4\ell}$ }: p-value(Sherpa=0.82 $\&$ MG=0.40)}
    \end{subfigure}\\
    \begin{subfigure}{.49\textwidth}
        \centering
        \includegraphics[width=.98\linewidth]{figures/Results/CrossSection_VBSSuppressed/xs_ptzzjj_CR.pdf}
        \caption{ \footnotesize{$p_{T,4\ell jj}$}: p-value(Sherpa=0.96 $\&$ MG=0.08)}
    \end{subfigure}
    \begin{subfigure}{.49\textwidth}
        \centering
        \includegraphics[width=.98\linewidth]{figures/Results/CrossSection_VBSSuppressed/xs_stzzjj_CR.pdf}
        \caption{ \footnotesize{$s_{T, 4\ell jj}$ }: p-value(Sherpa=0.52 $\&$ MG=0.11)}
    \end{subfigure}
    \caption{Unfolded differential cross-sections in the VBS-Suppressed region.}  \label{fig:unfolded_xs_VBS_Suppressed_a}
\end{figure}

\begin{figure}[!htb]
    \centering
    \begin{subfigure}{.49\textwidth}
        \centering
        \includegraphics[width=.98\linewidth]{figures/Results/CrossSection_VBSSuppressed/xs_cosThetaStar1_CR.pdf}
        \caption{ \footnotesize{$\cos \theta^{*}_{\ell 1 \ell 2}$}: p-value(Sherpa=0.90 $\&$ MG=0.81)}
    \end{subfigure}
    \begin{subfigure}{.49\textwidth}
        \centering
        \includegraphics[width=.98\linewidth]{figures/Results/CrossSection_VBSSuppressed/xs_cosThetaStar3_CR.pdf}
        \caption{ \footnotesize{$\cos \theta^{*}_{\ell 3 \ell 4}$ }: p-value(Sherpa=0.43 $\&$ MG=0.41)}
    \end{subfigure}\\
    \begin{subfigure}{.49\textwidth}
        \centering
        \includegraphics[width=.98\linewidth]{figures/Results/CrossSection_VBSSuppressed/xs_dphi_CR.pdf}
        \caption{ \footnotesize{$\Delta \phi _{jj}^{signed}$ }: p-value(Sherpa=0.26 $\&$ MG=0.02)}
    \end{subfigure}
    \begin{subfigure}{.49\textwidth}
        \centering
        \includegraphics[width=.98\linewidth]{figures/Results/CrossSection_VBSSuppressed/xs_dy_CR.pdf}
        \caption{ \footnotesize{$\Delta y_{jj}$ }: p-value(Sherpa=0.57 $\&$ MG=0.32)}
    \end{subfigure}\\
    \begin{subfigure}{.49\textwidth}
        \centering
        \includegraphics[width=.98\linewidth]{figures/Results/CrossSection_VBSSuppressed/xs_centrality_CR.pdf}
        \caption{ \footnotesize{$\zeta$ }: p-value(Sherpa=0.48 $\&$ MG=0.31)}
    \end{subfigure}
    \caption{Unfolded differential cross-sections in the VBS-Suppressed region.}  \label{fig:unfolded_xs_VBS_Suppressed_b}
\end{figure}
\end{appendices}

\end{document}




