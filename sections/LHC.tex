\section{The Large Hadron Collider}
\label{sec:LHC}

The LHC was designed to collide hadrons, either protons or heavy ions, at instantaneous luminosity up to $10^{34}cm^{-2}s^{-1}$ and center-of-mass energies ($\sqrt{s}$) up to $14$ TeV \cite{LHCMachine}. The LHC, the most powerful particle accelerator, is the successor of the Large Electron Positron Collider (LEP), which collided electrons and positrons at $\sqrt{s}$ up to $250$ GeV \cite{LEP}. The LHC reuses the same tunnel system as LEP with a circumference of $27$ km built $50$ to $175$ meters underground at the French-Swiss border outside Geneva, Switzerland. 

The first run of the LHC, Run-1, started in $2010$ at $\sqrt{s} = 7$ TeV, which was later increased to $8$ TeV. Run-1 lasted until $2013$, after which LHC was shut down for two years of planned upgrades to enhance the accelerator's and the detector's performances. LHC resumed its operation for Run-2 from $2015$ to $2018$ at the $\sqrt{s} = 13$ TeV. The data collected during the Run-2 period is analyzed for this thesis. 

Accelerating the proton to the desired center-of-mass energy is a multi-step process shown schematically in Figure \ref{fig:ProtonAcc}. First, the protons are created from hydrogen gas by removing the electrons through ionization with an intense electric field. A proton beam is then formed in the LINAC-2 linear accelerator and injected into the circular PS Booster, which accelerates the $50$ MeV proton to energies of $1.4$ GeV \cite{LHCGuide}. The beams are then injected into the Proton Synchrotron (PS), accelerating them to energies of $25$ GeV \cite{LHCGuide}. The proton beams are then injected into the Super Proton Synchrotron (SPS) to further accelerate at $450$ GeV energies and injected into the main LHC rings. The two opposite proton beams reach the desired final energy of $6.5$ TeV using radio frequency (RF) acceleration cavities \cite{LHCGuide}. The accelerated proton beams are maintained for several hours of data-taking.

\begin{figure}[!htbp]
    \centering
    \includegraphics[width=1.2\linewidth]{figures/LHC/ProtonAccelerator.jpeg}
    \caption{ A detailed layout of multiple-steps that goes into proton acceleration before entering the main LHC ring \cite{ProtonAcclerator}.\label{fig:ProtonAcc}}
\end{figure}

The final RF accelerated proton beams are in evenly-spaced discrete bunches, each consisting of about $10^{11}$ protons. The bunches are separated by $25$ ns spacing \cite{LHCGuide}. The proton beams are guided in the tunnel by a magnetic field using superconducting dipole and quadruplet magnets. The main LHC ring comprises $1232$ dipole magnets that provide a strong magnetic field of $8$ T to bend the beams and about $392$ quadruplet magnets to focus the beams in the transverse plane \cite{LHCGuide}. The superconducting magnets are cooled down to $1.9$ K, which requires two vacuum systems to hold the cryomagnets and the helium distribution lines. To avoid unnecessary interactions, the beams are accelerated and maintained in an ultra-high vacuum of $10^{-13}$ atm \cite{LHCGuide}. 

The two opposite beam lines meet in four interaction points, creating proton-proton collisions. The collisions are recorded by the LHC's four main detectors: ATLAS, CMS, LHCb, and ALICE. ALICE is designed to investigate the heavy-ion collisions and the quark-gluon plasma, whereas the LHCb experiment is designed to study flavor physics \cite{LHCGuide}.