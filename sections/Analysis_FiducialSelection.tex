\section{Fiducial Phase Space}
\label{sec:FidSel}

The ATLAS detector has limited acceptance in the physical phase space, and the selected objects and events are reconstructed within this acceptance phase space. Thus, a fiducial phase space imitating the detector acceptance is defined using physics objects at the particle level to measure the unfolded differential cross-sections. The particle-level signal events are essential to derive the inputs needed to correct the detector effects. Moreover, the measured unfolded cross-sections are compared to the SM-predicted particle-level cross-sections evaluated from these events. Thus, it is essential to carefully select the particle-level signal events in the defined fiducial phase space. This section summarizes the kinematic requirements defining the fiducial phase space of the analysis and the selection of the particle-level signal events. 

The kinematic selections for the fiducial phase space are close to the detector-level object and event selections. The kinematic requirements applied are motivated by the nature of the electroweak production of $pp\rightarrow ZZ^* ( \rightarrow 4\ell) jj$ $[\ell = e,~\mu]$, where the two SF-OC lepton pairs from two $Z$ bosons are produced centrally with respect to the highly energetic dijet. Moreover, the fiducial phase space contains no leptons from tau decays. Both fiducial-level electrons and muons are required to be dressed. The dressing procedure accounts for the energy losses of leptons through photon emissions via bremsstrahlung. Dressed leptons are constructed by adding the four-momenta of nearby photons within the lepton's small $\Delta R < 0.1$ cone. Several kinematic cuts summarized in Table \ref{tab:FidObjectCut} are applied individually to the muons, electrons, and jets to ensure the selected particle-level objects fall within the detector's acceptance before defining the events. Motivated by the discussion of physics object reconstruction in Section \ref{sec:ParticleReconstruction}, each electron is required to have $p_{T} > 7$ GeV and $|\eta| < 2.47$, whereas the muons satisfy $p_{T} > 5$ GeV and $|\eta| < 2.7$. 

\begin{table}[!htbp]
    \centering
    \caption{Details of the kinematic pre-selection applied to the particle-level electrons, muons, and jets. The required kinematic cuts are applied to the dressed leptons.
    \label{tab:FidObjectCut}}
    \begin{tabular}{|| l || c | c | c ||}
        \hline
        Selections      & Electrons             &       Muons        &          Jets            \\
        \hline\hline
        $\Pt~$          & $> 7$ GeV             &       $ >5$ GeV    &      $>30$GeV        \\
        \hline 
        $|\eta|$            &  $< 2.47  $           &       $ < 2.7 $        &      $ < 4.5$            \\
        \hline
    \end{tabular}
\end{table}  

Event quadruplets are formed from the events with at least four leptons by requiring two SF-OC lepton pairs, with leading and sub-leading lepton $p_{T}>20$ GeV and angular separation between any two leptons to satisfy $\Delta R > 0.05$. The invariant mass of any SF-OC lepton pair is required to satisfy $m_{\ell \ell } > 5$ GeV. These particle-level selections have similar motivation to those defined for the detector-level requirements in Section \ref{sec:EventSel}. In any event with more than two SF-OC lepton pairs, the quadruplet is formed by choosing the two pairs that minimize the distance to the $Z$ resonance pole. Once the quadruplet is formed, the leading-lepton pair is defined as the one with a higher absolute rapidity value, i.e., $|y_{ij}|$. Finally, an additional criterion on the invariant mass of the quadruplet of $m_{4\ell} > 130$ GeV is imposed. 

Similarly, the event dijet is constructed from the two leading jets with the opposite sign of pseudo-rapidity ($\eta$) to imitate the detector-level VBS dijet production, where jets are reconstructed on the opposite side of the detector. Similar to detector-level, the particle-level jets are required to satisfy $|n| < 4.5$, $p_{T,~leading~jet} > 40$ GeV, and $p_{T,~sub-leading~jet} > 30$ GeV. The dijet is required to have a significant rapidity separation of $|\Delta y_{jj}| > 2$ and high invariant mass of $m_{jj} > 300$ GeV to resemble dijet produced in electroweak $ZZ^*(\rightarrow 4 \ell) jj$ production. Table \ref{tab:QuadDijetFidCut} summarizes the requirements to select the particle-level quadruplet and dijet in an event.           
    
\begin{table}[!htbp]
    \caption{Details of the kinematic selections applied to form a particle-level quadruplet and a particle-level dijet in the fiducial volume. 
    \label{tab:QuadDijetFidCut}}
    \begin{tabular}{|| l || c ||}
        \hline
        Selections              &           Cut \\
        \hline\hline
        Lepton Kinematics       & $P_{T,~leading~lepton} > 20 $ GeV\\
                                & $P_{T,~sub-leading~lepton} > 20 $ GeV\\
        \hline 
        Pair Requirement        & $\Delta R_{\ell i,\ell     j} > 0.05 $\\
                                & SF-OC with $\mll > 5$ GeV\\
        \hline
        Quadruplet Requirement  & $2$ pair candidates with smallest $|\mOneTwo  - m_{Z} | + |\mThreeFour    - m_{Z} |$  \\
                                & Leading pair: pair with highest $|y_{ij}|$\\
                                & Sub-leading pair: pair with lowest $|y_{ij}|$\\
                                & $\mFourL > 130 $ GeV\\
        \hline
        Di-jet Requirement      & $p_{T,~leading~jet} > 40$ GeV \\
                                & $|\Delta y_{jj}| > 2 $ \\ 
                                & $m_{jj} > 300$ GeV    \\
        \hline
    \end{tabular}
\end{table}
