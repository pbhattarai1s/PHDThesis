\section{Datasets and Monte Carlo Simulation}
\label{sec:DataSetAndMonteCarlo}

\subsection{LHC Dataset}
\label{subsec:Dataset}

The measurement uses the LHC collision data, the ATLAS Run-$2$ dataset collected by the ATLAS experiment during its operation in $2015$, $2016$, $2017$, and $2018$. This dataset corresponds to proton-proton collisions at the center-of-mass energy of $\sqrt{s} = 13$ TeV and total integrated luminosity of $139 \pm 2.4$ fb$^{-1}$ measured by the LUCID-2 detector \cite{ATLASLuminosityDetector}\cite{ATLASRun2IntegratedLumi}. The LUCID-2 detector is a Cherenkov detector with 16 photo-multipliers (PMT) that provides counts of particles from each colliding bunch. The counts integrated over time periods of about $60$ seconds are called luminosity blocks (LB). The instantaneous luminosity is given by 
\begin{equation}
  \mathcal{L} = \frac{R}{\sigma_{vis}},
\end{equation}
where $R$ is the event rate and $\sigma_{vis}$ is the calibration constant measured during the LHC's special runs, which took place at the beginning of each year of data taking. The calibration constant accounts for the non-linear response of the LUCID-2 detector. The uncertainty on the integrated luminosity is obtained from the combination of the measurements of each year of the LHC run.

Each data-taking run period of the LHC is further divided into sub-periods of one to three weeks that vary in beam and detector conditions. The dataset used in physics analyses is required to satisfy a series of data quality checks discussed in detail in Ref \cite{ATLASRun2DataTaking}. The data passing these requirements collectively form a Good Run List (GRL) and consists of several LB. Figure \ref{fig:InstLuminosity} shows the total integrated luminosity delivered by LHC in the green distribution, recorded by the ATLAS experiment in the yellow distribution and part of the GRL in the blue distribution. The plateaus correspond to the end-of-year shutdowns of LHC, and the slopes correspond to the increasing instantaneous luminosity in different data-taking periods. 

\begin{figure}[!htb]
\centering
\includegraphics[width=.8\linewidth]{figures/AnalysisOverview/IntegratedLumiRun2.pdf}  
  \caption{Total integrated luminosity collected during data taking period in Run-$2$ \cite{ATLASRun2DataTaking}. }
\label{fig:InstLuminosity}
\end{figure}

\subsection{Monte Carlo Samples }
\label{subsec:MCSamples}

As briefly mentioned in Section \ref{sec:Pheno}, the $pp \rightarrow ZZ^* (\rightarrow 4\ell) jj$ events are simulated by MC generators. The first step in generating SM prediction is \textit{event generation} which incorporates the matrix element calculations for the hard-scatter $pp \rightarrow ZZ^* (\rightarrow 4\ell) jj$ process, the parton showering, hadronization, and the effect of the underlying events. The generated events are then \textit{simulated} to interact with the ATLAS material using the Geant4 simulation toolkit following the description in Ref \cite{GEANT4}. The energy deposits of the simulated events in the detectors are then \textit{digitized} where the simulated signal hits are overlayed with cavern background events hits and additional hits from soft QCD events to simulate the effect of the pile-up. Finally, the events are \textit{reconstructed} using the same procedure to reconstruct data objects using detector geometry corresponding to the data-taking period. The final simulations are available in terms of \textit{physics derivations}, which are used in the final analysis. Figure \ref{fig:MCGenerationSchematic} shows a schematic overview of the MC generation. 
\begin{figure}[!htb]
\centering
\includegraphics[width=.3\linewidth]{figures/AnalysisOverview/MCSchematic.png}  
  \caption{Various steps in MC sample generation.}
\label{fig:MCGenerationSchematic}
\end{figure}

Each physics process is simulated using different generation campaigns corresponding to the different conditions of Run-2 ATLAS data-taking periods. As shown in Figure \ref{fig:PileupDiffRuns}, the pile-up distribution is different for the different data-taking periods. The MC-generated events are modified to correctly simulate the effect of pile-up distribution to simulate that of the data. 

\begin{figure}[!htb]
  \centering
  \includegraphics[width=.8\linewidth]{figures/AnalysisOverview/mu_ProfileRun2.pdf}
  \caption{Pile-up distributions in different Run-2 data-taking period.\label{fig:PileupDiffRuns} \cite{ATLASRun2DataTaking}}
\end{figure}

\subsubsection{Signal Samples}
\label{subsubsec:SigSamples}
As discussed in Section \ref{sec:EWKPheno}, two types of interactions, QCD and EWK, give us $pp \rightarrow ZZ^*(\rightarrow 4 \ell) jj$ final state. The two types of QCD process, quark induced $qqZZ$ $[qq \rightarrow ZZ^*(\rightarrow 4 \ell) jj]$ and gluon induced $ggZZ$ $[gg \rightarrow ZZ^* (\rightarrow 4\ell) ]jj$ are simulated using the \textsc{Sherpa} $2.2.2$ MC generator. The parton initiated $qqZZ$ samples corresponding to Figure \ref{fig:ZZjjFeynmanDiag_QCD_qq} are generated with NLO accuracy in QCD up to one additional parton emission and LO accuracy for up to three additional partons emission. The loop-induced $ggZZ$ samples emerging at NNLO in $\alpha_{S}$ corresponding to Figure \ref{fig:ZZjjFeynmanDiag_QCD_gg} are generated using LO-accurate matrix elements for up to one additional parton emission \cite{EventGenWithSherpa}. The generator uses an NNPDF3.0NNLO PDF set evaluated using different measurements from several experiments, such as deep-inelastic inclusive cross-sections measurement from HERA-II, the combined charm data from HERA, jet production, vector boson rapidity and transverse momentum measurements from ATLAS, CMS and LHCb, total cross sections of top quark pair production from ATLAS and CMS and W+c data from CMS \cite{PDFForRunII}. Parton showering is done by \textsc{Sherpa}'s internal algorithm based on Catani–Seymour dipole factorization matrix element \cite{SherpaPS}. The matrix element calculations are matched and merged using the $ME+PS@NLO$ prescription \cite{PSMatching}. The matching process requires the hard jets to match the parton-level quarks and gluons from matrix elements, whereas the merging allows the MC generators to merge two or more low-mass jets. 

An alternative \textsc{MadGraph5} samples produced at NLO accuracy for up to one additional parton emission and LO accuracy for up to three additional parton emission \cite{MADGRAPHNLO} are also used in the measurement for the parton induced $qqZZ$ samples. The generator uses the A14NNPDF23LO PDF set, and the ME is interfaced with \textsc{Pythia8} for parton showering, merging, and matching \cite{Pythia8}. 

The EWK production $qqZZjj$ $[qq \rightarrow ZZ^{*}(\rightarrow 4 \ell) jj]$ is simulated using a \textsc{PowhegV2} generator using an MSTW2008 PDF set with NLO accuracy in QCD and interfaced with \textsc{Pythia8} for parton showering and hadronization \cite{PowhegV2}. An alternative sample at LO accuracy is also used in the measurement from \textsc{MadGraph5} with A14NNPDF23LO PDF set and \textsc{Pythia8} showering \cite{MADGRAPHNLO}. The \textsc{PowhegV2} NLO prediction of electroweak $qqZZjj$ does not contain the contribution from electroweak triboson $VZZ$ processes where two vector bosons decay leptonically and one decay hadronically. The contribution from these electroweak triboson processes is predicted using the {Sherpa} $2.2.2$ MC generator at LO accuracy for up to two additional parton emissions and added to the \textsc{PowhegV2} predictions. Table \ref{tab:SigMC} summarizes the signal MC used in the measurement. 

\begin{table}[!htbp]
\footnotesize
\centering
\begin{tabular}{l l c c c }
\hline\hline
Process & Description & Generator  & PDF & Accuracy\\
\hline \hline
QCD $qqZZ$ &        &        &       &   \\
\multirow{2}{*}{ $q\bar{q} \rightarrow ZZ^{*}( \rightarrow 4\ell) jj$ } 
                    & \multirow{2}{*}{inclusive} & \textsc{Sherpa}$2.2.2$ & NNPDF3.0NNLO & \multirow{2}{*} {$0,1 j @NLO + 2,3 j @LO $} \\ 
        &  & \textsc{MadGraph} & A14NNPDF23LO & \\
        &       &        &       &   \\
\hline
QCD $ggZZ$ loop&        &        &       &   \\
 $gg \rightarrow ZZ^{*} (\rightarrow 4\ell) jj$ &  $m_{4 \ell } > 130$ GeV  & \textsc{Sherpa}$2.2.2$ & NNPDF3.0NNLO & $0,1 j @LO $ \\

&       &        &       &   \\

\hline 
EWK $qqZZjj$ &      &        &       &   \\
\multirow{2}{*}{ $q\bar{q} \rightarrow ZZ^{*}( \rightarrow 4\ell) jj$ } 
                    & \multirow{2}{*}{$m_{4\ell} > 130$ GeV } & \textsc{PowhegV2} & MSTW2008 &  $\ge 2 j$ (EWK) @ NLO QCD \\ 
        &   & \textsc{MadGraph} & A14NNPDF23LO & $\ge 2 j$ (EWK) @LO \\
        &       &        &       &   \\
\hline 
EWK $VZZ$ & & & & \\
$q\bar{q} \rightarrow VZZ^{*} \rightarrow 4\ell jj$ &       &   \textsc{Sherpa}$2.2.2$   &  NNPDF3.0NNLO     & $1,2j @LO$    \\
\hline\hline

\end{tabular}
\normalsize
\caption{List of signal MC samples used in the analysis. Each process consists of three different generation campaigns corresponding to the data-taking conditions of the ATLAS Run2 data-taking periods.\\ \label{tab:SigMC}}
\end{table}

\subsubsection{Background Samples}
\label{subsubsec:BkgSamples}

In addition to the QCD and EWK production discussed above, two other processes, triboson ($WWZ, ~WZZ, ~ZZZ$) and $Z$-bosons production in association with a top quark pair ($t\bar{t}Z$), also contributes to the $ 4\ell jj$ final state. The triboson processes are modeled with \textsc{Sherpa}$2.2.2$ generator at NLO accuracy in QCD for zero or one additional parton emissions and LO accuracy for up to two additional parton emissions. The triboson samples only include the fully leptonic decays of the vector bosons. Therefore, there is no overlap between the background triboson and the signal EWK $qqZZjj$ samples. The $t\bar{t}Z$ processes are modeled by \textsc{Sherpa}$2.2.0$ generator at LO accuracy with up to one additional parton emission using the MEPS@LO set-up \cite{Sherpa220}. The same algorithms as in the QCD $qqZZ$ sample generation are used for parton showering, matching, and merging. The MC simulation of the triboson and $t\bar{t}Z$ samples are subtracted directly from the data. Table \ref{tab:BkgMC} summarizes the details of these samples. 

\begin{table}[!htbp]
\footnotesize
\centering
\begin{tabular}{l l c c c }
\hline\hline
Process & Description & Generator  & PDF & Accuracy\\
\hline \hline
 &      &        &       &   \\
 $pp \rightarrow W^{(*)}W^{(*)}Z^{(*)} \rightarrow 4\ell 2\nu $  & \multirow{3}{*}{inclusive} & \textsc{Sherpa}$2.2.2$ & \multirow{3}{*}{NNPDF3.0NNLO} & \multirow{3}{*}{$0,1 j @NLO + 2 j @LO $} \\ 
 
$pp \rightarrow W^{(*)}Z^{(*)}Z^{(*)} \rightarrow 5\ell 1\nu$  &  & \textsc{Sherpa}$2.2.2$ &   &  \\ 
$pp \rightarrow Z^{(*)} Z^{(*)} Z^{(*)} \rightarrow 6\ell $ &  & \textsc{Sherpa}$2.2.2$ &  &  \\ 
        
\hline 
&       &        &       &   \\
$pp \rightarrow t\bar{t}+Z(\rightarrow 2\ell)$ & $m_{ll} > 5$ GeV & \textsc{Sherpa}$2.2.0$ & NNPDF3.0NNLO & LO \\

\hline\hline
\end{tabular}
\normalsize
\caption{List of background MC samples used in the analysis. Each process consists of three different generation campaigns corresponding to the data-taking conditions of the ATLAS Run2 data-taking periods.\\ \label{tab:BkgMC}}
\end{table}

\subsubsection{Samples for Non-prompt Background}
\label{subsubsec:FakeBkgSamples}
In addition to the triboson and $t\bar{t}$Z samples, the analysis has additional backgrounds coming from events with one or more non-prompt or fake leptons. These non-prompt backgrounds are estimated using a data-driven method discussed in detail in Section \ref{subsec:FakeBackground}. MC samples are used to develop and validate the data-driven non-prompt background estimation procedure. Three sources of events could contribute as a source for non-prompt background events. The first type of events is from a Z-boson production in association with jets $pp \rightarrow Z^{*} (\rightarrow 2\ell) +jets$, which is simulated for both three or more leptons using \textsc{Sherpa}$2.2.1$. The subdominant process is events from $t\bar{t}\rightarrow 2\ell$ production in which both top quarks decay semileptonically, which is simulated with \textsc{Powheg+Pythia8} and uses the A14NNPDF23LO PDF set \cite{PowhegPythia}. The third type of non-prompt backgrounds arises from the WZ production in which both bosons decay leptonically $pp \rightarrow WZ \rightarrow 3 \ell 1\nu $ and is simulated using \textsc{Sherpa}$2.2.2$. Table \ref{tab:FakeBkgMC} summarizes the different processes and MC generators used in various studies related to the data-driven fake factor method to estimate the non-prompt background. A simulated non-prompt background event in the $4\ell jj$ final state could comprise an event with a quadruplet formed from one or more fake leptons from a non-prompt source, such as misidentification and other remaining leptons from these three physics processes.

\begin{table}[!htbp]
\footnotesize
\centering
\begin{tabular}{l l c c c }
\hline\hline
Process & Description & Generator  & PDF & Accuracy\\
\hline \hline
 &      &        &       &   \\
 $pp \rightarrow Z^{*} (\rightarrow 2e)+jets $  & \multirow{3}{*}{inclusive} & \multirow{3}{*}{\textsc{Sherpa}$2.2.2$} & \multirow{3}{*}{NNPDF3.0NNLO} & \multirow{3}{*}{$NLO+2j,LO+4j $} \\ 
 
$pp \rightarrow Z^{*} (\rightarrow 2\mu) +jets $  &  &  &   &  \\ 
$pp \rightarrow Z^{*} (\rightarrow 2\tau) +jets $ &  &  &  &  \\ 
        
\hline 
&       &        &       &   \\
$pp \rightarrow t\bar{t} \rightarrow 2\ell + X $ & inclusive & \textsc{Powheg+Pythia8} & A14NNPDF23LO & LO \\
\hline 
&       &        &       &   \\
$pp \rightarrow WZ \rightarrow 3 \ell 1\nu $ & inclusive & \textsc{Sherpa}$2.2.2$ & NNPDF3.0NNLO & $NLO + 1j, LO+3j $\\
\hline\hline

\end{tabular}
\normalsize
\caption{List of MC samples used in the estimation and validation of the data-driven non-prompt background estimation.\\ \label{tab:FakeBkgMC}}
\end{table}

\subsection{Event Weights}
\label{subsec:EventWt}

The predictions from the MC generators are often generated with a higher effective luminosity than the data to reduce the statistical uncertainties in simulations. Therefore, the raw predictions of the MC are completely unscaled and cannot be compared to the data recorded by the detector directly. Different weights need to be accounted for to scale each generated MC event.

The MC-generated events beyond LO have a generator event weight which is needed once a sufficient number of events are simulated to attain the correct cross-section distribution. Some generated events have negative weight to account for the cancellations caused by interference that arises during matrix element calculation. The generator event weight also accounts for the Sudakow form factors associated with QCD emissions in parton showering. Each MC-generated event first needs to be scaled based on its generator event weight and normalized by the total sum of all the generated event weights. Once the event weight is multiplied by the cross-section and the integrated luminosity of the data, the distribution is correctly normalized. 

A higher-order cross-section calculation might be available for some physics processes. Such predictions are re-weighted by scaling the event weight using the available \textit{k-factor}. Moreover, the simulations of certain physics processes are computationally intensive, and MC generators might impose kinematic filters to simulate these processes. The efficiency of such filters, $\epsilon_{filt}$, are also considered in the event weight. 

A set of corrections in the event weight are related to detector measurements. As discussed in Section \ref{sec:ParticleReconstruction}, the \textit{scale factors (SF)} correct the efficiencies in MC related to the reconstruction, identification, isolation, and trigger to match that of the measured data. These scale factors are applied to the event weight. Finally, the pile-up re-weighting corrects the MC event weights to match the distribution of the average number of interactions per bunch crossing observed in the data and is also considered in the event weight.

Therefore, each event weight in MC can be represented as, 
\begin{equation}
w_{event} = \frac{\sigma~\cdot~\text{k-factor}\cdot~~\epsilon_{filt}~\cdot w_{generator} \cdot {L_{data} \cdot w_{reco}}}{\sum{w_{generator}}},
\label{eqn:EventWeight}
\end{equation}
where $\sigma$ is the cross-section of the process, $w_{generator}$ is the generator event weight, $L_{data}$ is the integrated luminosity from data, and $w_{reco}$ accounts for all detector measurement related re-weighting in detector-level predictions.