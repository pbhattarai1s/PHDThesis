\section{Reconstruction Selection}
\label{sec:ObjReconstruction}

This section summarizes the detector-level phase space selections applied to three physics objects, electrons, muons, and jets used in the measurement. Each physics object of the analysis has two categories: \textit{baseline} and \textit{signal} objects. Physics objects satisfying a set of kinematic selections or looser identification criteria are categorized as \textit{baseline} whereas, the baseline leptons that pass either stricter kinematic selections or additional isolation and track-to-vertex association (TTVA) requirements are \textit{signal}.

\subsection{Electrons}
\label{subsec:ElecRecon}

As discussed in Section \ref{subsec:ParticleRecon_Elec}, electrons are reconstructed by matching the inner detector track (ID) to an energy cluster in the electromagnetic calorimeter. Baseline electron objects are required to satisfy the kinematic selections of $\Pt~ > 7$ GeV $ \&~ |\eta| < 2.47$ and a loose likelihood identification of working point \textit{LHVeryLoose}. To avoid the electrons from pileup, a loose vertex association requirement of $|z_{0}sin\theta| < 0.5 $ mm and an overlap removal discussed in section \ref{subsec:OR} is applied to the baseline electron candidates.

Signal electrons are required to pass a more stringent loose likelihood identification, \textit{LHLooseBL}, which requires at least one hit in the innermost layer of the pixel detector. The signal electrons are distinguished by requiring the baseline electrons to have impact parameter requirements of $d0/ \dZeroSig < 5$ and an isolation working point identification of \textit{LooseVarRad}. Table \ref{tab:Electron_RecoSel} summarizes the several selections imposed to define the baseline and signal electrons.

\begin{table}[ht]
	\centering
		\caption{Definition of the baseline and signal electrons.\label{tab:Electron_RecoSel}}
		\begin{tabular}{|| l || c | c ||}
		\hline
		Selection Category & \textbf{Baseline} & \textbf{Signal} \\
		\hline\hline
		Kinematic cuts & $p_{T}~ > 7$ GeV & $ p_{T}~ > 7$ GeV \\
		            & $|\eta| < 2.47$  &  $|\eta| < 2.47$\\
		\hline  
		Identification & LHVeryLoose & LHLooseBL \\
		\hline 
		Vertex Association & $|z_{0}sin\theta| < 0.5$ mm & $|z_{0}sin\theta|< 0.5$ mm\\
		\hline
		Overlap removal & Lepton-favored & Lepton-favored\\
		\hline
		Isolation Working Point & $-$ & LooseVarRad\\
		\hline 
		Impact Parameters & $-$ & $d_{0}/ \sigma_{d_{0}} < 5$ \\
		\hline
	\end{tabular}
\end{table}

\subsection{Muons}
\label{subsec:MuonRecon}
As discussed in section \ref{subsec:ParticleRecon_Muon}, muons are reconstructed in multiple ways based on information from the inner detector (ID), the muon spectrometer (MS), and the calorimeters. All baseline muons are required to satisfy $ |\eta| < 2.7 $, $p_{T} > 5$ GeV, a loose impact parameter requirements of $|z_{0}sin\theta| < 0.5 $ mm, lepton-favoring overlap removal and \textit{Loose} identification working point. The signal muons are identified by requiring additional isolation identification of \textit{PflowLooseVarRad} and TTVA requirements of $d_{0}/\sigma_{d_{0}} < 3$. Table \ref{tab:muon_baseline_signal} summarizes baseline and signal muons selection requirements.

\begin{table}[ht]
	\centering
		\caption{Definition of the baseline and signal muons.\label{tab:muon_baseline_signal}}
		\begin{tabular}{|| l || c | c ||}
		\hline
		Selection Category & \textbf{Baseline} & \textbf{Signal} \\
		\hline\hline
		Kinematic cuts & $p_{T}~ > 5$ GeV & $p_{T}~ > 5$ GeV \\
					& Calo-tagged $ p_{T}~ > 15$ GeV & Calo-tagged $ p_{T}~ > 15$ GeV \\
		      & $|\eta| < 2.7$ & $|\eta| < 2.7$\\
		\hline
		Identification & Loose & Loose \\
		\hline 
		Vertex Association & $|z_{0}sin\theta| < 0.5$ mm & $|z_{0}sin\theta|< 0.5$ mm\\
		\hline
		Overlap removal & Lepton-favored & Lepton-favored\\
		\hline
		Isolation Working Point & $-$ & PflowLooseVarRad\\
		\hline 
		Impact Parameters & $-$ & $d_{0}/\sigma_{d_{0}} < 3$ \\
		\hline
	\end{tabular}
\end{table}

\subsection{Jets}
\label{subsec:JetRecon}
Jets are reconstructed with the particle flow anti$-K_{T}$ clustering algorithm using a radius parameter of $R = 0.4$ as discussed in section \ref{subsec:ParticleRecon_Jets}. The jets reconstructed using the particle flow algorithm are required to satisfy $p_{T}~ > 15$ GeV, $ |\eta| < 4.5 $ kinematic cuts, and the lepton-favored overlap removal to be classified as baseline jets. Baseline jets satisfying the \textit{Tight} working point of the jet to the vertex tagger tool are classified as signal jets. \textit{Jet-vertex-tagger (JVT)} is applied to the baseline jets with $ |\eta| < 2.4 $ whereas the \textit{forward-jet-vertex-tagger (fJVT)} tool is applied to the baseline jets with $ |\eta| > 2.5 $. Table \ref{tab:jets} summarizes the details of baseline and signal jets selection. 

\begin{table}[ht]
	\centering
	\caption{Definition of the baseline and signal jets.\label{tab:jets}}
		\begin{tabular}{|| l || c | c ||}
		\hline
		Selection Category & \textbf{Baseline} & \textbf{Signal} \\
		\hline\hline
		Kinematic cuts & $\Pt~ > 30$ GeV & $\Pt~ > 30$ GeV \\
			 & $|\eta| < 4.5$ & $|\eta| < 4.5$\\
		\hline 
		Identification & AntiKt4EMPFlow & AntiKt4EMPFlow\\
		\hline
		Overlap removal & Lepton-favored & Lepton-favored\\
		\hline
		Jet-Vertex-Tagger & $-$ & $ |\eta| < 2.4 $ JVT ("Tight")\\
				& $-$ & $|\eta| > 2.5 $ fJVT ("Tight")\\
		\hline
	\end{tabular}
\end{table}

\subsection{Overlap Removal}
\label{subsec:OR}

An \textit{overlap removal} procedure is applied to remove the physics objects reconstructed from the same detector signal. The measurement uses a lepton-favored overlap removal which selects leptons over jets. Overlap removal is an iterative process in which only objects surviving all previous steps are used in the subsequent steps. Table \ref{tab:overlap_removal} summarizes the overlap removal steps, where the $\Delta R$ is the angular separation between objects calculated using rapidity.

\begin{table}[ht]
	\centering
		\caption{Overlap removal used in the analysis. An object removed in one step does not enter into the subsequent step. \label{tab:overlap_removal}}
		\begin{tabular}{|| l || c | c ||}
		\hline
		Remove Object & Accept Object & Overlap Criteria \\
		\hline\hline
		Electron & Electron & Share a track or have overlapping calorimeter cluster.\\
				&		& Keep electron with higher $p_{T}~$\\
		\hline
		Muon & Electron & Share ID track, and the muon is calo-tagged\\
		\hline
		Electron & Muon & Share ID track\\
		\hline
		Jet & Electron & $\Delta R_{e-jet} < 0.2$ \\
		\hline 
		Jet & Muon & $\Delta R_{\mu-jet} < 0.2/$ghost-associated and $N_{jet~tracks} < 3$\\
		\hline
	\end{tabular}
\end{table}